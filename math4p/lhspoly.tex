%% =======================================================
%% (c) Tobias Schoofs
%% =======================================================
%% Math 4 Programmers - The Inverse Element
%% =======================================================

% Plain Style
\documentclass[tikz]{scrreprt}

% Springer Style
%\documentclass[envcountsame]{llncs}

%% ODER: format ==         = "\mathrel{==}"
%% ODER: format /=         = "\neq "
%
%
\makeatletter
\@ifundefined{lhs2tex.lhs2tex.sty.read}%
  {\@namedef{lhs2tex.lhs2tex.sty.read}{}%
   \newcommand\SkipToFmtEnd{}%
   \newcommand\EndFmtInput{}%
   \long\def\SkipToFmtEnd#1\EndFmtInput{}%
  }\SkipToFmtEnd

\newcommand\ReadOnlyOnce[1]{\@ifundefined{#1}{\@namedef{#1}{}}\SkipToFmtEnd}
\usepackage{amstext}
\usepackage{amssymb}
\usepackage{stmaryrd}
\DeclareFontFamily{OT1}{cmtex}{}
\DeclareFontShape{OT1}{cmtex}{m}{n}
  {<5><6><7><8>cmtex8
   <9>cmtex9
   <10><10.95><12><14.4><17.28><20.74><24.88>cmtex10}{}
\DeclareFontShape{OT1}{cmtex}{m}{it}
  {<-> ssub * cmtt/m/it}{}
\newcommand{\texfamily}{\fontfamily{cmtex}\selectfont}
\DeclareFontShape{OT1}{cmtt}{bx}{n}
  {<5><6><7><8>cmtt8
   <9>cmbtt9
   <10><10.95><12><14.4><17.28><20.74><24.88>cmbtt10}{}
\DeclareFontShape{OT1}{cmtex}{bx}{n}
  {<-> ssub * cmtt/bx/n}{}
\newcommand{\tex}[1]{\text{\texfamily#1}}	% NEU

\newcommand{\Sp}{\hskip.33334em\relax}


\newcommand{\Conid}[1]{\mathit{#1}}
\newcommand{\Varid}[1]{\mathit{#1}}
\newcommand{\anonymous}{\kern0.06em \vbox{\hrule\@width.5em}}
\newcommand{\plus}{\mathbin{+\!\!\!+}}
\newcommand{\bind}{\mathbin{>\!\!\!>\mkern-6.7mu=}}
\newcommand{\rbind}{\mathbin{=\mkern-6.7mu<\!\!\!<}}% suggested by Neil Mitchell
\newcommand{\sequ}{\mathbin{>\!\!\!>}}
\renewcommand{\leq}{\leqslant}
\renewcommand{\geq}{\geqslant}
\usepackage{polytable}

%mathindent has to be defined
\@ifundefined{mathindent}%
  {\newdimen\mathindent\mathindent\leftmargini}%
  {}%

\def\resethooks{%
  \global\let\SaveRestoreHook\empty
  \global\let\ColumnHook\empty}
\newcommand*{\savecolumns}[1][default]%
  {\g@addto@macro\SaveRestoreHook{\savecolumns[#1]}}
\newcommand*{\restorecolumns}[1][default]%
  {\g@addto@macro\SaveRestoreHook{\restorecolumns[#1]}}
\newcommand*{\aligncolumn}[2]%
  {\g@addto@macro\ColumnHook{\column{#1}{#2}}}

\resethooks

\newcommand{\onelinecommentchars}{\quad-{}- }
\newcommand{\commentbeginchars}{\enskip\{-}
\newcommand{\commentendchars}{-\}\enskip}

\newcommand{\visiblecomments}{%
  \let\onelinecomment=\onelinecommentchars
  \let\commentbegin=\commentbeginchars
  \let\commentend=\commentendchars}

\newcommand{\invisiblecomments}{%
  \let\onelinecomment=\empty
  \let\commentbegin=\empty
  \let\commentend=\empty}

\visiblecomments

\newlength{\blanklineskip}
\setlength{\blanklineskip}{0.66084ex}

\newcommand{\hsindent}[1]{\quad}% default is fixed indentation
\let\hspre\empty
\let\hspost\empty
\newcommand{\NB}{\textbf{NB}}
\newcommand{\Todo}[1]{$\langle$\textbf{To do:}~#1$\rangle$}

\EndFmtInput
\makeatother
%

%% =======================================================
%% (c) Tobias Schoofs
%% =======================================================
%% Commands 4 Programmers
%% =======================================================

%include lhs2TeX.fmt
%include lhs2TeX.sty

\usepackage[pdftex]{graphicx}
\usepackage{ucs}
\usepackage[utf8x]{inputenc} 
\usepackage{tabto}
%\usepackage[T1]{fontenc} 
\usepackage[portuguese, english]{babel}
\usepackage{amsfonts}

\usepackage{amsmath}
\usepackage{amssymb}
\usepackage{amsthm}
\usepackage{amscd}

\usepackage{siunitx}

\usepackage{tikz}
\usepackage{pgfplots}
%\usetikzlibrary{graphs,graphdrawing,arrows.meta}
\usetikzlibrary{graphs} % graphdrawing,arrows.meta}
%\usegdlibrary{trees}

\long\def\ignore#1{}

\newcommand{\acronym}[1]{\textsc{#1}}

\newcommand{\term}[1]{\textit{#1}}
\newcommand{\latin}[1]{\textit{#1}}
\newcommand{\speech}[1]{\textit{#1}}

\newcommand{\ie}{\textit{i.e.}}
\newcommand{\eg}{\textit{e.g.}}
\newcommand{\etc}{\textit{etc.}}
\newcommand{\viz}{\textit{viz.}}

\newcommand{\haskell}[1]{{\ttfamily #1}}

\newcommand{\Rom}[1]{\uppercase\expandafter{\romannumeral #1\relax}}

\DeclareRobustCommand{\stirlingOne}{\genfrac[]{0pt}{}}
\DeclareRobustCommand{\stirlingTwo}{\genfrac\{\}{0pt}{}}
\DeclareRobustCommand{\eulerianOne}{\genfrac<>{0pt}{}}
\newcommand{\eulerianTwo}[2]{\left\langle\!\!\left\langle{#1 \atop #2}\right\rangle\!\!\right\rangle}

\newcommand{\connect}[2]{
  \draw [-,color=black] (#1) to (#2)
}

% \renewcommand{\gcd}{\textsc{gcd}}


\begin{document}
\setlength{\parindent}{0pt}
\setlength{\parskip}{8pt}

\chapter{Polynomials} % c07
\section{Numeral Systems}
\ignore{
\begingroup\par\noindent\advance\leftskip\mathindent\(
\begin{pboxed}\SaveRestoreHook
\column{B}{@{}>{\hspre}l<{\hspost}@{}}%
\column{3}{@{}>{\hspre}l<{\hspost}@{}}%
\column{E}{@{}>{\hspre}l<{\hspost}@{}}%
\>[B]{}\mathbf{module}\;\Conid{NumSystem}{}\<[E]%
\\
\>[B]{}\mathbf{where}{}\<[E]%
\\
\>[B]{}\hsindent{3}{}\<[3]%
\>[3]{}\mathbf{import}\;\Conid{\Conid{System}.Random}\;(\Varid{randomRIO}){}\<[E]%
\\
\>[B]{}\hsindent{3}{}\<[3]%
\>[3]{}\mathbf{import}\;\Conid{Natural}{}\<[E]%
\\
\>[B]{}\hsindent{3}{}\<[3]%
\>[3]{}\mathbf{import}\;\Conid{Quoz}{}\<[E]%
\\
\>[B]{}\hsindent{3}{}\<[3]%
\>[3]{}\mathbf{import}\;\Conid{Real}{}\<[E]%
\ColumnHook
\end{pboxed}
\)\par\noindent\endgroup\resethooks
}

A numeral system consists of a finite set of digits, $D$,
and a base, $b$, for which $b=|D|$, \ie\ $b$ is the cardinality of $D$.
The binary system, for instance, 
uses the digits $D=\lbrace 0,1\rbrace$.
The cardinality of $D$ in this case, hence, is 2.
The decimal system uses the digits $D=\lbrace 0\dots 9\rbrace$ and,
thus, has the base $b=10$.
The hexadecimal system uses the digits $D=\lbrace 0\dots 15\rbrace$,
often given as $D=\lbrace 0\dots 9,a,b,c,d,e,f\rbrace$,
and, therefore, has the base $b=16$.

Numbers in any numeral system are usually represented as strings
of digits. The string

\[
10101010,
\]

for instance, may represent a number in the binary system.
(It could be a number in decimal format, too, though.)
The string 

\[
170,
\]

by contrast, cannot be a binary number, because
it contains the digit 7, which is not element of $D$
in the binary system.
It can represent a decimal (or a hexadecimal number).
The string

\[
aa,
\]

can represent a number in the hexadecimal system
(but not one of in the binary or decimal system).

We interpret such a string, \ie\ convert it
to the decimal system, by rewriting it 
as a formula of the form:

\[
a_nb^n + a_{n-1}b^{n-1} + \dots + a_0b^0,
\]

where $a_i$ are the digits that appear in the string,
$b$ is the base and $n$ is position of the left-most digit
starting to count with 0 on the right-hand side of the string.
The string $10101010$ in binary notation,hence, is interpreted as

\[
1\times 2^7 + 0\times 2^6 + 1\times 2^5 + 0\times 2^4 + 
1\times 2^3 + 0\times 2^2 + 1\times 2^1 + 0\times 2^0,
\]

which can be simplified to

\[
2^7 + 2^5 + 2^3 + 2,
\]

which, in its turn, is

\[
128 + 32 + 8 + 2 = 170.
\]

The string 170 in decimal notation is interpreted as

\[
10^2 + 7\times 10 = 170.
\]

Interpreting a string in the notation it is written in
yields just that string.
The string $aa$ in hexadecimal notation is interpreted as

\[
a\times 16 + a.
\]

The digit $a$, however, is just 10. We, hence, get the equation

\[
10\times 16 + 10 = 160 + 10 = 170.
\]

What do we get, when we relax some of the constraints
defining a numeral system?
Instead of using a finite set of digits,
we could use a number field, $F$, (finite or infinite)
so that any member of that field qualifies as coefficient
in the formulas we used above to interpret numbers
in the decimal system. We would then relax the rule
that the base must be the cardinality of the field.
Instead, we allow any member $x$ of the field 
to serve as a base.
Formulas we get from those new rules would follow the recipe:

\[
a_nx^n + a_{n-1}x^{n-1} + \dots + a_0x^0
\]

or shorter:

\[
\sum_{i=0}^n{a_ix^i}
\]

with $a_i, x \in F$.

Such beasts are indeed well-known. They are very prominent, in fact,
and their name is \term{polynomial}.

The name \emph{poly}nomial stems from the fact
that polynomials may be composed of many terms;
a monomial, by contrast, is a polynomial 
that consists of only one term.
For instance,

\[
5x^2
\]

is a monomial. A binomial is a polynomial
that consists of two terms. This is
an example of a binomial:

\[
x^5 + 2x.
\]

There is nothing special about
monomials and binomials, at least nothing
that would affect their definition as polynomials.
Monomials and binomials are just
polynomials that happen to have only
one or, respectively, two terms.

Polynomials share many properties with numbers.
Like numbers, arithmetic, including
addition, subtraction, multiplication and division
as well as exponentiation, can be defined over polynomials. 
In some cases, numbers reveal their close relation
to polynomials. The binomial theorem states,
for instance, that a product of the form 

\[
(a+b)(a+b)
\]

translates to a formula involving binomial coefficients:

\[
a^2 + 2ab + b^2.
\]

We can interpret this formula as the product 
of the polynomial $x+a$:

\[
(x+a)(x+a),
\]

which yields just another polynomial:

\[
x^2 + 2ax + a^2
\]

Let us replace $a$ for the number 3
and fix $x=10$. We get:

\begin{equation}
(10+3)(10+3) = 10^2 + 2\times 3\times 10 + 3^2 = 100 + 60 + 9 = 169,
\end{equation}

which is just the result of the multiplication $13\times 13$.
Usually, it is harder to recognise this kind of relations 
numbers have with the binomial theorem (and, hence, with polynomials),
because most binomial coefficients are too big to be represented
by a single-digit number. Already in the product $14\times 14$,
the binomial coefficients are hidden. Let us look at this multiplication
treated as the polynomial $(x+a)$ with $x=10$ and $a=4$:

\[
(10 + 4) (10 + 4) = 
10^2 + 2\times 4\times 10 + 4^2 =
100 + 2\times 40 + 16.
\]

When we look at the resulting number, we do not recognise
the binomial coefficient anymore -- they are \emph{carried} away:
$100 + 2\times 40 + 16 = 100 + 80 + 16 = 196$.

Indeed, polynomials are not numbers.
Those are different concepts.
Another important difference is that polynomials do not establish
a clear order. For any two distinct numbers, we can clearly say
which of the two is the greater and which is the smaller one.
We cannot decide that based on the formula of the polynomial alone.
One way to decide quickly which of two numbers is the grater one
is to look at the number of their digits. The one with more digits
is necessarily the greater one. In any numeral system it holds that:

\[
a_3b^3 + a_2b^2 + a_1b + a_0 > c_2b^2 + c_1b + c_0
\]

independent of the values of the $a$s and the $c$s.
For polynomials, this is not true. Consider the following example:

\[
x^3 + x^2 + x + 1 > 100x^2?
\]

For $x=10$, the left-hand side of the inequation is
$1000 + 100 + 10 + 1 = 1111$;
the right-hand side, however, is $100\times 100 = 10000$.

In spite of such differences, we can represent polynomials
very similar to how we represented numbers,
namely as a list of coefficients. This is a valid
implementation in Haskell:

\begin{minipage}{\textwidth}
\begingroup\par\noindent\advance\leftskip\mathindent\(
\begin{pboxed}\SaveRestoreHook
\column{B}{@{}>{\hspre}l<{\hspost}@{}}%
\column{3}{@{}>{\hspre}l<{\hspost}@{}}%
\column{5}{@{}>{\hspre}l<{\hspost}@{}}%
\column{E}{@{}>{\hspre}l<{\hspost}@{}}%
\>[3]{}\mathbf{type}\;\Conid{Poly}\;\Varid{a}\mathrel{=}\Conid{P}\;[\mskip1.5mu \Varid{a}\mskip1.5mu]{}\<[E]%
\\
\>[3]{}\hsindent{2}{}\<[5]%
\>[5]{}\mathbf{deriving}\;(\Conid{Show}){}\<[E]%
\ColumnHook
\end{pboxed}
\)\par\noindent\endgroup\resethooks
\end{minipage}

We add a safe constructor:

\begin{minipage}{\textwidth}
\begingroup\par\noindent\advance\leftskip\mathindent\(
\begin{pboxed}\SaveRestoreHook
\column{B}{@{}>{\hspre}l<{\hspost}@{}}%
\column{3}{@{}>{\hspre}l<{\hspost}@{}}%
\column{5}{@{}>{\hspre}l<{\hspost}@{}}%
\column{12}{@{}>{\hspre}l<{\hspost}@{}}%
\column{23}{@{}>{\hspre}l<{\hspost}@{}}%
\column{E}{@{}>{\hspre}l<{\hspost}@{}}%
\>[3]{}\Varid{poly}\mathbin{::}(\Conid{Eq}\;\Varid{a},\Conid{Num}\;\Varid{a})\Rightarrow [\mskip1.5mu \Varid{a}\mskip1.5mu]\to \Conid{Poly}\;\Varid{a}{}\<[E]%
\\
\>[3]{}\Varid{poly}\;[\mskip1.5mu \mskip1.5mu]\mathrel{=}\Varid{error}\;\text{\tt \char34 not~a~polynomial\char34}{}\<[E]%
\\
\>[3]{}\Varid{poly}\;\Varid{as}\mathrel{=}\Conid{P}\;(\Varid{cleanz}\;\Varid{as}){}\<[E]%
\\[\blanklineskip]%
\>[3]{}\Varid{cleanz}\mathbin{::}(\Conid{Eq}\;\Varid{a},\Conid{Num}\;\Varid{a})\Rightarrow [\mskip1.5mu \Varid{a}\mskip1.5mu]\to [\mskip1.5mu \Varid{a}\mskip1.5mu]{}\<[E]%
\\
\>[3]{}\Varid{cleanz}\;\Varid{xs}\mathrel{=}\Varid{reverse}\mathbin{\$}\Varid{go}\;(\Varid{reverse}\;\Varid{xs}){}\<[E]%
\\
\>[3]{}\hsindent{2}{}\<[5]%
\>[5]{}\mathbf{where}\;{}\<[12]%
\>[12]{}\Varid{go}\;[\mskip1.5mu \mskip1.5mu]{}\<[23]%
\>[23]{}\mathrel{=}[\mskip1.5mu \mskip1.5mu]{}\<[E]%
\\
\>[12]{}\Varid{go}\;[\mskip1.5mu \mathrm{0}\mskip1.5mu]{}\<[23]%
\>[23]{}\mathrel{=}[\mskip1.5mu \mathrm{0}\mskip1.5mu]{}\<[E]%
\\
\>[12]{}\Varid{go}\;(\mathrm{0}\mathbin{:}\Varid{xs}){}\<[23]%
\>[23]{}\mathrel{=}\Varid{go}\;\Varid{xs}{}\<[E]%
\\
\>[12]{}\Varid{go}\;\Varid{xs}{}\<[23]%
\>[23]{}\mathrel{=}\Varid{xs}{}\<[E]%
\ColumnHook
\end{pboxed}
\)\par\noindent\endgroup\resethooks
\end{minipage}

The constructor makes sure that the resulting polynomial
has at least one coefficient and that all the coefficients
are actually numbers and comparable for equality.
The function \ensuremath{\Varid{cleanz}} called in the constructor
removes leading zeros (which are redundant), just as we did 
when we defined natural numbers.
But note that we reverse, first, the list of coefficients
passed to \ensuremath{\Varid{go}} and, second, the result of \ensuremath{\Varid{go}}.
This means that we store the coefficients from left to right
in ascending order. Usually, we write polynomials out
in descending order of their weight, \ie\:

\[
x^n + x^{n-1} + \dots + x^0.
\]

But, here, we store them in the order:

\[
x^0 + x^1 + \dots + x^{n-1} + x^n.
\]

The following function gets the list of coefficients back:

\begin{minipage}{\textwidth}
\begingroup\par\noindent\advance\leftskip\mathindent\(
\begin{pboxed}\SaveRestoreHook
\column{B}{@{}>{\hspre}l<{\hspost}@{}}%
\column{3}{@{}>{\hspre}l<{\hspost}@{}}%
\column{E}{@{}>{\hspre}l<{\hspost}@{}}%
\>[3]{}\Varid{coeffs}\mathbin{::}\Conid{Poly}\;\Varid{a}\to [\mskip1.5mu \Varid{a}\mskip1.5mu]{}\<[E]%
\\
\>[3]{}\Varid{coeffs}\;(\Conid{P}\;\Varid{as})\mathrel{=}\Varid{as}{}\<[E]%
\ColumnHook
\end{pboxed}
\)\par\noindent\endgroup\resethooks
\end{minipage}

Here is a function to pretty-print polynomials:

\begin{minipage}{\textwidth}
\begingroup\par\noindent\advance\leftskip\mathindent\(
\begin{pboxed}\SaveRestoreHook
\column{B}{@{}>{\hspre}l<{\hspost}@{}}%
\column{3}{@{}>{\hspre}l<{\hspost}@{}}%
\column{5}{@{}>{\hspre}l<{\hspost}@{}}%
\column{12}{@{}>{\hspre}l<{\hspost}@{}}%
\column{16}{@{}>{\hspre}l<{\hspost}@{}}%
\column{30}{@{}>{\hspre}l<{\hspost}@{}}%
\column{35}{@{}>{\hspre}l<{\hspost}@{}}%
\column{38}{@{}>{\hspre}c<{\hspost}@{}}%
\column{38E}{@{}l@{}}%
\column{41}{@{}>{\hspre}l<{\hspost}@{}}%
\column{52}{@{}>{\hspre}l<{\hspost}@{}}%
\column{E}{@{}>{\hspre}l<{\hspost}@{}}%
\>[3]{}\Varid{pretty}\mathbin{::}(\Conid{Num}\;\Varid{a},\Conid{Show}\;\Varid{a},\Conid{Eq}\;\Varid{a})\Rightarrow \Conid{Poly}\;\Varid{a}\to \Conid{String}{}\<[E]%
\\
\>[3]{}\Varid{pretty}\;\Varid{p}\mathrel{=}\Varid{go}\;(\Varid{reverse}\mathbin{\$}\Varid{weigh}\;\Varid{p}){}\<[E]%
\\
\>[3]{}\hsindent{2}{}\<[5]%
\>[5]{}\mathbf{where}\;{}\<[12]%
\>[12]{}\Varid{go}\;{}\<[16]%
\>[16]{}[\mskip1.5mu \mskip1.5mu]\mathrel{=}\text{\tt \char34 \char34}{}\<[E]%
\\
\>[12]{}\Varid{go}\;{}\<[16]%
\>[16]{}((\Varid{i},\Varid{c})\mathbin{:}\Varid{cs})\mathrel{=}{}\<[30]%
\>[30]{}\mathbf{let}\;{}\<[35]%
\>[35]{}\Varid{x}{}\<[38]%
\>[38]{}\mid {}\<[38E]%
\>[41]{}\Varid{i}\equiv \mathrm{0}{}\<[52]%
\>[52]{}\mathrel{=}\text{\tt \char34 \char34}{}\<[E]%
\\
\>[38]{}\mid {}\<[38E]%
\>[41]{}\Varid{i}\equiv \mathrm{1}{}\<[52]%
\>[52]{}\mathrel{=}\text{\tt \char34 x\char34}{}\<[E]%
\\
\>[38]{}\mid {}\<[38E]%
\>[41]{}\Varid{otherwise}{}\<[52]%
\>[52]{}\mathrel{=}\text{\tt \char34 x\char94 \char34}\plus \Varid{show}\;\Varid{i}{}\<[E]%
\\
\>[35]{}\Varid{t}{}\<[38]%
\>[38]{}\mid {}\<[38E]%
\>[41]{}\Varid{c}\equiv \mathrm{1}{}\<[52]%
\>[52]{}\mathrel{=}\Varid{x}{}\<[E]%
\\
\>[38]{}\mid {}\<[38E]%
\>[41]{}\Varid{otherwise}{}\<[52]%
\>[52]{}\mathrel{=}\Varid{show}\;\Varid{c}\plus \Varid{x}{}\<[E]%
\\
\>[35]{}\Varid{o}{}\<[38]%
\>[38]{}\mid {}\<[38E]%
\>[41]{}\Varid{null}\;\Varid{cs}{}\<[52]%
\>[52]{}\mathrel{=}\text{\tt \char34 \char34}{}\<[E]%
\\
\>[38]{}\mid {}\<[38E]%
\>[41]{}\Varid{otherwise}{}\<[52]%
\>[52]{}\mathrel{=}\text{\tt \char34 ~+~\char34}{}\<[E]%
\\
\>[30]{}\mathbf{in}\;\mathbf{if}\;\Varid{c}\equiv \mathrm{0}\;\mathbf{then}\;\Varid{go}\;\Varid{cs}\;\mathbf{else}\;\Varid{t}\plus \Varid{o}\plus \Varid{go}\;\Varid{cs}{}\<[E]%
\\[\blanklineskip]%
\>[3]{}\Varid{weigh}\mathbin{::}(\Conid{Num}\;\Varid{a})\Rightarrow \Conid{Poly}\;\Varid{a}\to [\mskip1.5mu (\Conid{Integer},\Varid{a})\mskip1.5mu]{}\<[E]%
\\
\>[3]{}\Varid{weigh}\;(\Conid{P}\;[\mskip1.5mu \mskip1.5mu])\mathrel{=}[\mskip1.5mu \mskip1.5mu]{}\<[E]%
\\
\>[3]{}\Varid{weigh}\;(\Conid{P}\;\Varid{as})\mathrel{=}(\Varid{zip}\;[\mskip1.5mu \mathrm{0}\mathinner{\ldotp\ldotp}\mskip1.5mu]\;\Varid{as}){}\<[E]%
\ColumnHook
\end{pboxed}
\)\par\noindent\endgroup\resethooks
\end{minipage}

The function demonstrates how we actually 
interpret the list of coefficients.
We first \ensuremath{\Varid{weigh}} them by zipping the list of coefficients 
with a list of integers starting at 0.
One could say: we count the coefficients.
Note that we start with 0, so that
the first coefficient gets the weight 0, 
the second gets the weight 1 and so on.
That, again, reflects our descending
ordering of coefficients.

The reversed weighted list is then passed to \ensuremath{\Varid{go}},
which does the actual printing.
We first determine the substring describing $x$:
if $i$, the weight, is 0, we do not want to write the $x$,
since $x^0=1$. If $i=1$, we just write $x$.
Otherwise we write $x^i$.

Then we determine the term composed of coefficient and $x$.
If the coefficient, $c$ is 1, we just write $x$;
otherwise, we concatenate $c$ with $x$.
Note, however, that we later consider an additional case,
namely, when $c=0$. In this case, we ignore the whole term.

We still consider the operation. 
If the remainder of the list is \ensuremath{\Varid{null}}, \ie\ 
we are now handling the last term, \ensuremath{\Varid{o}} is the empty string.
Otherwise, it is the plus symbol.
Here is room for improvement:
when the coefficient is negative, we do not really need
the operation, since we then write $+ -cx$.
Nicer would be to write only $-cx$.

Finally, we put everything together concatenating a string
composed of term, operation and \ensuremath{\Varid{go}} applied on the remainder 
of the list. 

Here is a list of polynomials and how they are
represented with our Haksell type:

\begin{center}
\begingroup
\renewcommand{\arraystretch}{1.5}
\begin{tabular}{|c||c|}
\hline
$x^2 + x + 1$ & \ensuremath{\Varid{poly}\;[\mskip1.5mu \mathrm{1},\mathrm{1},\mathrm{1}\mskip1.5mu]}\\\hline
$5x^5 + 4x^4 + 3x^3 + 2x^2 + x$ &
\ensuremath{\Varid{poly}\;[\mskip1.5mu \mathrm{0},\mathrm{1},\mathrm{2},\mathrm{3},\mathrm{4},\mathrm{5}\mskip1.5mu]}\\\hline
$5x^4 + 4x^3 + 3x^2 + 2x + 1$  &
\ensuremath{\Varid{poly}\;[\mskip1.5mu \mathrm{1},\mathrm{2},\mathrm{3},\mathrm{4},\mathrm{5}\mskip1.5mu]}\\\hline
$5x^4 + 3x^2 + 1$  &
\ensuremath{\Varid{poly}\;[\mskip1.5mu \mathrm{1},\mathrm{0},\mathrm{3},\mathrm{0},\mathrm{5}\mskip1.5mu]}\\\hline
\end{tabular}
\endgroup
\end{center}

An important concept related to polynomials is the 
\term{degree}. The degree is a measurement of the
\emph{size} of the polynomial. In concrete terms,
it is the greatest exponent in the polynomial.
For us, it is the weight of the right-most element
in the polynomial or, much simpler, the length
of the list of coefficients minus one -- since,
we start with zero!
The following function computes the degree
of a given polynomial:

\begin{minipage}{\textwidth}
\begingroup\par\noindent\advance\leftskip\mathindent\(
\begin{pboxed}\SaveRestoreHook
\column{B}{@{}>{\hspre}l<{\hspost}@{}}%
\column{3}{@{}>{\hspre}l<{\hspost}@{}}%
\column{E}{@{}>{\hspre}l<{\hspost}@{}}%
\>[3]{}\Varid{degree}\mathbin{::}\Conid{Poly}\;\Varid{a}\to \Conid{Int}{}\<[E]%
\\
\>[3]{}\Varid{degree}\;(\Conid{P}\;\Varid{as})\mathrel{=}\Varid{length}\;\Varid{as}\mathbin{-}\mathrm{1}{}\<[E]%
\ColumnHook
\end{pboxed}
\)\par\noindent\endgroup\resethooks
\end{minipage}

Note, by the way, that polynomials of degree 0,
those with only one trivial term, 
are just constant numbers.

Finally, here is a useful function that 
creates random polynomials with \ensuremath{\Conid{Natural}}
coefficients:

\begin{minipage}{\textwidth}
\begingroup\par\noindent\advance\leftskip\mathindent\(
\begin{pboxed}\SaveRestoreHook
\column{B}{@{}>{\hspre}l<{\hspost}@{}}%
\column{3}{@{}>{\hspre}l<{\hspost}@{}}%
\column{5}{@{}>{\hspre}l<{\hspost}@{}}%
\column{22}{@{}>{\hspre}l<{\hspost}@{}}%
\column{E}{@{}>{\hspre}l<{\hspost}@{}}%
\>[3]{}\Varid{randomPoly}\mathbin{::}\Conid{Natural}\to \Conid{Int}\to \Conid{IO}\;(\Conid{Poly}\;\Conid{Natural}){}\<[E]%
\\
\>[3]{}\Varid{randomPoly}\;\Varid{n}\;\Varid{d}\mathrel{=}\mathbf{do}{}\<[E]%
\\
\>[3]{}\hsindent{2}{}\<[5]%
\>[5]{}\Varid{cs}\leftarrow \Varid{cleanz}\mathbin{<\$>}\Varid{mapM}\;(\mathbin{\char92 \char95 }\to \Varid{randomCoeff}\;\Varid{n})\;[\mskip1.5mu \mathrm{1}\mathinner{\ldotp\ldotp}\Varid{d}\mskip1.5mu]{}\<[E]%
\\
\>[3]{}\hsindent{2}{}\<[5]%
\>[5]{}\mathbf{if}\;\Varid{length}\;\Varid{cs}\mathbin{<}\Varid{d}\;\mathbf{then}\;\Varid{randomPoly}\;\Varid{n}\;\Varid{d}{}\<[E]%
\\
\>[5]{}\hsindent{17}{}\<[22]%
\>[22]{}\mathbf{else}\;\Varid{return}\;(\Conid{P}\;\Varid{cs}){}\<[E]%
\\[\blanklineskip]%
\>[3]{}\Varid{randomCoeff}\mathbin{::}\Conid{Natural}\to \Conid{IO}\;\Conid{Natural}{}\<[E]%
\\
\>[3]{}\Varid{randomCoeff}\;\Varid{n}\mathrel{=}\Varid{randomNatural}\;(\mathrm{0},\Varid{n}\mathbin{-}\mathrm{1}){}\<[E]%
\ColumnHook
\end{pboxed}
\)\par\noindent\endgroup\resethooks
\end{minipage}

The function receives a \ensuremath{\Conid{Natural}} and an \ensuremath{\Conid{Int}}.
The \ensuremath{\Conid{Int}} indicates the degree of the polynomial
we want to obtain. The \ensuremath{\Conid{Natural}} is used to
restrict the size of the coefficients we want
to see in the polynomial.
In \ensuremath{\Varid{randomCoeff}}, we use the \ensuremath{\Varid{randomNatural}}
defined in the previous chapter to generate 
a random number between 0 and $n-1$. You might
suspect already where that will lead us:
to polynomials modulo some number.
But before we get there, 
we will study polynomial arithmetic.
\section{Polynomial Arithmetic}
\ignore{
\begingroup\par\noindent\advance\leftskip\mathindent\(
\begin{pboxed}\SaveRestoreHook
\column{B}{@{}>{\hspre}l<{\hspost}@{}}%
\column{3}{@{}>{\hspre}l<{\hspost}@{}}%
\column{E}{@{}>{\hspre}l<{\hspost}@{}}%
\>[B]{}\mathbf{module}\;\Conid{PolyArith}{}\<[E]%
\\
\>[B]{}\mathbf{where}{}\<[E]%
\\
\>[B]{}\hsindent{3}{}\<[3]%
\>[3]{}\mathbf{import}\;\Conid{Natural}{}\<[E]%
\\
\>[B]{}\hsindent{3}{}\<[3]%
\>[3]{}\mathbf{import}\;\Conid{Quoz}{}\<[E]%
\\
\>[B]{}\hsindent{3}{}\<[3]%
\>[3]{}\mathbf{import}\;\Conid{Real}{}\<[E]%
\\
\>[B]{}\hsindent{3}{}\<[3]%
\>[3]{}\mathbf{import}\;\Conid{NumSystem}{}\<[E]%
\\
\>[B]{}\hsindent{3}{}\<[3]%
\>[3]{}\mathbf{import}\;\Conid{Modular}{}\<[E]%
\ColumnHook
\end{pboxed}
\)\par\noindent\endgroup\resethooks
}

We start with addition and subtraction,
which, in German, are summarised by
the beautiful word \term{strichrechnung}
meaning literally ``dash calculation'' as
opposed to \term{punkt\-rech\-nung} or ``dot calculation'',
which would be multiplication and division.

Polynomial \term{strichrechnung} is easy.
Key is to realise that the structure of polynomials 
is already defined by \term{strichrechnung}:
it is composed of terms each of which is a product
of some number and a power of $x$.
When we add (or subtract) two polynomials, we just sort them
according to the exponents of their terms
and add (or subtract) terms with equal exponents:

\begin{equation}
\begin{array}{crcrcccr}
  & ax^n     & + & bx^{n-1}     & + & \dots & + & c\\
+ & dx^n     & + & ex^{n-1}     & + & \dots & + & f\\
= & (a+d)x^n & + & (b+e)x^{n-1} & + & \dots & + & c+f
\end{array}
\end{equation}

With our polynomial representation, it is easy 
to implement this kind of operation. One might think
it was designed especially to support addition and
subtraction. Here is a valid implementation:

\begin{minipage}{\textwidth}
\begingroup\par\noindent\advance\leftskip\mathindent\(
\begin{pboxed}\SaveRestoreHook
\column{B}{@{}>{\hspre}l<{\hspost}@{}}%
\column{3}{@{}>{\hspre}l<{\hspost}@{}}%
\column{5}{@{}>{\hspre}l<{\hspost}@{}}%
\column{12}{@{}>{\hspre}l<{\hspost}@{}}%
\column{25}{@{}>{\hspre}l<{\hspost}@{}}%
\column{30}{@{}>{\hspre}c<{\hspost}@{}}%
\column{30E}{@{}l@{}}%
\column{33}{@{}>{\hspre}l<{\hspost}@{}}%
\column{E}{@{}>{\hspre}l<{\hspost}@{}}%
\>[3]{}\Varid{add}\mathbin{::}(\Conid{Num}\;\Varid{a},\Conid{Eq}\;\Varid{a})\Rightarrow \Conid{Poly}\;\Varid{a}\to \Conid{Poly}\;\Varid{a}\to \Conid{Poly}\;\Varid{a}{}\<[E]%
\\
\>[3]{}\Varid{add}\mathrel{=}\Varid{strich}\;(\mathbin{+}){}\<[E]%
\\[\blanklineskip]%
\>[3]{}\Varid{sub}\mathbin{::}(\Conid{Num}\;\Varid{a},\Conid{Eq}\;\Varid{a})\Rightarrow \Conid{Poly}\;\Varid{a}\to \Conid{Poly}\;\Varid{a}\to \Conid{Poly}\;\Varid{a}{}\<[E]%
\\
\>[3]{}\Varid{sub}\mathrel{=}\Varid{strich}\;(\mathbin{-}){}\<[E]%
\\[\blanklineskip]%
\>[3]{}\Varid{sump}\mathbin{::}(\Conid{Num}\;\Varid{a},\Conid{Eq}\;\Varid{a})\Rightarrow [\mskip1.5mu \Conid{Poly}\;\Varid{a}\mskip1.5mu]\to \Conid{Poly}\;\Varid{a}{}\<[E]%
\\
\>[3]{}\Varid{sump}\mathrel{=}\Varid{foldl'}\;\Varid{add}\;(\Conid{P}\;[\mskip1.5mu \mathrm{0}\mskip1.5mu]){}\<[E]%
\\[\blanklineskip]%
\>[3]{}\Varid{strich}\mathbin{::}(\Conid{Num}\;\Varid{a},\Conid{Eq}\;\Varid{a})\Rightarrow (\Varid{a}\to \Varid{a}\to \Varid{a})\to \Conid{Poly}\;\Varid{a}\to \Conid{Poly}\;\Varid{a}\to \Conid{Poly}\;\Varid{a}{}\<[E]%
\\
\>[3]{}\Varid{strich}\;\Varid{o}\;(\Conid{P}\;\Varid{x})\;(\Conid{P}\;\Varid{y}){}\<[25]%
\>[25]{}\mathrel{=}\Conid{P}\;(\Varid{strichlist}\;\Varid{o}\;\Varid{x}\;\Varid{y}){}\<[E]%
\\[\blanklineskip]%
\>[3]{}\Varid{strichlist}\mathbin{::}(\Conid{Num}\;\Varid{a},\Conid{Eq}\;\Varid{a})\Rightarrow (\Varid{a}\to \Varid{a}\to \Varid{a})\to [\mskip1.5mu \Varid{a}\mskip1.5mu]\to [\mskip1.5mu \Varid{a}\mskip1.5mu]\to [\mskip1.5mu \Varid{a}\mskip1.5mu]{}\<[E]%
\\
\>[3]{}\Varid{strichlist}\;\Varid{o}\;\Varid{xs}\;\Varid{ys}\mathrel{=}\Varid{cleanz}\;(\Varid{go}\;\Varid{xs}\;\Varid{ys}){}\<[E]%
\\
\>[3]{}\hsindent{2}{}\<[5]%
\>[5]{}\mathbf{where}\;{}\<[12]%
\>[12]{}\Varid{go}\;[\mskip1.5mu \mskip1.5mu]\;\Varid{bs}{}\<[30]%
\>[30]{}\mathrel{=}{}\<[30E]%
\>[33]{}\Varid{bs}{}\<[E]%
\\
\>[12]{}\Varid{go}\;\Varid{as}\;[\mskip1.5mu \mskip1.5mu]{}\<[30]%
\>[30]{}\mathrel{=}{}\<[30E]%
\>[33]{}\Varid{as}{}\<[E]%
\\
\>[12]{}\Varid{go}\;(\Varid{a}\mathbin{:}\Varid{as})\;(\Varid{b}\mathbin{:}\Varid{bs}){}\<[30]%
\>[30]{}\mathrel{=}{}\<[30E]%
\>[33]{}\Varid{a}\mathbin{`\Varid{o}`}\Varid{b}\mathbin{:}\Varid{go}\;\Varid{as}\;\Varid{bs}{}\<[E]%
\ColumnHook
\end{pboxed}
\)\par\noindent\endgroup\resethooks
\end{minipage}

Note that \ensuremath{\Varid{sump}} function, which implements \ensuremath{\Varid{sum}} for polynomials.
Here is one more function that might be useful 
later on; it folds \ensuremath{\Varid{strichlist}} on a list of lists of coefficients:

\begin{minipage}{\textwidth}
\begingroup\par\noindent\advance\leftskip\mathindent\(
\begin{pboxed}\SaveRestoreHook
\column{B}{@{}>{\hspre}l<{\hspost}@{}}%
\column{3}{@{}>{\hspre}l<{\hspost}@{}}%
\column{E}{@{}>{\hspre}l<{\hspost}@{}}%
\>[3]{}\Varid{strichf}\mathbin{::}(\Conid{Num}\;\Varid{a},\Conid{Eq}\;\Varid{a})\Rightarrow (\Varid{a}\to \Varid{a}\to \Varid{a})\to [\mskip1.5mu [\mskip1.5mu \Varid{a}\mskip1.5mu]\mskip1.5mu]\to [\mskip1.5mu \Varid{a}\mskip1.5mu]{}\<[E]%
\\
\>[3]{}\Varid{strichf}\;\Varid{o}\mathrel{=}\Varid{foldl'}\;(\Varid{strichlist}\;\Varid{o})\;[\mskip1.5mu \mskip1.5mu]{}\<[E]%
\ColumnHook
\end{pboxed}
\)\par\noindent\endgroup\resethooks
\end{minipage}

\term{Punktrechnung}, \ie\ multiplication and division,
are a bit more complex -- because of the distribution law.
Let us start with the simple case where we distribute
a monomial over a polynomial:

\begin{minipage}{\textwidth}
\begingroup\par\noindent\advance\leftskip\mathindent\(
\begin{pboxed}\SaveRestoreHook
\column{B}{@{}>{\hspre}l<{\hspost}@{}}%
\column{3}{@{}>{\hspre}l<{\hspost}@{}}%
\column{5}{@{}>{\hspre}l<{\hspost}@{}}%
\column{12}{@{}>{\hspre}l<{\hspost}@{}}%
\column{25}{@{}>{\hspre}c<{\hspost}@{}}%
\column{25E}{@{}l@{}}%
\column{28}{@{}>{\hspre}l<{\hspost}@{}}%
\column{E}{@{}>{\hspre}l<{\hspost}@{}}%
\>[3]{}\Varid{mul1}\mathbin{::}\Conid{Num}\;\Varid{a}\Rightarrow (\Varid{a}\to \Varid{a}\to \Varid{a})\to \Conid{Int}\to [\mskip1.5mu \Varid{a}\mskip1.5mu]\to \Varid{a}\to [\mskip1.5mu \Varid{a}\mskip1.5mu]{}\<[E]%
\\
\>[3]{}\Varid{mul1}\;\Varid{o}\;\Varid{i}\;\Varid{as}\;\Varid{a}\mathrel{=}\Varid{zeros}\;\Varid{i}\plus \Varid{go}\;\Varid{as}\;\Varid{a}{}\<[E]%
\\
\>[3]{}\hsindent{2}{}\<[5]%
\>[5]{}\mathbf{where}\;{}\<[12]%
\>[12]{}\Varid{go}\;[\mskip1.5mu \mskip1.5mu]\;\anonymous {}\<[25]%
\>[25]{}\mathrel{=}{}\<[25E]%
\>[28]{}[\mskip1.5mu \mskip1.5mu]{}\<[E]%
\\
\>[12]{}\Varid{go}\;(\Varid{c}\mathbin{:}\Varid{cs})\;\Varid{x}{}\<[25]%
\>[25]{}\mathrel{=}{}\<[25E]%
\>[28]{}\Varid{c}\mathbin{`\Varid{o}`}\Varid{x}\mathbin{:}\Varid{go}\;\Varid{cs}\;\Varid{x}{}\<[E]%
\\[\blanklineskip]%
\>[3]{}\Varid{zeros}\mathbin{::}\Conid{Num}\;\Varid{a}\Rightarrow \Conid{Int}\to [\mskip1.5mu \Varid{a}\mskip1.5mu]{}\<[E]%
\\
\>[3]{}\Varid{zeros}\;\Varid{i}\mathrel{=}\Varid{take}\;\Varid{i}\mathbin{\$}\Varid{repeat}\;\mathrm{0}{}\<[E]%
\ColumnHook
\end{pboxed}
\)\par\noindent\endgroup\resethooks
\end{minipage}

The function \ensuremath{\Varid{mul1}} takes a single term (the monomial)
and distributes it over the coefficients of a polynomial 
using the operation \ensuremath{\Varid{o}}.
Each term in the polynomial 
is combined with the single term.
This corresponds to the operation:

\begin{equation}
\begin{array}{rcrcrcccr}
dx^m & \times & ax^n      & + & bx^{n-1}    & + & \dots & + & c\\
     & =      & adx^{m+n} & + & bdx^{n-1+m} & + & \dots & + & cdx^m
\end{array}
\end{equation}

The function \ensuremath{\Varid{mul1}} receives on more parameter,
namely the \ensuremath{\Conid{Int}} $i$ and uses it to generate a sequence of zeros
that is put in front of the resulting coefficient list.
As we will see shortly, the list of zeros reflects the weight
of the single term. In fact, we do not implement the manipulation
of the exponents we see in the abstract formula directly.
Instead, the addition $+m$ is implicitly handled by placing
$m$ zeros at the head of the list resulting in a new polynomial
of degree $m+d$ where $d$ is the degree of the original polynomial.
A simple example:

\[
5x^2 \times (4x^3 + 3x^2 + 2x + 1) = 20x^5 + 15x^4 + 10x^3 + 5x^2
\]

would be:

\ensuremath{\Varid{mul1}\;\mathrm{2}\;[\mskip1.5mu \mathrm{1},\mathrm{2},\mathrm{3},\mathrm{4}\mskip1.5mu]\;\mathrm{5}}

which is: 

\ensuremath{\Varid{zero}\;\mathrm{2}\plus (\mathrm{5}\mathbin{*}[\mskip1.5mu \mathrm{1},\mathrm{2},\mathrm{3},\mathrm{4}\mskip1.5mu])} $=$ \ensuremath{[\mskip1.5mu \mathrm{0},\mathrm{0},\mathrm{5},\mathrm{10},\mathrm{15},\mathrm{20}\mskip1.5mu]}

We, hence, would add 2 zeros, since 2 is the degree
of the monomial.

Now, when we multiply two polynomials, we need to map
all terms in one of the polynomials on the other polynomial
using \ensuremath{\Varid{mul1}}. We further need to pass the weight of
the individual terms of the first polynomial as the \ensuremath{\Conid{Int}}
parameter of \ensuremath{\Varid{mul1}}. What we want to do is:

\ensuremath{[\mskip1.5mu \Varid{mul1}\;(\mathbin{*})\;\Varid{i}\;(\Varid{coeffs}\;\Varid{p1})\;\Varid{p}\mid (\Varid{i},\Varid{p})\leftarrow \Varid{zip}\;[\mskip1.5mu \mathrm{0}\mathinner{\ldotp\ldotp}\mskip1.5mu]\;(\Varid{coeffs}\;\Varid{p2})\mskip1.5mu]}.

What would we get applying this formula on
the polynomials, say, 
\ensuremath{[\mskip1.5mu \mathrm{1},\mathrm{2},\mathrm{3},\mathrm{4}\mskip1.5mu]} and \ensuremath{[\mskip1.5mu \mathrm{5},\mathrm{6},\mathrm{7},\mathrm{8}\mskip1.5mu]}?
Let us have a look:

\ensuremath{[\mskip1.5mu \Varid{mul1}\;(\mathbin{*})\;\Varid{i}\;([\mskip1.5mu \mathrm{5},\mathrm{6},\mathrm{7},\mathrm{8}\mskip1.5mu])\;\Varid{p}\mid (\Varid{i},\Varid{p})\leftarrow \Varid{zip}\;[\mskip1.5mu \mathrm{0}\mathinner{\ldotp\ldotp}\mskip1.5mu]\;[\mskip1.5mu \mathrm{1},\mathrm{2},\mathrm{3},\mathrm{4}\mskip1.5mu]\mskip1.5mu]}\\
\ensuremath{[\mskip1.5mu [\mskip1.5mu \mathrm{5},\mathrm{6},\mathrm{7},\mathrm{8}\mskip1.5mu],[\mskip1.5mu \mathrm{0},\mathrm{10},\mathrm{12},\mathrm{14},\mathrm{16}\mskip1.5mu],[\mskip1.5mu \mathrm{0},\mathrm{0},\mathrm{15},\mathrm{18},\mathrm{21},\mathrm{24}\mskip1.5mu],[\mskip1.5mu \mathrm{0},\mathrm{0},\mathrm{0},\mathrm{20},\mathrm{24},\mathrm{28},\mathrm{32}\mskip1.5mu]\mskip1.5mu]}.

We see a list of four lists, 
one for each coefficient of \ensuremath{[\mskip1.5mu \mathrm{1},\mathrm{2},\mathrm{3},\mathrm{4}\mskip1.5mu]}.
The first list is the result of distributing 1 
over all the coefficients in \ensuremath{[\mskip1.5mu \mathrm{5},\mathrm{6},\mathrm{7},\mathrm{8}\mskip1.5mu]}.
Since 1 is the first element,
its weight is 0: no zeros are put before the resulting list.
The second list results from distributing 2 over \ensuremath{[\mskip1.5mu \mathrm{5},\mathrm{6},\mathrm{7},\mathrm{8}\mskip1.5mu]}.
Since 2 is the second element, its weight is 1:
we add one zero.
The same process is repeated for 3 and 4 resulting
in the third and fourth result list.
Since 3 is the the third element, the third resulting list
gets two zeros and, since 4 is the fourth element,
the fourth list gets three zeros.

How do we transform this list of lists back
into a single list of coefficients? Very easy:
we add them together using \ensuremath{\Varid{strichf}}:

\ensuremath{\Varid{strichf}\;(\mathbin{+})}
\ensuremath{[\mskip1.5mu [\mskip1.5mu \mathrm{5},\mathrm{6},\mathrm{7},\mathrm{8}\mskip1.5mu],[\mskip1.5mu \mathrm{0},\mathrm{10},\mathrm{12},\mathrm{14},\mathrm{16}\mskip1.5mu],[\mskip1.5mu \mathrm{0},\mathrm{0},\mathrm{15},\mathrm{18},\mathrm{21},\mathrm{24}\mskip1.5mu],[\mskip1.5mu \mathrm{0},\mathrm{0},\mathrm{0},\mathrm{20},\mathrm{24},\mathrm{28},\mathrm{32}\mskip1.5mu]\mskip1.5mu]}

which is

\ensuremath{[\mskip1.5mu \mathrm{5},\mathrm{16},\mathrm{34},\mathrm{60},\mathrm{61},\mathrm{52},\mathrm{32}\mskip1.5mu]}.

This means that

\begin{equation}
(4x^3 + 3x^2 + 2x + 1) \times (8x^3 + 7x^2 + 6x + 5) =
32x^6 + 52x^5 + 61x^4 + 60x^3 + 34x^2 + 16x + 5.
\end{equation}

Here is the whole algorithm:

\begin{minipage}{\textwidth}
\begingroup\par\noindent\advance\leftskip\mathindent\(
\begin{pboxed}\SaveRestoreHook
\column{B}{@{}>{\hspre}l<{\hspost}@{}}%
\column{3}{@{}>{\hspre}l<{\hspost}@{}}%
\column{5}{@{}>{\hspre}l<{\hspost}@{}}%
\column{12}{@{}>{\hspre}l<{\hspost}@{}}%
\column{14}{@{}>{\hspre}c<{\hspost}@{}}%
\column{14E}{@{}l@{}}%
\column{16}{@{}>{\hspre}c<{\hspost}@{}}%
\column{16E}{@{}l@{}}%
\column{17}{@{}>{\hspre}l<{\hspost}@{}}%
\column{19}{@{}>{\hspre}l<{\hspost}@{}}%
\column{28}{@{}>{\hspre}c<{\hspost}@{}}%
\column{28E}{@{}l@{}}%
\column{31}{@{}>{\hspre}l<{\hspost}@{}}%
\column{E}{@{}>{\hspre}l<{\hspost}@{}}%
\>[3]{}\Varid{mul}\mathbin{::}(\Conid{Show}\;\Varid{a},\Conid{Num}\;\Varid{a},\Conid{Eq}\;\Varid{a})\Rightarrow \Conid{Poly}\;\Varid{a}\to \Conid{Poly}\;\Varid{a}\to \Conid{Poly}\;\Varid{a}{}\<[E]%
\\
\>[3]{}\Varid{mul}\;\Varid{p1}\;\Varid{p2}{}\<[14]%
\>[14]{}\mid {}\<[14E]%
\>[17]{}\Varid{d2}\mathbin{>}\Varid{d1}{}\<[28]%
\>[28]{}\mathrel{=}{}\<[28E]%
\>[31]{}\Varid{mul}\;\Varid{p2}\;\Varid{p1}{}\<[E]%
\\
\>[14]{}\mid {}\<[14E]%
\>[17]{}\Varid{otherwise}{}\<[28]%
\>[28]{}\mathrel{=}{}\<[28E]%
\>[31]{}\Conid{P}\;(\Varid{strichf}\;(\mathbin{+})\;\Varid{ms}){}\<[E]%
\\
\>[3]{}\hsindent{2}{}\<[5]%
\>[5]{}\mathbf{where}\;{}\<[12]%
\>[12]{}\Varid{d1}{}\<[16]%
\>[16]{}\mathrel{=}{}\<[16E]%
\>[19]{}\Varid{degree}\;\Varid{p1}{}\<[E]%
\\
\>[12]{}\Varid{d2}{}\<[16]%
\>[16]{}\mathrel{=}{}\<[16E]%
\>[19]{}\Varid{degree}\;\Varid{p2}{}\<[E]%
\\
\>[12]{}\Varid{ms}{}\<[16]%
\>[16]{}\mathrel{=}{}\<[16E]%
\>[19]{}[\mskip1.5mu \Varid{mul1}\;(\mathbin{*})\;\Varid{i}\;(\Varid{coeffs}\;\Varid{p1})\;\Varid{p}\mathrel{\vee}(\Varid{i},\Varid{p})\leftarrow \Varid{zip}\;[\mskip1.5mu \mathrm{0}\mathinner{\ldotp\ldotp}\mskip1.5mu]\;(\Varid{coeffs}\;\Varid{p2})\mskip1.5mu]{}\<[E]%
\ColumnHook
\end{pboxed}
\)\par\noindent\endgroup\resethooks
\end{minipage}

On top of multiplication, we can implement power.
We will, of course, not implement a na\"ive approach
based on repeated multiplication alone. Instead,
we will use the \term{square-and-multiply} approach
we have already used before for numbers.
Here is the code:

\begin{minipage}{\textwidth}
\begingroup\par\noindent\advance\leftskip\mathindent\(
\begin{pboxed}\SaveRestoreHook
\column{B}{@{}>{\hspre}l<{\hspost}@{}}%
\column{3}{@{}>{\hspre}l<{\hspost}@{}}%
\column{5}{@{}>{\hspre}l<{\hspost}@{}}%
\column{12}{@{}>{\hspre}l<{\hspost}@{}}%
\column{23}{@{}>{\hspre}c<{\hspost}@{}}%
\column{23E}{@{}l@{}}%
\column{26}{@{}>{\hspre}l<{\hspost}@{}}%
\column{37}{@{}>{\hspre}l<{\hspost}@{}}%
\column{59}{@{}>{\hspre}l<{\hspost}@{}}%
\column{E}{@{}>{\hspre}l<{\hspost}@{}}%
\>[3]{}\Varid{powp}\mathbin{::}(\Conid{Show}\;\Varid{a},\Conid{Num}\;\Varid{a},\Conid{Eq}\;\Varid{a})\Rightarrow \Conid{Natural}\to \Conid{Poly}\;\Varid{a}\to \Conid{Poly}\;\Varid{a}{}\<[E]%
\\
\>[3]{}\Varid{powp}\;\Varid{f}\;\Varid{poly}\mathrel{=}\Varid{go}\;\Varid{f}\;(\Conid{P}\;[\mskip1.5mu \mathrm{1}\mskip1.5mu])\;\Varid{poly}{}\<[E]%
\\
\>[3]{}\hsindent{2}{}\<[5]%
\>[5]{}\mathbf{where}\;{}\<[12]%
\>[12]{}\Varid{go}\;\mathrm{0}\;\Varid{y}\;\anonymous {}\<[23]%
\>[23]{}\mathrel{=}{}\<[23E]%
\>[26]{}\Varid{y}{}\<[E]%
\\
\>[12]{}\Varid{go}\;\mathrm{1}\;\Varid{y}\;\Varid{x}{}\<[23]%
\>[23]{}\mathrel{=}{}\<[23E]%
\>[26]{}\Varid{mul}\;\Varid{y}\;\Varid{x}{}\<[E]%
\\
\>[12]{}\Varid{go}\;\Varid{n}\;\Varid{y}\;\Varid{x}{}\<[23]%
\>[23]{}\mid {}\<[23E]%
\>[26]{}\Varid{even}\;\Varid{n}{}\<[37]%
\>[37]{}\mathrel{=}\Varid{go}\;(\Varid{n}\mathbin{\Varid{`div`}}\mathrm{2})\;\Varid{y}\;{}\<[59]%
\>[59]{}(\Varid{mul}\;\Varid{x}\;\Varid{x}){}\<[E]%
\\
\>[23]{}\mid {}\<[23E]%
\>[26]{}\Varid{otherwise}{}\<[37]%
\>[37]{}\mathrel{=}\Varid{go}\;((\Varid{n}\mathbin{-}\mathrm{1})\mathbin{\Varid{`div`}}\mathrm{2})\;{}\<[59]%
\>[59]{}(\Varid{mul}\;\Varid{y}\;\Varid{x})\;{}\<[E]%
\\
\>[59]{}(\Varid{mul}\;\Varid{x}\;\Varid{x}){}\<[E]%
\ColumnHook
\end{pboxed}
\)\par\noindent\endgroup\resethooks
\end{minipage}

The function \ensuremath{\Varid{powp}} receives a natural number,
that is the exponent, and a polynomial.
We kick off by calling \ensuremath{\Varid{go}} with the exponent, $f$,
a base polynomial \ensuremath{\Conid{P}\;[\mskip1.5mu \mathrm{1}\mskip1.5mu]}, \ie\ unity, and the polynomial
we want to raise to the power of \ensuremath{\Varid{f}}.
If $f=0$, we are done and return the base polynomial.
This reflects the case $x^0=1$.
If $f=1$, we multiply the base polynomial by the input polynomial.
Otherwise, if the exponent is even,
we halve it, pass the base polynomial on and square the input.
Otherwise, we pass the product of the base polynomial and the input
on instead of the base polynomial as it is.
This implementation differs a bit from the implementation
we presented before for numbers, but it implements the same
algorithm.

Here is a simple example: we raise the polynomial
$x + 1$ to the power of 5. In the first round, we compute

\ensuremath{\Varid{go}\;\mathrm{5}\;(\Conid{P}\;[\mskip1.5mu \mathrm{1}\mskip1.5mu])\;(\Conid{P}\;[\mskip1.5mu \mathrm{1},\mathrm{1}\mskip1.5mu])},

which, since 5 is odd, results in 

\ensuremath{\Varid{go}\;\mathrm{2}\;(\Conid{P}\;[\mskip1.5mu \mathrm{1},\mathrm{1}\mskip1.5mu])\;(\Conid{P}\;[\mskip1.5mu \mathrm{1},\mathrm{2},\mathrm{1}\mskip1.5mu])}.

This, in its turn, results in

\ensuremath{\Varid{go}\;\mathrm{1}\;(\Conid{P}\;[\mskip1.5mu \mathrm{1},\mathrm{1}\mskip1.5mu])\;(\Conid{P}\;[\mskip1.5mu \mathrm{1},\mathrm{4},\mathrm{6},\mathrm{4},\mathrm{1}\mskip1.5mu])}.

This is the final step and results in 

\ensuremath{\Varid{mul}\;(\Conid{P}\;[\mskip1.5mu \mathrm{1},\mathrm{1}\mskip1.5mu])\;(\Conid{P}\;[\mskip1.5mu \mathrm{1},\mathrm{4},\mathrm{6},\mathrm{4},\mathrm{1}\mskip1.5mu])}, 

which is

\ensuremath{\Conid{P}\;[\mskip1.5mu \mathrm{1},\mathrm{5},\mathrm{10},\mathrm{10},\mathrm{5},\mathrm{1}\mskip1.5mu]},

the polynomial $x^5 + 5x^4 + 10x^3 + 10x^2 + 5x + 1$.
You might have noticed that our Haskell notation
shows the binomial coefficients $\binom{n}{k}$ for
$n=0$, $n=1$, $n=2$, $n=4$ and $n=5$.
We never see $n=3$, which would be 
\ensuremath{\Conid{P}\;[\mskip1.5mu \mathrm{1},\mathrm{3},\mathrm{3},\mathrm{1}\mskip1.5mu]}, because we leave the multiplication
\ensuremath{\Varid{mul}\;(\Conid{P}\;[\mskip1.5mu \mathrm{1},\mathrm{1}\mskip1.5mu])\;(\Conid{P}\;[\mskip1.5mu \mathrm{1},\mathrm{2},\mathrm{1}\mskip1.5mu])} out.
For this specific case with exponent 5,
leaving out this step is where square-and-multiply
is more efficient than multiplying five times.
With growing exponents, the saving quickly grows
to a significant order.

Division is, as usual, a bit more complicated than multiplication.
But it is not too different from number division. First,
we define polynomial division as Euclidean division, that is
we search the solution for the equation

\begin{equation}
\frac{a}{b} = q + r
\end{equation}

where $r < b$ and $bq+r=a$.

The manual process is as follows:
we divide the first term of $a$ by the first term of $b$.
The quotient goes to the result; then we multiply it by $b$
and set $a$ to $a$ minus that result.
Now we repeat the process
until the degree of $a$
is less than that of $b$.

Here is an example:

\[
\frac{4x^5 - x^4 + 2x^3 + x^2 - 1}{x^2 + 1}.
\]

We start by dividing $4x^5$ by $x^2$.
The quotient is $4x^3$, which we add to the result.
We multiply: $4x^3 \times (x^2 + 1) = 4x^5 + 4x^3$
and subtract the result from $a$:

\begin{equation}
\begin{array}{crcrcrcrcr}
  & 4x^5 & - &  x^4 & + & 2x^3 & + & x^2 & - & 1\\
- & 4x^5 &   &      & + & 4x^3 &   &     &   &  \\
= &      & - &  x^4 & - & 2x^3 & + & x^2 & - & 1
\end{array}
\end{equation}

We continue with
$-x^4$ and divide it by $x^2$, which is
$-x^2$. 
The overall result now is $4x^3 - x^2$.
We multiply $-x^2 \times (x^2 + 1) = -x^4 - x^2$
and subtract that from what remains from $a$:

\begin{equation}
\begin{array}{ccrcrcrcr}
  & - &  x^4 & - & 2x^3 & + &  x^2 & - & 1\\
- & - &  x^4 &   &      & - &  x^2 &   &  \\
= &   &      & - & 2x^3 & + & 2x^2 & - & 1
\end{array}
\end{equation}

We continue with $-2x^3$, which, divided by
$x^2$ is $-2x$. 
We multiply $-2x \times (x^2 + 1) = -2x^3 - 2x$
and subtract:

\begin{equation}
\begin{array}{ccrcrcrcr}
  & - & 2x^3 & + & 2x^2 & + &    & - & 1\\
- & - & 2x^3 &   &      & - & 2x &   &  \\
= &   &      &   & 2x^2 & + & 2x & - & 1 
\end{array}
\end{equation}

The result now is 
$4x^3 - x^2 - 2x$.
We continue with $2x^2$, which,
divided by $x^2$ is 2. 
We multiply $2\times (x^2 + 1) = 2x^2 + 2$
and subtract:

\begin{equation}
\begin{array}{ccrcrcrcr}
  & 2x^2 & + & 2x & - & 1\\
- & 2x^2 &   &    & + & 2\\
= &      &   & 2x & - & 3 
\end{array}
\end{equation}

The result now is
$4x^3 - x^2 - 2x + 2$.
We finally have $2x - 3$,
which is smaller in degree than $b$.
The result, hence, is
$(4x^3 - x^2 - 2x + 2, 2x - 3)$.

Here is an implementation of division in Haskell:

\begin{minipage}{\textwidth}
\begingroup\par\noindent\advance\leftskip\mathindent\(
\begin{pboxed}\SaveRestoreHook
\column{B}{@{}>{\hspre}l<{\hspost}@{}}%
\column{3}{@{}>{\hspre}l<{\hspost}@{}}%
\column{5}{@{}>{\hspre}l<{\hspost}@{}}%
\column{12}{@{}>{\hspre}l<{\hspost}@{}}%
\column{20}{@{}>{\hspre}c<{\hspost}@{}}%
\column{20E}{@{}l@{}}%
\column{23}{@{}>{\hspre}l<{\hspost}@{}}%
\column{28}{@{}>{\hspre}l<{\hspost}@{}}%
\column{30}{@{}>{\hspre}l<{\hspost}@{}}%
\column{32}{@{}>{\hspre}c<{\hspost}@{}}%
\column{32E}{@{}l@{}}%
\column{35}{@{}>{\hspre}l<{\hspost}@{}}%
\column{43}{@{}>{\hspre}c<{\hspost}@{}}%
\column{43E}{@{}l@{}}%
\column{46}{@{}>{\hspre}l<{\hspost}@{}}%
\column{E}{@{}>{\hspre}l<{\hspost}@{}}%
\>[3]{}\Varid{divp}\mathbin{::}{}\<[12]%
\>[12]{}(\Conid{Show}\;\Varid{a},\Conid{Num}\;\Varid{a},\Conid{Eq}\;\Varid{a},\Conid{Fractional}\;\Varid{a},\Conid{Ord}\;\Varid{a})\Rightarrow {}\<[E]%
\\
\>[12]{}\Conid{Poly}\;\Varid{a}\to \Conid{Poly}\;\Varid{a}\to (\Conid{Poly}\;\Varid{a},\Conid{Poly}\;\Varid{a}){}\<[E]%
\\
\>[3]{}\Varid{divp}\;(\Conid{P}\;\Varid{as})\;(\Conid{P}\;\Varid{bs})\mathrel{=}\mathbf{let}\;(\Varid{q},\Varid{r})\mathrel{=}\Varid{go}\;[\mskip1.5mu \mskip1.5mu]\;\Varid{as}\;\mathbf{in}\;(\Conid{P}\;\Varid{q},\Conid{P}\;\Varid{r}){}\<[E]%
\\
\>[3]{}\hsindent{2}{}\<[5]%
\>[5]{}\mathbf{where}\;{}\<[12]%
\>[12]{}\Varid{db}\mathrel{=}\Varid{degree}\;(\Conid{P}\;\Varid{bs}){}\<[E]%
\\
\>[12]{}\Varid{go}\;\Varid{q}\;\Varid{r}{}\<[20]%
\>[20]{}\mid {}\<[20E]%
\>[23]{}\Varid{degree}\;(\Conid{P}\;\Varid{r})\mathbin{<}\Varid{db}{}\<[43]%
\>[43]{}\mathrel{=}{}\<[43E]%
\>[46]{}(\Varid{q},\Varid{r}){}\<[E]%
\\
\>[20]{}\mid {}\<[20E]%
\>[23]{}\Varid{null}\;\Varid{r}\mathrel{\vee}\Varid{r}\equiv [\mskip1.5mu \mathrm{0}\mskip1.5mu]{}\<[43]%
\>[43]{}\mathrel{=}{}\<[43E]%
\>[46]{}(\Varid{q},\Varid{r}){}\<[E]%
\\
\>[20]{}\mid {}\<[20E]%
\>[23]{}\Varid{otherwise}{}\<[43]%
\>[43]{}\mathrel{=}{}\<[43E]%
\\
\>[23]{}\mathbf{let}\;{}\<[28]%
\>[28]{}\Varid{t}{}\<[32]%
\>[32]{}\mathrel{=}{}\<[32E]%
\>[35]{}\Varid{last}\;\Varid{r}\mathbin{/}\Varid{last}\;\Varid{bs}{}\<[E]%
\\
\>[28]{}\Varid{d}{}\<[32]%
\>[32]{}\mathrel{=}{}\<[32E]%
\>[35]{}\Varid{degree}\;(\Conid{P}\;\Varid{r})\mathbin{-}\Varid{db}{}\<[E]%
\\
\>[28]{}\Varid{ts}{}\<[32]%
\>[32]{}\mathrel{=}{}\<[32E]%
\>[35]{}\Varid{zeros}\;\Varid{d}\plus [\mskip1.5mu \Varid{t}\mskip1.5mu]{}\<[E]%
\\
\>[28]{}\Varid{m}{}\<[32]%
\>[32]{}\mathrel{=}{}\<[32E]%
\>[35]{}\Varid{mulist}\;\Varid{ts}\;\Varid{bs}{}\<[E]%
\\
\>[23]{}\mathbf{in}\;\Varid{go}\;{}\<[30]%
\>[30]{}(\Varid{cleanz}\mathbin{\$}\Varid{strichlist}\;(\mathbin{+})\;\Varid{q}\;\Varid{ts})\;{}\<[E]%
\\
\>[30]{}(\Varid{cleanz}\mathbin{\$}\Varid{strichlist}\;(\mathbin{-})\;\Varid{r}\;\Varid{m}){}\<[E]%
\\[\blanklineskip]%
\>[3]{}\Varid{mulist}\mathbin{::}(\Conid{Show}\;\Varid{a},\Conid{Num}\;\Varid{a},\Conid{Eq}\;\Varid{a})\Rightarrow [\mskip1.5mu \Varid{a}\mskip1.5mu]\to [\mskip1.5mu \Varid{a}\mskip1.5mu]\to [\mskip1.5mu \Varid{a}\mskip1.5mu]{}\<[E]%
\\
\>[3]{}\Varid{mulist}\;\Varid{c1}\;\Varid{c2}\mathrel{=}\Varid{coeffs}\mathbin{\$}\Varid{mul}\;(\Conid{P}\;\Varid{c1})\;(\Conid{P}\;\Varid{c2}){}\<[E]%
\ColumnHook
\end{pboxed}
\)\par\noindent\endgroup\resethooks
\end{minipage}

First note that division expects its arguments
to be polynomials over a \ensuremath{\Conid{Fractional}} data type.
We do not allow polynomials over integers to be used
with this implementation. The reason is that we do not
want to use Euclidean division on the coefficients.
That could indeed be very confusing. Furthermore,
polynomials are most often used with rational or real
coefficients. Restricting division to integers
(using Euclidean division) would, therefore, not make
much sense.

Observe further that we call \ensuremath{\Varid{go}} with an empty set --
that is the initial value of $q$, \ie\ the final result --
and $as$ -- that is initially the number to be divided,
the number we called $a$ above.
The function \ensuremath{\Varid{go}} has two base cases:
if the degree of $r$, the remainder and initially $as$,
is less than the degree of the divisor $b$, we are done.
The result is our current $(q,r)$. 
The same is true if $r$ is \ensuremath{\Varid{null}} or 
contains only the constant 0.
In this case, there is no remainder: $b$ divides $a$.

Otherwise, we divide the \ensuremath{\Varid{last}} of $r$ by the \ensuremath{\Varid{last}} of $b$.
Note that those are the term with the highest degree
in each polynomial.
This division is just a number division of the two
coefficients. We still have to compute the new exponent,
which is the exponent of \ensuremath{\Varid{last}\;\Varid{r}} minus the exponent of 
\ensuremath{\Varid{last}\;\Varid{b}}, \ie\ their weight. We do this by subtracting
their degrees and then inserting zeros 
at the head of the result \ensuremath{\Varid{ts}}.
This result, \ensuremath{\Varid{ts}}, is then added to $q$.
We further compute $ts \times bs$ and subtract
the result from $r$. The function \ensuremath{\Varid{mulist}} we use for this purpose
is just a wrapper around \ensuremath{\Varid{mul}} using
lists of coefficients instead of \ensuremath{\Conid{Poly}} variables.
With the resulting $(q,r)$, we go into the next round.

Let us try this with our example from above: 

\[
\frac{4x^5 - x^4 + 2x^3 + x^2 - 1}{x^2 + 1}.
\]

We call \ensuremath{\Varid{divp}\;(\Conid{P}\;[\mskip1.5mu \mathbin{-}\mathrm{1},\mathrm{0},\mathrm{1},\mathrm{2},\mathbin{-}\mathrm{1},\mathrm{4}\mskip1.5mu])\;(\Conid{P}\;[\mskip1.5mu \mathrm{1},\mathrm{0},\mathrm{1}\mskip1.5mu])} and get
\ensuremath{(\Conid{P}\;[\mskip1.5mu \mathrm{2},\mathbin{-}\mathrm{2},\mathbin{-}\mathrm{1},\mathrm{4}\mskip1.5mu],\Conid{P}\;[\mskip1.5mu \mathbin{-}\mathrm{3},\mathrm{2}\mskip1.5mu])}, which translates to the polynomials
$4x^3-x^2-2x+2$ and $2x - 3$. 
This is the same result we obtained above 
with the manual procedure.

From here on, we can implement functions based on division,
such as \ensuremath{\Varid{divides}}:

\begin{minipage}{\textwidth}
\begingroup\par\noindent\advance\leftskip\mathindent\(
\begin{pboxed}\SaveRestoreHook
\column{B}{@{}>{\hspre}l<{\hspost}@{}}%
\column{3}{@{}>{\hspre}l<{\hspost}@{}}%
\column{15}{@{}>{\hspre}l<{\hspost}@{}}%
\column{18}{@{}>{\hspre}l<{\hspost}@{}}%
\column{29}{@{}>{\hspre}c<{\hspost}@{}}%
\column{29E}{@{}l@{}}%
\column{33}{@{}>{\hspre}l<{\hspost}@{}}%
\column{E}{@{}>{\hspre}l<{\hspost}@{}}%
\>[3]{}\Varid{divides}\mathbin{::}{}\<[15]%
\>[15]{}(\Conid{Show}\;\Varid{a},\Conid{Num}\;\Varid{a},\Conid{Eq}\;\Varid{a},\Conid{Ord}\;\Varid{a})\Rightarrow {}\<[E]%
\\
\>[15]{}\Conid{Poly}\;\Varid{a}\to \Conid{Poly}\;\Varid{a}\to \Conid{Bool}{}\<[E]%
\\
\>[3]{}\Varid{divides}\;\Varid{a}\;\Varid{b}\mathrel{=}{}\<[18]%
\>[18]{}\mathbf{case}\;\Varid{b}\mathbin{`\Varid{divp}`}{}\<[33]%
\>[33]{}\Varid{a}\;\mathbf{of}{}\<[E]%
\\
\>[18]{}(\anonymous ,\Conid{P}\;[\mskip1.5mu \mathrm{0}\mskip1.5mu]){}\<[29]%
\>[29]{}\to {}\<[29E]%
\>[33]{}\Conid{True}{}\<[E]%
\\
\>[18]{}\anonymous {}\<[29]%
\>[29]{}\to {}\<[29E]%
\>[33]{}\Conid{False}{}\<[E]%
\ColumnHook
\end{pboxed}
\)\par\noindent\endgroup\resethooks
\end{minipage}

the remainder:

\begin{minipage}{\textwidth}
\begingroup\par\noindent\advance\leftskip\mathindent\(
\begin{pboxed}\SaveRestoreHook
\column{B}{@{}>{\hspre}l<{\hspost}@{}}%
\column{3}{@{}>{\hspre}l<{\hspost}@{}}%
\column{12}{@{}>{\hspre}l<{\hspost}@{}}%
\column{15}{@{}>{\hspre}l<{\hspost}@{}}%
\column{E}{@{}>{\hspre}l<{\hspost}@{}}%
\>[3]{}\Varid{remp}\mathbin{::}{}\<[12]%
\>[12]{}(\Conid{Show}\;\Varid{a},\Conid{Num}\;\Varid{a},\Conid{Eq}\;\Varid{a},\Conid{Ord}\;\Varid{a})\Rightarrow {}\<[E]%
\\
\>[12]{}\Conid{Poly}\;\Varid{a}\to \Conid{Poly}\;\Varid{a}\to \Conid{Bool}{}\<[E]%
\\
\>[3]{}\Varid{remp}\;\Varid{a}\;\Varid{b}\mathrel{=}{}\<[15]%
\>[15]{}\mathbf{let}\;(\anonymous ,\Varid{r})\mathrel{=}\Varid{b}\mathbin{`\Varid{d}`}\Varid{a}\;\mathbf{in}\;\Varid{r}{}\<[E]%
\ColumnHook
\end{pboxed}
\)\par\noindent\endgroup\resethooks
\end{minipage}

and, of course, the \acronym{gcd}:

\begin{minipage}{\textwidth}
\begingroup\par\noindent\advance\leftskip\mathindent\(
\begin{pboxed}\SaveRestoreHook
\column{B}{@{}>{\hspre}l<{\hspost}@{}}%
\column{3}{@{}>{\hspre}l<{\hspost}@{}}%
\column{12}{@{}>{\hspre}l<{\hspost}@{}}%
\column{13}{@{}>{\hspre}c<{\hspost}@{}}%
\column{13E}{@{}l@{}}%
\column{16}{@{}>{\hspre}l<{\hspost}@{}}%
\column{27}{@{}>{\hspre}l<{\hspost}@{}}%
\column{E}{@{}>{\hspre}l<{\hspost}@{}}%
\>[3]{}\Varid{gcdp}\mathbin{::}{}\<[12]%
\>[12]{}(\Conid{Show}\;\Varid{a},\Conid{Num}\;\Varid{a},\Conid{Eq}\;\Varid{a},\Conid{Fractional}\;\Varid{a},\Conid{Ord}\;\Varid{a})\Rightarrow {}\<[E]%
\\
\>[12]{}\Conid{Poly}\;\Varid{a}\to \Conid{Poly}\;\Varid{a}\to \Conid{Poly}\;\Varid{a}{}\<[E]%
\\
\>[3]{}\Varid{gcdp}\;\Varid{a}\;\Varid{b}{}\<[13]%
\>[13]{}\mid {}\<[13E]%
\>[16]{}\Varid{degree}\;\Varid{b}\mathbin{>}\Varid{degree}\;\Varid{a}\mathrel{=}\Varid{gcdp}\;\Varid{b}\;\Varid{a}{}\<[E]%
\\
\>[13]{}\mid {}\<[13E]%
\>[16]{}\Varid{zerop}\;\Varid{b}{}\<[27]%
\>[27]{}\mathrel{=}\Varid{a}{}\<[E]%
\\
\>[13]{}\mid {}\<[13E]%
\>[16]{}\Varid{otherwise}{}\<[27]%
\>[27]{}\mathrel{=}\mathbf{let}\;(\anonymous ,\Varid{r})\mathrel{=}\Varid{divp}\;\Varid{a}\;\Varid{b}\;\mathbf{in}\;\Varid{gcdp}\;\Varid{b}\;\Varid{r}{}\<[E]%
\ColumnHook
\end{pboxed}
\)\par\noindent\endgroup\resethooks
\end{minipage}

We use a simple function to check whether
a polynomial is zero:

\begin{minipage}{\textwidth}
\begingroup\par\noindent\advance\leftskip\mathindent\(
\begin{pboxed}\SaveRestoreHook
\column{B}{@{}>{\hspre}l<{\hspost}@{}}%
\column{3}{@{}>{\hspre}l<{\hspost}@{}}%
\column{18}{@{}>{\hspre}l<{\hspost}@{}}%
\column{E}{@{}>{\hspre}l<{\hspost}@{}}%
\>[3]{}\Varid{zerop}\mathbin{::}(\Conid{Num}\;\Varid{a},\Conid{Eq}\;\Varid{a})\Rightarrow \Conid{Poly}\;\Varid{a}\to \Conid{Bool}{}\<[E]%
\\
\>[3]{}\Varid{zerop}\;(\Conid{P}\;[\mskip1.5mu \mathrm{0}\mskip1.5mu]){}\<[18]%
\>[18]{}\mathrel{=}\Conid{True}{}\<[E]%
\\
\>[3]{}\Varid{zerpo}\;\anonymous {}\<[18]%
\>[18]{}\mathrel{=}\Conid{False}{}\<[E]%
\ColumnHook
\end{pboxed}
\)\par\noindent\endgroup\resethooks
\end{minipage}

We can demonstrate \ensuremath{\Varid{gcdp}} nicely on binomial coefficients.
For instance, the \acronym{gcd} of the polynomials
$x^5 + 5x^4 + 10x^3 + 10x^2 + 5x + 1$ and
$x^3 + 3x^2 + 3x + 1$, thus

\ensuremath{\Varid{gcdp}\;(\Conid{P}\;[\mskip1.5mu \mathrm{1},\mathrm{5},\mathrm{10},\mathrm{10},\mathrm{5},\mathrm{1}\mskip1.5mu])\;(\Conid{P}\;[\mskip1.5mu \mathrm{1},\mathrm{3},\mathrm{3},\mathrm{1}\mskip1.5mu])}

is $x^3 + 3x^2 + 3x + 1$.

Since polynomials consisting of binomial coefficients of $n$,
where $n$ is the degree of the polynomial,
are always a product
of polynomials composed of smaller binomial coefficients
in the same way,
the \acronym{gcd} of two polynomials
consisting only of binomial coefficients,
is always the smaller of the two.
In other cases, that is, when the smaller does not divide
the greater, this implementation of the \acronym{gcd}
can lead to confusing results. For instance,
we multiply \ensuremath{\Conid{P}\;[\mskip1.5mu \mathrm{1},\mathrm{2},\mathrm{1}\mskip1.5mu]} by another polynomial, say,
\ensuremath{\Conid{P}\;[\mskip1.5mu \mathrm{1},\mathrm{2},\mathrm{3}\mskip1.5mu]}. The result is \ensuremath{\Conid{P}\;[\mskip1.5mu \mathrm{1},\mathrm{4},\mathrm{8},\mathrm{8},\mathrm{3}\mskip1.5mu]}. Now,

\ensuremath{\Varid{gcdp}\;(\Conid{P}\;[\mskip1.5mu \mathrm{1},\mathrm{5},\mathrm{10},\mathrm{10},\mathrm{5},\mathrm{1}\mskip1.5mu])\;(\Conid{P}\;[\mskip1.5mu \mathrm{1},\mathrm{4},\mathrm{8},\mathrm{8},\mathrm{3}\mskip1.5mu])}

does not yield the expected result \ensuremath{\Conid{P}\;[\mskip1.5mu \mathrm{1},\mathrm{2},\mathrm{1}\mskip1.5mu]}.
The reason is that the \acronym{gcd} is an operation
defined on integers, but we implemented it on top
of fractionals. That is often not what we want.
Anyway, here, we will actually use the \acronym{gcd} 
only in finite fields. 
Until now, we have discussed polynomials in infinite fields.
We now turn our attention to polynomial arithmetic
in a finite field and, hence, to modular polynomial arithmetic.

With modular arithmetic, all coefficients in the polynomial
are modulo $n$. That means we have to reduce those numbers.
This, of course, does only make sense with integers.
We first implement some helpers to reduce numbers modulo $n$
reusing functions implemented in the previous chapter.

The first function takes an integer modulo $n$:

\begin{minipage}{\textwidth}
\begingroup\par\noindent\advance\leftskip\mathindent\(
\begin{pboxed}\SaveRestoreHook
\column{B}{@{}>{\hspre}l<{\hspost}@{}}%
\column{3}{@{}>{\hspre}l<{\hspost}@{}}%
\column{13}{@{}>{\hspre}c<{\hspost}@{}}%
\column{13E}{@{}l@{}}%
\column{16}{@{}>{\hspre}l<{\hspost}@{}}%
\column{35}{@{}>{\hspre}c<{\hspost}@{}}%
\column{35E}{@{}l@{}}%
\column{38}{@{}>{\hspre}l<{\hspost}@{}}%
\column{E}{@{}>{\hspre}l<{\hspost}@{}}%
\>[3]{}\Varid{mmod}\mathbin{::}\Conid{Zahl}\to \Conid{Zahl}\to \Conid{Zahl}{}\<[E]%
\\
\>[3]{}\Varid{mmod}\;\Varid{n}\;\Varid{p}{}\<[13]%
\>[13]{}\mid {}\<[13E]%
\>[16]{}\Varid{n}\mathbin{<}\mathrm{0}\mathrel{\wedge}(\mathbin{-}\Varid{n})\mathbin{>}\Varid{p}{}\<[35]%
\>[35]{}\mathrel{=}{}\<[35E]%
\>[38]{}\Varid{mmod}\;(\mathbin{-}(\Varid{mmod}\;(\mathbin{-}\Varid{n}))\;\Varid{p})\;\Varid{p}{}\<[E]%
\\
\>[13]{}\mid {}\<[13E]%
\>[16]{}\Varid{n}\mathbin{<}\mathrm{0}{}\<[35]%
\>[35]{}\mathrel{=}{}\<[35E]%
\>[38]{}\Varid{mmod}\;(\Varid{p}\mathbin{+}\Varid{n})\;\Varid{p}{}\<[E]%
\\
\>[13]{}\mid {}\<[13E]%
\>[16]{}\Varid{otherwise}{}\<[35]%
\>[35]{}\mathrel{=}{}\<[35E]%
\>[38]{}\Varid{n}\mathbin{\Varid{`rem`}}\Varid{p}{}\<[E]%
\ColumnHook
\end{pboxed}
\)\par\noindent\endgroup\resethooks
\end{minipage}

Equipped with this function, we can easily implement multiplication:

\begin{minipage}{\textwidth}
\begingroup\par\noindent\advance\leftskip\mathindent\(
\begin{pboxed}\SaveRestoreHook
\column{B}{@{}>{\hspre}l<{\hspost}@{}}%
\column{3}{@{}>{\hspre}l<{\hspost}@{}}%
\column{E}{@{}>{\hspre}l<{\hspost}@{}}%
\>[3]{}\Varid{modmul}\mathbin{::}\Conid{Zahl}\to \Conid{Zahl}\to \Conid{Zahl}\to \Conid{Zahl}{}\<[E]%
\\
\>[3]{}\Varid{modmul}\;\Varid{p}\;\Varid{f1}\;\Varid{f2}\mathrel{=}(\Varid{f1}\mathbin{*}\Varid{f2})\mathbin{`\Varid{mmod}`}\Varid{p}{}\<[E]%
\ColumnHook
\end{pboxed}
\)\par\noindent\endgroup\resethooks
\end{minipage}

For division, we reuse the \ensuremath{\Varid{inverse}} function:

\begin{minipage}{\textwidth}
\begingroup\par\noindent\advance\leftskip\mathindent\(
\begin{pboxed}\SaveRestoreHook
\column{B}{@{}>{\hspre}l<{\hspost}@{}}%
\column{3}{@{}>{\hspre}l<{\hspost}@{}}%
\column{5}{@{}>{\hspre}l<{\hspost}@{}}%
\column{E}{@{}>{\hspre}l<{\hspost}@{}}%
\>[3]{}\Varid{modiv}\mathbin{::}\Conid{Zahl}\to \Conid{Zahl}\to \Conid{Zahl}\to \Conid{Zahl}{}\<[E]%
\\
\>[3]{}\Varid{modiv}\;\Varid{p}\;\Varid{n}\;\Varid{d}\mathrel{=}\Varid{modmul}\;\Varid{p}\;\Varid{n}\;\Varid{d'}{}\<[E]%
\\
\>[3]{}\hsindent{2}{}\<[5]%
\>[5]{}\mathbf{where}\;\Varid{d'}\mathrel{=}\Varid{\Conid{M}.inverse}\;\Varid{d}\;\Varid{p}{}\<[E]%
\ColumnHook
\end{pboxed}
\)\par\noindent\endgroup\resethooks
\end{minipage}

Now, we turn to polynomials. Here is, first, a function
that transforms a polynomial into one modulo $n$:

\begin{minipage}{\textwidth}
\begingroup\par\noindent\advance\leftskip\mathindent\(
\begin{pboxed}\SaveRestoreHook
\column{B}{@{}>{\hspre}l<{\hspost}@{}}%
\column{3}{@{}>{\hspre}l<{\hspost}@{}}%
\column{E}{@{}>{\hspre}l<{\hspost}@{}}%
\>[3]{}\Varid{pmod}\mathbin{::}\Conid{Poly}\;\Conid{Zahl}\to \Conid{Zahl}\to \Conid{Poly}\;\Conid{Zahl}{}\<[E]%
\\
\>[3]{}\Varid{pmod}\;(\Conid{P}\;\Varid{cs})\;\Varid{p}\mathrel{=}\Conid{P}\;[\mskip1.5mu \Varid{c}\mathbin{`\Varid{mmod}`}\Varid{p}\mid \Varid{c}\leftarrow \Varid{cs}\mskip1.5mu]{}\<[E]%
\ColumnHook
\end{pboxed}
\)\par\noindent\endgroup\resethooks
\end{minipage}

In other words, we just map \ensuremath{\Varid{mmod}} on all coefficients.
Let us look at some polynomials modulo a number, say, 7.
The polynomial \ensuremath{\Conid{P}\;[\mskip1.5mu \mathrm{1},\mathrm{2},\mathrm{3},\mathrm{4}\mskip1.5mu]}
we already used above is just the same modulo 7.
The polynomial \ensuremath{\Conid{P}\;[\mskip1.5mu \mathrm{5},\mathrm{6},\mathrm{7},\mathrm{8}\mskip1.5mu]}, however, changes:

\ensuremath{\Conid{P}\;[\mskip1.5mu \mathrm{5},\mathrm{6},\mathrm{7},\mathrm{8}\mskip1.5mu]\mathbin{`\Varid{pmod}`}\mathrm{7}}

is \ensuremath{\Conid{P}\;[\mskip1.5mu \mathrm{5},\mathrm{6},\mathrm{0},\mathrm{1}\mskip1.5mu]} or, in other words,
$8x^3 + 7x^2 + 6x + 5$ turns, modulo 7, into 
$x^3 + 6x + 5$.

The polynomial $x + 1$ raised to the power of 5 is
$x^5 + 5x^4 + 10x^3 + 10x^2 + 5x + 1$. Modulo 7, this
reduces to $x^5 + 5x^4 + 3x^3 + 3x^3 + 5x + 1$.
That is: the binomial coefficients modulo $n$ change.
For instance,

\ensuremath{\Varid{map}\;(\Varid{choose2}\;\mathrm{6})\;[\mskip1.5mu \mathrm{0}\mathinner{\ldotp\ldotp}\mathrm{6}\mskip1.5mu]}

is

1,6,15,20,15,6,1.

Modulo 7, we get

1,6,1,6,1,6,1.

\ensuremath{\Varid{map}\;(\Varid{choose2}\;\mathrm{7})\;[\mskip1.5mu \mathrm{0}\mathinner{\ldotp\ldotp}\mathrm{7}\mskip1.5mu]}

is

1,7,21,35,35,21,7,1.

Without big surprise, we see this modulo 7
drastically simplified:

1,0,0,0,0,0,0,1.

Here are addition and subtraction, which are very easy
to convert to modular arithmetic:

\begin{minipage}{\textwidth}
\begingroup\par\noindent\advance\leftskip\mathindent\(
\begin{pboxed}\SaveRestoreHook
\column{B}{@{}>{\hspre}l<{\hspost}@{}}%
\column{3}{@{}>{\hspre}l<{\hspost}@{}}%
\column{E}{@{}>{\hspre}l<{\hspost}@{}}%
\>[3]{}\Varid{addmp}\mathbin{::}\Conid{Zahl}\to \Conid{Poly}\;\Conid{Zahl}\to \Conid{Poly}\;\Conid{Zahl}\to \Conid{Poly}\;\Conid{Zahl}{}\<[E]%
\\
\>[3]{}\Varid{addmp}\;\Varid{n}\;\Varid{p1}\;\Varid{p2}\mathrel{=}\Varid{strich}\;(\mathbin{+})\;\Varid{p1}\;\Varid{p2}\mathbin{`\Varid{pmod}`}\Varid{n}{}\<[E]%
\\
\>[3]{}\Varid{submp}\mathbin{::}\Conid{Zahl}\to \Conid{Poly}\;\Conid{Zahl}\to \Conid{Poly}\;\Conid{Zahl}\to \Conid{Poly}\;\Conid{Zahl}{}\<[E]%
\\
\>[3]{}\Varid{submp}\;\Varid{n}\;\Varid{p1}\;\Varid{p2}\mathrel{=}\Varid{strich}\;(\mathbin{-})\;\Varid{p1}\;\Varid{p2}\mathbin{`\Varid{pmod}`}\Varid{n}{}\<[E]%
\ColumnHook
\end{pboxed}
\)\par\noindent\endgroup\resethooks
\end{minipage}

Multiplication:

\begin{minipage}{\textwidth}
\begingroup\par\noindent\advance\leftskip\mathindent\(
\begin{pboxed}\SaveRestoreHook
\column{B}{@{}>{\hspre}l<{\hspost}@{}}%
\column{3}{@{}>{\hspre}l<{\hspost}@{}}%
\column{5}{@{}>{\hspre}l<{\hspost}@{}}%
\column{12}{@{}>{\hspre}l<{\hspost}@{}}%
\column{16}{@{}>{\hspre}c<{\hspost}@{}}%
\column{16E}{@{}l@{}}%
\column{18}{@{}>{\hspre}c<{\hspost}@{}}%
\column{18E}{@{}l@{}}%
\column{19}{@{}>{\hspre}l<{\hspost}@{}}%
\column{21}{@{}>{\hspre}l<{\hspost}@{}}%
\column{32}{@{}>{\hspre}c<{\hspost}@{}}%
\column{32E}{@{}l@{}}%
\column{35}{@{}>{\hspre}l<{\hspost}@{}}%
\column{E}{@{}>{\hspre}l<{\hspost}@{}}%
\>[3]{}\Varid{mulmp}\mathbin{::}\Conid{Zahl}\to \Conid{Poly}\;\Conid{Zahl}\to \Conid{Poly}\;\Conid{Zahl}\to \Conid{Poly}\;\Conid{Zahl}{}\<[E]%
\\
\>[3]{}\Varid{mulmp}\;\Varid{p}\;\Varid{p1}\;\Varid{p2}{}\<[18]%
\>[18]{}\mid {}\<[18E]%
\>[21]{}\Varid{d2}\mathbin{>}\Varid{d1}{}\<[32]%
\>[32]{}\mathrel{=}{}\<[32E]%
\>[35]{}\Varid{mulmp}\;\Varid{p}\;\Varid{p2}\;\Varid{p1}{}\<[E]%
\\
\>[18]{}\mid {}\<[18E]%
\>[21]{}\Varid{otherwise}{}\<[32]%
\>[32]{}\mathrel{=}{}\<[32E]%
\>[35]{}\Conid{P}\;[\mskip1.5mu \Varid{m}\mathbin{`\Varid{mmod}`}\Varid{p}\mid \Varid{m}\leftarrow \Varid{strichf}\;(\mathbin{+})\;\Varid{ms}\mskip1.5mu]{}\<[E]%
\\
\>[3]{}\hsindent{2}{}\<[5]%
\>[5]{}\mathbf{where}\;{}\<[12]%
\>[12]{}\Varid{ms}{}\<[16]%
\>[16]{}\mathrel{=}{}\<[16E]%
\>[19]{}[\mskip1.5mu \Varid{mul1}\;\Varid{o}\;\Varid{i}\;(\Varid{coeffs}\;\Varid{p1})\;\Varid{c}\mid (\Varid{i},\Varid{c})\leftarrow \Varid{zip}\;[\mskip1.5mu \mathrm{0}\mathinner{\ldotp\ldotp}\mskip1.5mu]\;(\Varid{coeffs}\;\Varid{p2})\mskip1.5mu]{}\<[E]%
\\
\>[12]{}\Varid{d1}{}\<[16]%
\>[16]{}\mathrel{=}{}\<[16E]%
\>[19]{}\Varid{degree}\;\Varid{p1}{}\<[E]%
\\
\>[12]{}\Varid{d2}{}\<[16]%
\>[16]{}\mathrel{=}{}\<[16E]%
\>[19]{}\Varid{degree}\;\Varid{p2}{}\<[E]%
\\
\>[12]{}\Varid{o}{}\<[16]%
\>[16]{}\mathrel{=}{}\<[16E]%
\>[19]{}\Varid{modmul}\;\Varid{p}{}\<[E]%
\ColumnHook
\end{pboxed}
\)\par\noindent\endgroup\resethooks
\end{minipage}

We repeat the multiplication from above 

\ensuremath{\Varid{mul}\;(\Conid{P}\;[\mskip1.5mu \mathrm{1},\mathrm{2},\mathrm{3},\mathrm{4}\mskip1.5mu])\;(\Conid{P}\;[\mskip1.5mu \mathrm{5},\mathrm{6},\mathrm{7},\mathrm{8}\mskip1.5mu])} 

which was

\ensuremath{\Conid{P}\;[\mskip1.5mu \mathrm{5},\mathrm{16},\mathrm{34},\mathrm{60},\mathrm{61},\mathrm{52},\mathrm{32}\mskip1.5mu]}

Modulo 7, this result is

\ensuremath{\Conid{P}\;[\mskip1.5mu \mathrm{5},\mathrm{2},\mathrm{6},\mathrm{4},\mathrm{5},\mathrm{3},\mathrm{4}\mskip1.5mu]}.

The modulo multiplication

\ensuremath{\Varid{mulmp}\;\mathrm{7}\;(\Conid{P}\;[\mskip1.5mu \mathrm{1},\mathrm{2},\mathrm{3},\mathrm{4}\mskip1.5mu])\;(\Conid{P}\;[\mskip1.5mu \mathrm{5},\mathrm{6},\mathrm{0},\mathrm{1}\mskip1.5mu])}

yields the same result:

\ensuremath{\Conid{P}\;[\mskip1.5mu \mathrm{5},\mathrm{2},\mathrm{6},\mathrm{4},\mathrm{5},\mathrm{3},\mathrm{4}\mskip1.5mu]}

Division:

\begin{minipage}{\textwidth}
\begingroup\par\noindent\advance\leftskip\mathindent\(
\begin{pboxed}\SaveRestoreHook
\column{B}{@{}>{\hspre}l<{\hspost}@{}}%
\column{3}{@{}>{\hspre}l<{\hspost}@{}}%
\column{5}{@{}>{\hspre}l<{\hspost}@{}}%
\column{12}{@{}>{\hspre}l<{\hspost}@{}}%
\column{20}{@{}>{\hspre}c<{\hspost}@{}}%
\column{20E}{@{}l@{}}%
\column{23}{@{}>{\hspre}l<{\hspost}@{}}%
\column{28}{@{}>{\hspre}l<{\hspost}@{}}%
\column{30}{@{}>{\hspre}l<{\hspost}@{}}%
\column{32}{@{}>{\hspre}c<{\hspost}@{}}%
\column{32E}{@{}l@{}}%
\column{35}{@{}>{\hspre}l<{\hspost}@{}}%
\column{43}{@{}>{\hspre}l<{\hspost}@{}}%
\column{E}{@{}>{\hspre}l<{\hspost}@{}}%
\>[3]{}\Varid{divmp}\mathbin{::}\Conid{Zahl}\to \Conid{Poly}\;\Conid{Zahl}\to \Conid{Poly}\;\Conid{Zahl}\to (\Conid{Poly}\;\Conid{Zahl},\Conid{Poly}\;\Conid{Zahl}){}\<[E]%
\\
\>[3]{}\Varid{divmp}\;\Varid{p}\;(\Conid{P}\;\Varid{as})\;(\Conid{P}\;\Varid{bs})\mathrel{=}\mathbf{let}\;(\Varid{q},\Varid{r})\mathrel{=}\Varid{go}\;[\mskip1.5mu \mathrm{0}\mskip1.5mu]\;\Varid{as}\;\mathbf{in}\;(\Conid{P}\;\Varid{q},\Conid{P}\;\Varid{r}){}\<[E]%
\\
\>[3]{}\hsindent{2}{}\<[5]%
\>[5]{}\mathbf{where}\;{}\<[12]%
\>[12]{}\Varid{db}\mathrel{=}\Varid{degree}\;(\Conid{P}\;\Varid{bs}){}\<[E]%
\\
\>[12]{}\Varid{go}\;\Varid{q}\;\Varid{r}{}\<[20]%
\>[20]{}\mid {}\<[20E]%
\>[23]{}\Varid{degree}\;(\Conid{P}\;\Varid{r})\mathbin{<}\Varid{db}{}\<[43]%
\>[43]{}\mathrel{=}(\Varid{q},\Varid{r}){}\<[E]%
\\
\>[20]{}\mid {}\<[20E]%
\>[23]{}\Varid{null}\;\Varid{r}\mathrel{\vee}\Varid{r}\equiv [\mskip1.5mu \mathrm{0}\mskip1.5mu]{}\<[43]%
\>[43]{}\mathrel{=}(\Varid{q},\Varid{r}){}\<[E]%
\\
\>[20]{}\mid {}\<[20E]%
\>[23]{}\Varid{otherwise}{}\<[43]%
\>[43]{}\mathrel{=}{}\<[E]%
\\
\>[23]{}\mathbf{let}\;{}\<[28]%
\>[28]{}\Varid{t}{}\<[32]%
\>[32]{}\mathrel{=}{}\<[32E]%
\>[35]{}\Varid{modiv}\;\Varid{p}\;(\Varid{last}\;\Varid{r})\;(\Varid{last}\;\Varid{bs}){}\<[E]%
\\
\>[28]{}\Varid{d}{}\<[32]%
\>[32]{}\mathrel{=}{}\<[32E]%
\>[35]{}\Varid{degree}\;(\Conid{P}\;\Varid{r})\mathbin{-}\Varid{db}{}\<[E]%
\\
\>[28]{}\Varid{ts}{}\<[32]%
\>[32]{}\mathrel{=}{}\<[32E]%
\>[35]{}\Varid{zeros}\;\Varid{d}\plus [\mskip1.5mu \Varid{t}\mskip1.5mu]{}\<[E]%
\\
\>[28]{}\Varid{m}{}\<[32]%
\>[32]{}\mathrel{=}{}\<[32E]%
\>[35]{}\Varid{mulmlist}\;\Varid{p}\;\Varid{ts}\;\Varid{bs}{}\<[E]%
\\
\>[23]{}\mathbf{in}\;\Varid{go}\;{}\<[30]%
\>[30]{}[\mskip1.5mu \Varid{c}\mathbin{`\Varid{mmod}`}\Varid{p}\mid \Varid{c}\leftarrow \Varid{cleanz}\mathbin{\$}\Varid{strichlist}\;(\mathbin{+})\;\Varid{q}\;\Varid{ts}\mskip1.5mu]\;{}\<[E]%
\\
\>[30]{}[\mskip1.5mu \Varid{c}\mathbin{`\Varid{mmod}`}\Varid{p}\mid \Varid{c}\leftarrow \Varid{cleanz}\mathbin{\$}\Varid{strichlist}\;(\mathbin{-})\;\Varid{r}\;\Varid{m}\mskip1.5mu]{}\<[E]%
\ColumnHook
\end{pboxed}
\)\par\noindent\endgroup\resethooks
\end{minipage}

\acronym{gcd}:

\begin{minipage}{\textwidth}
\begingroup\par\noindent\advance\leftskip\mathindent\(
\begin{pboxed}\SaveRestoreHook
\column{B}{@{}>{\hspre}l<{\hspost}@{}}%
\column{3}{@{}>{\hspre}l<{\hspost}@{}}%
\column{16}{@{}>{\hspre}c<{\hspost}@{}}%
\column{16E}{@{}l@{}}%
\column{19}{@{}>{\hspre}l<{\hspost}@{}}%
\column{E}{@{}>{\hspre}l<{\hspost}@{}}%
\>[3]{}\Varid{gcdmp}\mathbin{::}\Conid{Zahl}\to \Conid{Poly}\;\Conid{Zahl}\to \Conid{Poly}\;\Conid{Zahl}\to \Conid{Poly}\;\Conid{Zahl}{}\<[E]%
\\
\>[3]{}\Varid{gcdmp}\;\Varid{p}\;\Varid{a}\;\Varid{b}{}\<[16]%
\>[16]{}\mid {}\<[16E]%
\>[19]{}\Varid{degree}\;\Varid{b}\mathbin{>}\Varid{degree}\;\Varid{a}\mathrel{=}\Varid{gcdmp}\;\Varid{p}\;\Varid{b}\;\Varid{a}{}\<[E]%
\\
\>[16]{}\mid {}\<[16E]%
\>[19]{}\Varid{zerop}\;\Varid{b}\mathrel{=}\Varid{a}{}\<[E]%
\\
\>[16]{}\mid {}\<[16E]%
\>[19]{}\Varid{otherwise}\mathrel{=}\mathbf{let}\;(\anonymous ,\Varid{r})\mathrel{=}\Varid{divmp}\;\Varid{p}\;\Varid{a}\;\Varid{b}\;\mathbf{in}\;\Varid{gcdmp}\;\Varid{p}\;\Varid{b}\;\Varid{r}{}\<[E]%
\ColumnHook
\end{pboxed}
\)\par\noindent\endgroup\resethooks
\end{minipage}

Let us try \ensuremath{\Varid{gcdmp}} on the variation we already tested above. 
We multiply the polynomial
$x^2 + 2x + 1$ by $3x^2 + 2x + 1$ modulo 7:

\ensuremath{\Varid{mulmp}\;\mathrm{7}\;(\Conid{P}\;[\mskip1.5mu \mathrm{1},\mathrm{2},\mathrm{1}\mskip1.5mu])\;(\Conid{P}\;[\mskip1.5mu \mathrm{1},\mathrm{2},\mathrm{3}\mskip1.5mu])}.

The result is \ensuremath{\Conid{P}\;[\mskip1.5mu \mathrm{1},\mathrm{4},\mathrm{1},\mathrm{1},\mathrm{3}\mskip1.5mu]}.

Now, we compute the \acronym{gcd} with \ensuremath{\Conid{P}\;[\mskip1.5mu \mathrm{1},\mathrm{5},\mathrm{10},\mathrm{10},\mathrm{5},\mathrm{1}\mskip1.5mu]} modulo 7:

\ensuremath{\Varid{gcdmp}\;\mathrm{7}\;(\Conid{P}\;[\mskip1.5mu \mathrm{1},\mathrm{5},\mathrm{3},\mathrm{3},\mathrm{5},\mathrm{1}\mskip1.5mu])\;(\Conid{P}\;[\mskip1.5mu \mathrm{1},\mathrm{4},\mathrm{1},\mathrm{1},\mathrm{3}\mskip1.5mu])}.

The result is \ensuremath{\Conid{P}\;[\mskip1.5mu \mathrm{1},\mathrm{2},\mathrm{1}\mskip1.5mu]}, as expected.

Finally, power:

\begin{minipage}{\textwidth}
\begingroup\par\noindent\advance\leftskip\mathindent\(
\begin{pboxed}\SaveRestoreHook
\column{B}{@{}>{\hspre}l<{\hspost}@{}}%
\column{3}{@{}>{\hspre}l<{\hspost}@{}}%
\column{5}{@{}>{\hspre}l<{\hspost}@{}}%
\column{12}{@{}>{\hspre}l<{\hspost}@{}}%
\column{22}{@{}>{\hspre}l<{\hspost}@{}}%
\column{35}{@{}>{\hspre}c<{\hspost}@{}}%
\column{35E}{@{}l@{}}%
\column{38}{@{}>{\hspre}l<{\hspost}@{}}%
\column{58}{@{}>{\hspre}l<{\hspost}@{}}%
\column{E}{@{}>{\hspre}l<{\hspost}@{}}%
\>[3]{}\Varid{powmp}\mathbin{::}\Conid{Zahl}\to \Conid{Zahl}\to \Conid{Poly}\;\Conid{Zahl}\to \Conid{Poly}\;\Conid{Zahl}{}\<[E]%
\\
\>[3]{}\Varid{powmp}\;\Varid{p}\;\Varid{f}\;\Varid{poly}\mathrel{=}\Varid{go}\;\Varid{f}\;(\Conid{P}\;[\mskip1.5mu \mathrm{1}\mskip1.5mu])\;\Varid{poly}{}\<[E]%
\\
\>[3]{}\hsindent{2}{}\<[5]%
\>[5]{}\mathbf{where}\;{}\<[12]%
\>[12]{}\Varid{go}\;\mathrm{0}\;\Varid{y}\;\anonymous {}\<[22]%
\>[22]{}\mathrel{=}\Varid{y}{}\<[E]%
\\
\>[12]{}\Varid{go}\;\mathrm{1}\;\Varid{y}\;\Varid{x}{}\<[22]%
\>[22]{}\mathrel{=}\Varid{mulmp}\;\Varid{p}\;\Varid{y}\;\Varid{x}{}\<[E]%
\\
\>[12]{}\Varid{go}\;\Varid{n}\;\Varid{y}\;\Varid{x}{}\<[22]%
\>[22]{}\mid \Varid{even}\;\Varid{n}{}\<[35]%
\>[35]{}\mathrel{=}{}\<[35E]%
\>[38]{}\Varid{go}\;(\Varid{n}\mathbin{\Varid{`div`}}\mathrm{2})\;\Varid{y}\;{}\<[58]%
\>[58]{}(\Varid{mulmp}\;\Varid{p}\;\Varid{x}\;\Varid{x}){}\<[E]%
\\
\>[22]{}\mid \Varid{otherwise}{}\<[35]%
\>[35]{}\mathrel{=}{}\<[35E]%
\>[38]{}\Varid{go}\;((\Varid{n}\mathbin{-}\mathrm{1})\mathbin{\Varid{`div`}}\mathrm{2})\;{}\<[58]%
\>[58]{}(\Varid{mulmp}\;\Varid{p}\;\Varid{y}\;\Varid{x})\;{}\<[E]%
\\
\>[58]{}(\Varid{mulmp}\;\Varid{p}\;\Varid{x}\;\Varid{x}){}\<[E]%
\ColumnHook
\end{pboxed}
\)\par\noindent\endgroup\resethooks
\end{minipage}

Here is a nice variant of Pascal's triangle generated by
\ensuremath{\Varid{map}\;(\lambda \Varid{x}\to \Varid{powmp}\;\mathrm{7}\;\Varid{x}\;(\Conid{P}\;[\mskip1.5mu \mathrm{1},\mathrm{1}\mskip1.5mu])\;[\mskip1.5mu \mathrm{1}\mathinner{\ldotp\ldotp}\mathrm{14}\mskip1.5mu]}:

\begin{minipage}{\textwidth}
\begin{center}
\ensuremath{\Conid{P}\;[\mskip1.5mu \mathrm{1},\mathrm{1}\mskip1.5mu]}\\
\ensuremath{\Conid{P}\;[\mskip1.5mu \mathrm{1},\mathrm{2},\mathrm{1}\mskip1.5mu]}\\
\ensuremath{\Conid{P}\;[\mskip1.5mu \mathrm{1},\mathrm{3},\mathrm{3},\mathrm{1}\mskip1.5mu]}\\
\ensuremath{\Conid{P}\;[\mskip1.5mu \mathrm{1},\mathrm{4},\mathrm{6},\mathrm{4},\mathrm{1}\mskip1.5mu]}\\
\ensuremath{\Conid{P}\;[\mskip1.5mu \mathrm{1},\mathrm{5},\mathrm{3},\mathrm{3},\mathrm{5},\mathrm{1}\mskip1.5mu]}\\
\ensuremath{\Conid{P}\;[\mskip1.5mu \mathrm{1},\mathrm{6},\mathrm{1},\mathrm{6},\mathrm{1},\mathrm{6},\mathrm{1}\mskip1.5mu]}\\
\ensuremath{\Conid{P}\;[\mskip1.5mu \mathrm{1},\mathrm{0},\mathrm{0},\mathrm{0},\mathrm{0},\mathrm{0},\mathrm{0},\mathrm{1}\mskip1.5mu]}\\
\ensuremath{\Conid{P}\;[\mskip1.5mu \mathrm{1},\mathrm{1},\mathrm{0},\mathrm{0},\mathrm{0},\mathrm{0},\mathrm{0},\mathrm{1},\mathrm{1}\mskip1.5mu]}\\
\ensuremath{\Conid{P}\;[\mskip1.5mu \mathrm{1},\mathrm{2},\mathrm{1},\mathrm{0},\mathrm{0},\mathrm{0},\mathrm{0},\mathrm{1},\mathrm{2},\mathrm{1}\mskip1.5mu]}\\
\ensuremath{\Conid{P}\;[\mskip1.5mu \mathrm{1},\mathrm{3},\mathrm{3},\mathrm{1},\mathrm{0},\mathrm{0},\mathrm{0},\mathrm{1},\mathrm{3},\mathrm{3},\mathrm{1}\mskip1.5mu]}\\
\ensuremath{\Conid{P}\;[\mskip1.5mu \mathrm{1},\mathrm{4},\mathrm{6},\mathrm{4},\mathrm{1},\mathrm{0},\mathrm{0},\mathrm{1},\mathrm{4},\mathrm{6},\mathrm{4},\mathrm{1}\mskip1.5mu]}\\
\ensuremath{\Conid{P}\;[\mskip1.5mu \mathrm{1},\mathrm{5},\mathrm{3},\mathrm{3},\mathrm{5},\mathrm{1},\mathrm{0},\mathrm{1},\mathrm{5},\mathrm{3},\mathrm{3},\mathrm{5},\mathrm{1}\mskip1.5mu]}\\
\ensuremath{\Conid{P}\;[\mskip1.5mu \mathrm{1},\mathrm{6},\mathrm{1},\mathrm{6},\mathrm{1},\mathrm{6},\mathrm{1},\mathrm{1},\mathrm{6},\mathrm{1},\mathrm{6},\mathrm{1},\mathrm{6},\mathrm{1}\mskip1.5mu]}\\
\ensuremath{\Conid{P}\;[\mskip1.5mu \mathrm{1},\mathrm{0},\mathrm{0},\mathrm{0},\mathrm{0},\mathrm{0},\mathrm{0},\mathrm{2},\mathrm{0},\mathrm{0},\mathrm{0},\mathrm{0},\mathrm{0},\mathrm{0},\mathrm{1}\mskip1.5mu]}
\end{center}
\end{minipage}

It is especially interesting to look at greater powers 
using exponents that are multiples of 7.
Before we continue with modular arithmetic,
which we need indeed to understand some of the deeper problems
related to polynomials, we will
investigate the application of polynomials using a famous device: 
Babbage's difference engine.
\section{The Difference Engine}
\ignore{
\begingroup\par\noindent\advance\leftskip\mathindent\(
\begin{pboxed}\SaveRestoreHook
\column{B}{@{}>{\hspre}l<{\hspost}@{}}%
\column{3}{@{}>{\hspre}l<{\hspost}@{}}%
\column{E}{@{}>{\hspre}l<{\hspost}@{}}%
\>[B]{}\mathbf{module}\;\Conid{PolyArith}{}\<[E]%
\\
\>[B]{}\mathbf{where}{}\<[E]%
\\
\>[B]{}\hsindent{3}{}\<[3]%
\>[3]{}\mathbf{import}\;\Conid{Natural}{}\<[E]%
\\
\>[B]{}\hsindent{3}{}\<[3]%
\>[3]{}\mathbf{import}\;\Conid{Zahl}{}\<[E]%
\\
\>[B]{}\hsindent{3}{}\<[3]%
\>[3]{}\mathbf{import}\;\Conid{Quoz}{}\<[E]%
\\
\>[B]{}\hsindent{3}{}\<[3]%
\>[3]{}\mathbf{import}\;\Conid{Real}{}\<[E]%
\\
\>[B]{}\hsindent{3}{}\<[3]%
\>[3]{}\mathbf{import}\;\Conid{NumSystem}{}\<[E]%
\\
\>[B]{}\hsindent{3}{}\<[3]%
\>[3]{}\mathbf{import}\;\Conid{PolyArith}{}\<[E]%
\ColumnHook
\end{pboxed}
\)\par\noindent\endgroup\resethooks
}

Polynomial arithmetic, as we have seen,
is very similar to number arithmetic.
What is the correspondent of interpreting
a number in a given numeral system
in the domain of polynomials?
Well, that is the \term{application} of the polynomial
to a given number. We would substitute $x$
for a number in the Field in which we are working
and just compute the formula.
For instance, the polynomial

\[
x^2 + x + 1
\]

can be applied to, say, 2.
Then we get the formula

\[
2^2 + 2 + 1,
\]

which is $4 + 2 + 1 = 7$.

For other values of $x$, it would of course
generate other values. For $x=0$, for instance,
it would give $0^2 + 0 + 1 = 1$; for $x=1$,
it is $1^2 + 1 + 1 = 3$; for $x=3$, it yields
$3^2 + 3 + 1 = 13$.

How would we apply a polynomial represented
by our Haskell type? We would need to go through the list
of coefficients, raise $x$ to the power of the weight
of each particular coefficient, multiply it by the coefficient
and, finally, add all the values together.
Here is an implementation:

\begin{minipage}{\textwidth}
\begingroup\par\noindent\advance\leftskip\mathindent\(
\begin{pboxed}\SaveRestoreHook
\column{B}{@{}>{\hspre}l<{\hspost}@{}}%
\column{3}{@{}>{\hspre}l<{\hspost}@{}}%
\column{21}{@{}>{\hspre}l<{\hspost}@{}}%
\column{E}{@{}>{\hspre}l<{\hspost}@{}}%
\>[3]{}\Varid{apply}\mathbin{::}\Conid{Num}\;\Varid{a}{}\<[21]%
\>[21]{}\Rightarrow \Conid{Poly}\;\Varid{a}\to \Varid{a}\to \Varid{a}{}\<[E]%
\\
\>[3]{}\Varid{apply}\;(\Conid{P}\;\Varid{cs})\;\Varid{x}{}\<[21]%
\>[21]{}\mathrel{=}\Varid{sum}\;[\mskip1.5mu \Varid{c}\mathbin{*}\Varid{x}\mathbin{\uparrow}\Varid{i}\mid (\Varid{i},\Varid{c})\leftarrow \Varid{zip}\;[\mskip1.5mu \mathrm{0}\mathinner{\ldotp\ldotp}\mskip1.5mu]\;\Varid{cs}\mskip1.5mu]{}\<[E]%
\ColumnHook
\end{pboxed}
\)\par\noindent\endgroup\resethooks
\end{minipage}

Let us try with a very simple polynomial, $x + 1$:

\begin{minipage}{\textwidth}
\ensuremath{\Varid{apply}\;(\Conid{P}\;[\mskip1.5mu \mathrm{1},\mathrm{1}\mskip1.5mu])\;\mathrm{0}} gives 1.\\
\ensuremath{\Varid{apply}\;(\Conid{P}\;[\mskip1.5mu \mathrm{1},\mathrm{1}\mskip1.5mu])\;\mathrm{1}} gives 2.\\
\ensuremath{\Varid{apply}\;(\Conid{P}\;[\mskip1.5mu \mathrm{1},\mathrm{1}\mskip1.5mu])\;\mathrm{2}} gives 3.\\
\ensuremath{\Varid{apply}\;(\Conid{P}\;[\mskip1.5mu \mathrm{1},\mathrm{1}\mskip1.5mu])\;\mathrm{3}} gives 4.
\end{minipage}

This polynomial, apparently, just counts the integers
adding one to the value to which we apply it.
It implements {\texttt i++}.

On the first sight, this result appears to be boring.
However, after a quick thought, there is a lesson to learn:
we get to know the polynomial, when we look
at the \term{sequence} it produces. So, let us implement
a function that maps \ensuremath{\Varid{apply}} to lists of numbers:

\begin{minipage}{\textwidth}
\begingroup\par\noindent\advance\leftskip\mathindent\(
\begin{pboxed}\SaveRestoreHook
\column{B}{@{}>{\hspre}l<{\hspost}@{}}%
\column{3}{@{}>{\hspre}l<{\hspost}@{}}%
\column{22}{@{}>{\hspre}l<{\hspost}@{}}%
\column{E}{@{}>{\hspre}l<{\hspost}@{}}%
\>[3]{}\Varid{mapply}\mathbin{::}\Conid{Num}\;\Varid{a}{}\<[22]%
\>[22]{}\Rightarrow \Conid{Poly}\;\Varid{a}\to [\mskip1.5mu \Varid{a}\mskip1.5mu]\to [\mskip1.5mu \Varid{a}\mskip1.5mu]{}\<[E]%
\\
\>[3]{}\Varid{mapply}\;\Varid{p}\mathrel{=}\Varid{map}\;(\Varid{apply}\;\Varid{p}){}\<[E]%
\ColumnHook
\end{pboxed}
\)\par\noindent\endgroup\resethooks
\end{minipage}

For simple polynomials, the sequences are predictable.
$x^2$, obviously, just produces the squares;
$x^3$ produces the cubes and so on.
Sequences created by powers of the simple polynomial $x+1$,
still, are quite predictable, \eg\:

\begin{minipage}{\textwidth}
\ensuremath{\Varid{mapply}\;(\Conid{P}\;[\mskip1.5mu \mathrm{1},\mathrm{2},\mathrm{1}\mskip1.5mu])\;[\mskip1.5mu \mathrm{0}\mathinner{\ldotp\ldotp}\mathrm{10}\mskip1.5mu]}: 1,4,9,16,25,36,49,64,81,100,121\\
\ensuremath{\Varid{mapply}\;(\Conid{P}\;[\mskip1.5mu \mathrm{1},\mathrm{3},\mathrm{3},\mathrm{1}\mskip1.5mu])\;[\mskip1.5mu \mathrm{0}\mathinner{\ldotp\ldotp}\mathrm{10}\mskip1.5mu]}: 1,8,27,64,125,216,343,512,729,1000,1331\\
\ensuremath{\Varid{mapply}\;(\Conid{P}\;[\mskip1.5mu \mathrm{1},\mathrm{4},\mathrm{6},\mathrm{4},\mathrm{1}\mskip1.5mu])\;[\mskip1.5mu \mathrm{0}\mathinner{\ldotp\ldotp}\mathrm{10}\mskip1.5mu]}: 1,16,81,256,625,1296,2401,4096,6561,10000,14641\\
\ensuremath{\Varid{mapply}\;(\Conid{P}\;[\mskip1.5mu \mathrm{1},\mathrm{5},\mathrm{10},\mathrm{10},\mathrm{5},\mathrm{1}\mskip1.5mu])\;[\mskip1.5mu \mathrm{0}\mathinner{\ldotp\ldotp}\mathrm{10}\mskip1.5mu]}:\\
1,32,243,1024,3125,7776,16807,32768,59049,100000,161051\\
\end{minipage}

The first line, easy to recognise, is the squares, but pushed one up,
\ie\ the application to 0 yields the value for $1^2$, 
the application to 1 yields the value for $2^2$ and so on.
The second, still easy to recognise,
is the cubes -- again pushed up by one.
The third line is the powers of four 
and the fourth line is the powers of five,
both pushed up by one.

That is not too surprising at the end, since
\ensuremath{\Conid{P}\;[\mskip1.5mu \mathrm{1},\mathrm{2},\mathrm{1}\mskip1.5mu]} is the result of squaring \ensuremath{\Conid{P}\;[\mskip1.5mu \mathrm{1},\mathrm{1}\mskip1.5mu]}, 
which generates the integers pushed one up;
\ensuremath{\Conid{P}\;[\mskip1.5mu \mathrm{1},\mathrm{3},\mathrm{3},\mathrm{1}\mskip1.5mu]} is the result of raising \ensuremath{\Conid{P}\;[\mskip1.5mu \mathrm{1},\mathrm{1}\mskip1.5mu]} to the third power
and so on.

Things become more interesting, when we deviate
from binomial coefficients. The sequence
produced by \ensuremath{\Varid{mappy}\;(\Conid{P}\;[\mskip1.5mu \mathrm{1},\mathrm{2},\mathrm{3},\mathrm{4}\mskip1.5mu])\;[\mskip1.5mu \mathrm{1}\mathinner{\ldotp\ldotp}\mathrm{10}\mskip1.5mu]}, for instance,
does not resemble such a simple sequence:
1, 10, 49, 142, 313, 586, 985, 1534, 2257, 3178, 4321.
Even the Online Encyclopedia has nothing interesting
to say about it.
The same is true for \ensuremath{\Varid{mappy}\;(\Conid{P}\;[\mskip1.5mu \mathrm{5},\mathrm{6},\mathrm{7},\mathrm{8}\mskip1.5mu])\;[\mskip1.5mu \mathrm{1}\mathinner{\ldotp\ldotp}\mathrm{10}\mskip1.5mu]},
which is 5, 26, 109, 302, 653, 1210, 2021, 3134, 4597, 6458, 8765.

This raises another interesting question:
given a sequence, is there a method by which 
we can we recognise the polynomial that created it?
Yes, there is. In fact, there are.
There was even a machine that helped guessing
polynomials from sequences. It was built in the early
$19^{th}$ century by Charles Babbage (1791 -- 1871),
an English polymath, mathematician, philosopher,
economist and inventor.

Babbage stands in the tradition of designers and constructors
of early computing machinery; predecessors of his
in this tradition were, for instance, 
Blaise Pascal (1623 -- 1662) and
Gottfried Wilhelm Leibniz (1646 -- 1716).
Babbage designed two series of machines,
first, the difference engines and, later, 
the analytical engines.

The analytical engine, unfortunately, was not built in his lifetime.
The final collapse of the project came 
in 1878, after Babbage's death in 1871, 
due to lack of finance. 
The analytical engine would have been 
a universal (Turing-complete) computer
very similar to our computers today,
but not working on electricity, but on steam and brawn.
It would have been programmed by punch cards that,
in Babbage's time, were used for controlling looms.
Programs would have resembled modern assembly languages
allowing control structures like selection and iteration.
In the context of a description of the analytical engine,
Ada Lovelace (1815 -- 1852), 
a friend of Babbage and daughter of Lord Byron,
described how to compute Bernoulli numbers with the machine.
She is, therefore, considered the first computer programmer
in history.

The difference engine, at which we will look here,
is much simpler. It was designed to analyse polynomials
and what it did was, according to Babbage, ``computing differences''.
During Babbage's lifetime, a first version was built and
sucessfully demonstrated. The construction
of a second, much more powerful version
which was financially backed by the government,
failed due to disputes between Babbage and his engineers.
This machine was finally built by the London Science Museum
in 1991 using material and engineering techniques available
in the $19^{th}$ century proving this way
that it was actually possible for Babbage and his engineers
to build such a machine.

The difference engine, as Babbage put it, computes differences,
namely the differences in a sequence of numbers.
It would take as input a sequence of the form

0,1,16,81,256,625,1296,2401,4096,6561,10000

and compute the differences between the single numbers:

\begin{equation}
\begin{array}{rcrcr}
  1 & - &  0 & = &   1 \\
 16 & - &  1 & = &  15 \\
 81 & - & 16 & = &  65 \\
256 & - & 81 & = & 175\\
\dots
\end{array}
\end{equation}

Here is a simple function that does this job for us:

\begin{minipage}{\textwidth}
\begingroup\par\noindent\advance\leftskip\mathindent\(
\begin{pboxed}\SaveRestoreHook
\column{B}{@{}>{\hspre}l<{\hspost}@{}}%
\column{3}{@{}>{\hspre}l<{\hspost}@{}}%
\column{19}{@{}>{\hspre}l<{\hspost}@{}}%
\column{E}{@{}>{\hspre}l<{\hspost}@{}}%
\>[3]{}\Varid{diffs}\mathbin{::}[\mskip1.5mu \Conid{Zahl}\mskip1.5mu]\to [\mskip1.5mu \Conid{Zahl}\mskip1.5mu]{}\<[E]%
\\
\>[3]{}\Varid{diffs}\;[\mskip1.5mu \mskip1.5mu]{}\<[19]%
\>[19]{}\mathrel{=}[\mskip1.5mu \mskip1.5mu]{}\<[E]%
\\
\>[3]{}\Varid{diffs}\;[\mskip1.5mu \anonymous \mskip1.5mu]{}\<[19]%
\>[19]{}\mathrel{=}[\mskip1.5mu \mskip1.5mu]{}\<[E]%
\\
\>[3]{}\Varid{diffs}\;(\Varid{a}\mathbin{:}\Varid{b}\mathbin{:}\Varid{cs}){}\<[19]%
\>[19]{}\mathrel{=}(\Varid{b}\mathbin{-}\Varid{a})\mathbin{:}\Varid{diffs}\;(\Varid{b}\mathbin{:}\Varid{cs}){}\<[E]%
\ColumnHook
\end{pboxed}
\)\par\noindent\endgroup\resethooks
\end{minipage}

Applied on the sequence above, \ensuremath{\Varid{diffs}} yields:

1,15,65,175,369,671,1105,1695,2465,3439

What is so special about it?
Perhaps, nothing. But let us repeat the process
using this sequence. It yields:

14,50,110,194,302,434,590,770,974

And once again:

36,60,84,108,132,156,180,204

And one more time:

24,24,24,24,24,24,24

Suddenly, we have a constant list.
How often did we apply \ensuremath{\Varid{diffs}}?
Four times -- and, as you may have realised,
the original sequence was generated by the polynomial
$x^4$, a polynomial of degree 4.
Is that coincidence?

For further investigation, we implement
the complete difference machine, which takes differences,
until it reaches a constant sequence.

\begin{minipage}{\textwidth}
\begingroup\par\noindent\advance\leftskip\mathindent\(
\begin{pboxed}\SaveRestoreHook
\column{B}{@{}>{\hspre}l<{\hspost}@{}}%
\column{3}{@{}>{\hspre}l<{\hspost}@{}}%
\column{5}{@{}>{\hspre}l<{\hspost}@{}}%
\column{12}{@{}>{\hspre}l<{\hspost}@{}}%
\column{16}{@{}>{\hspre}c<{\hspost}@{}}%
\column{16E}{@{}l@{}}%
\column{19}{@{}>{\hspre}l<{\hspost}@{}}%
\column{29}{@{}>{\hspre}l<{\hspost}@{}}%
\column{31}{@{}>{\hspre}l<{\hspost}@{}}%
\column{E}{@{}>{\hspre}l<{\hspost}@{}}%
\>[3]{}\Varid{dengine}\mathbin{::}[\mskip1.5mu \Conid{Zahl}\mskip1.5mu]\to [\mskip1.5mu [\mskip1.5mu \Conid{Zahl}\mskip1.5mu]\mskip1.5mu]{}\<[E]%
\\
\>[3]{}\Varid{dengine}\;\Varid{cs}{}\<[16]%
\>[16]{}\mid {}\<[16E]%
\>[19]{}\Varid{constant}\;\Varid{cs}\mathrel{=}[\mskip1.5mu \mskip1.5mu]{}\<[E]%
\\
\>[16]{}\mid {}\<[16E]%
\>[19]{}\Varid{otherwise}{}\<[31]%
\>[31]{}\mathrel{=}\Varid{ds}\mathbin{:}\Varid{dengine}\;\Varid{ds}{}\<[E]%
\\
\>[3]{}\hsindent{2}{}\<[5]%
\>[5]{}\mathbf{where}\;{}\<[12]%
\>[12]{}\Varid{ds}\mathrel{=}\Varid{diffs}\;\Varid{cs}{}\<[E]%
\\
\>[12]{}\Varid{constant}\;[\mskip1.5mu \mskip1.5mu]{}\<[29]%
\>[29]{}\mathrel{=}\Conid{True}{}\<[E]%
\\
\>[12]{}\Varid{constant}\;[\mskip1.5mu \anonymous \mskip1.5mu]{}\<[29]%
\>[29]{}\mathrel{=}\Conid{True}{}\<[E]%
\\
\>[12]{}\Varid{constant}\;(\Varid{x}\mathbin{:}\Varid{xs}){}\<[29]%
\>[29]{}\mathrel{=}\Varid{all}\;(\equiv \Varid{x})\;\Varid{xs}{}\<[E]%
\ColumnHook
\end{pboxed}
\)\par\noindent\endgroup\resethooks
\end{minipage}

Note that we restrict coefficients to integers.
This is just for clarity.
Usually, polynomials are defined over a field,
such as the rational or the real numbers.

To confirm our suspicion that the difference engine
creates $n$ difference sequences for a polynomial of degree $n$,
we apply the engine on $x$, $x^2$, $x^3$, $x^4$ and $x^5$
and count the sequences it creates:

\begin{minipage}{\textwidth}
\ensuremath{\Varid{length}\;(\Varid{dengine}\;(\Varid{mapply}\;(\Conid{P}\;[\mskip1.5mu \mathrm{0},\mathrm{1}\mskip1.5mu])\;[\mskip1.5mu \mathrm{0}\mathinner{\ldotp\ldotp}\mathrm{32}\mskip1.5mu]))}: 1\\
\ensuremath{\Varid{length}\;(\Varid{dengine}\;(\Varid{mapply}\;(\Conid{P}\;[\mskip1.5mu \mathrm{0},\mathrm{0},\mathrm{1}\mskip1.5mu])\;[\mskip1.5mu \mathrm{0}\mathinner{\ldotp\ldotp}\mathrm{32}\mskip1.5mu]))}: 2\\
\ensuremath{\Varid{length}\;(\Varid{dengine}\;(\Varid{mapply}\;(\Conid{P}\;[\mskip1.5mu \mathrm{0},\mathrm{0},\mathrm{0},\mathrm{1}\mskip1.5mu])\;[\mskip1.5mu \mathrm{0}\mathinner{\ldotp\ldotp}\mathrm{32}\mskip1.5mu]))}: 3\\
\ensuremath{\Varid{length}\;(\Varid{dengine}\;(\Varid{mapply}\;(\Conid{P}\;[\mskip1.5mu \mathrm{0},\mathrm{0},\mathrm{0},\mathrm{0},\mathrm{1}\mskip1.5mu])\;[\mskip1.5mu \mathrm{0}\mathinner{\ldotp\ldotp}\mathrm{32}\mskip1.5mu]))}: 4\\
\ensuremath{\Varid{length}\;(\Varid{dengine}\;(\Varid{mapply}\;(\Conid{P}\;[\mskip1.5mu \mathrm{0},\mathrm{0},\mathrm{0},\mathrm{0},\mathrm{0},\mathrm{1}\mskip1.5mu])\;[\mskip1.5mu \mathrm{0}\mathinner{\ldotp\ldotp}\mathrm{32}\mskip1.5mu]))}: 5
\end{minipage}

The engine already has a purpose:
it tells us the degree of the polynomial
that generates a given sequence.
It can do much more, though.
For instance, it lets us predict the next value
in the sequence.
To do so, we take the constant difference 
from the last sequence and add it to 
the last difference of the previous sequence;
we take that result and add it to the previous sequence
and so on, until we reach the first sequence.
Consider the sequence and its differences from above:

\begin{minipage}{\textwidth}
0,1,16,81,256,625,1296,2401,4096,6561,10000\\
1,15,65,175,369,671,1105,1695,2465,3439\\
14,50,110,194,302,434,590,770,974\\
36,60,84,108,132,156,180,204\\
24,24,24,24,24,24,24
\end{minipage}

We start at the bottom and compute $204 + 24 = 228$.
This is the next difference of the previous sequence.
We compute $974 + 228 = 1202$. We go one line up and
compute $3439 + 1202 = 4641$. This, finally, is the difference
to the next value in the input sequence, which, hence, is
$10000 + 4641 = 14641$ and, indeed, $11^4$.
Even without knowing the polynomial that actually generates
the sequence, we are now able to continue this sequence.
Here is a function that does that for us:

\begin{minipage}{\textwidth}
\begingroup\par\noindent\advance\leftskip\mathindent\(
\begin{pboxed}\SaveRestoreHook
\column{B}{@{}>{\hspre}l<{\hspost}@{}}%
\column{3}{@{}>{\hspre}l<{\hspost}@{}}%
\column{5}{@{}>{\hspre}l<{\hspost}@{}}%
\column{12}{@{}>{\hspre}l<{\hspost}@{}}%
\column{21}{@{}>{\hspre}l<{\hspost}@{}}%
\column{24}{@{}>{\hspre}l<{\hspost}@{}}%
\column{E}{@{}>{\hspre}l<{\hspost}@{}}%
\>[3]{}\Varid{predict}\mathbin{::}[\mskip1.5mu [\mskip1.5mu \Conid{Zahl}\mskip1.5mu]\mskip1.5mu]\to [\mskip1.5mu \Conid{Zahl}\mskip1.5mu]\to \Conid{Maybe}\;\Conid{Zahl}{}\<[E]%
\\
\>[3]{}\Varid{predict}\;\Varid{ds}\;\Varid{xs}\mathrel{=}{}\<[21]%
\>[21]{}\mathbf{case}\;\Varid{go}\;(\Varid{reverse}\;\Varid{ds})\;\mathbf{of}{}\<[E]%
\\
\>[21]{}\mathrm{0}{}\<[24]%
\>[24]{}\to \Conid{Nothing}{}\<[E]%
\\
\>[21]{}\Varid{d}{}\<[24]%
\>[24]{}\to \Conid{Just}\;(\Varid{d}\mathbin{+}(\Varid{last}\;\Varid{xs})){}\<[E]%
\\
\>[3]{}\hsindent{2}{}\<[5]%
\>[5]{}\mathbf{where}\;{}\<[12]%
\>[12]{}\Varid{go}\;[\mskip1.5mu \mskip1.5mu]\mathrel{=}\mathrm{0}{}\<[E]%
\\
\>[12]{}\Varid{go}\;(\Varid{a}\mathbin{:}\Varid{cs})\mathrel{=}\Varid{last}\;\Varid{a}\mathbin{+}\Varid{go}\;\Varid{cs}{}\<[E]%
\ColumnHook
\end{pboxed}
\)\par\noindent\endgroup\resethooks
\end{minipage}

The function takes two arguments:
the first is the list of difference sequences and
the second is the original sequence.
We apply \ensuremath{\Varid{go}} on the reverse of the sequences
(because we are working backwards).
For each sequence in this list, we get the last
and add it to the last of the previous until
we have exhausted the list.
If \ensuremath{\Varid{go}} yields 0, we assume that something went wrong.
The list of sequences may have been empty in the first place.
Otherwise, we add the result to the last of the original list.

Here are some more examples:

\begin{minipage}{\textwidth}
\ensuremath{\mathbf{let}\;\Varid{s}\mathrel{=}\Varid{mapply}\;(\Conid{P}\;[\mskip1.5mu \mathrm{0},\mathrm{1}\mskip1.5mu])\;[\mskip1.5mu \mathrm{0}\mathinner{\ldotp\ldotp}\mathrm{10}\mskip1.5mu]\;\mathbf{in}\;\Varid{predict}\;(\Varid{dengine}\;\Varid{s})\;\Varid{s}}: 11\\
\ensuremath{\mathbf{let}\;\Varid{s}\mathrel{=}\Varid{mapply}\;(\Conid{P}\;[\mskip1.5mu \mathrm{0},\mathrm{0},\mathrm{1}\mskip1.5mu])\;[\mskip1.5mu \mathrm{0}\mathinner{\ldotp\ldotp}\mathrm{10}\mskip1.5mu]\;\mathbf{in}\;\Varid{predict}\;(\Varid{dengine}\;\Varid{s})\;\Varid{s}}: 121\\
\ensuremath{\mathbf{let}\;\Varid{s}\mathrel{=}\Varid{mapply}\;(\Conid{P}\;[\mskip1.5mu \mathrm{0},\mathrm{0},\mathrm{0},\mathrm{1}\mskip1.5mu])\;[\mskip1.5mu \mathrm{0}\mathinner{\ldotp\ldotp}\mathrm{10}\mskip1.5mu]\;\mathbf{in}\;\Varid{predict}\;(\Varid{dengine}\;\Varid{s})\;\Varid{s}}: 1331\\
\ensuremath{\mathbf{let}\;\Varid{s}\mathrel{=}\Varid{mapply}\;(\Conid{P}\;[\mskip1.5mu \mathrm{0},\mathrm{0},\mathrm{0},\mathrm{0},\mathrm{1}\mskip1.5mu])\;[\mskip1.5mu \mathrm{0}\mathinner{\ldotp\ldotp}\mathrm{10}\mskip1.5mu]\;\mathbf{in}\;\Varid{predict}\;(\Varid{dengine}\;\Varid{s})\;\Varid{s}}: 14641\\
\ensuremath{\mathbf{let}\;\Varid{s}\mathrel{=}\Varid{mapply}\;(\Conid{P}\;[\mskip1.5mu \mathrm{0},\mathrm{0},\mathrm{0},\mathrm{0},\mathrm{0},\mathrm{1}\mskip1.5mu])\;[\mskip1.5mu \mathrm{0}\mathinner{\ldotp\ldotp}\mathrm{10}\mskip1.5mu]\;\mathbf{in}\;\Varid{predict}\;(\Varid{dengine}\;\Varid{s})\;\Varid{s}}: 161051
\end{minipage}

Let us go back to the question of how to find the polynomial
given the sequence that this polynomial generates.
With the help of the difference engine, we already know
the degree of the polynomial. 
Supposed, we know that the first element in the sequence
was generated applying 0 to the unknown polynomial and
the second one was generated applying 1,
the third by applying 2 and so on,
we have all information we need.

From the degree, we know the form of the polynomial.
A polynomial of degree 1 has the form $a_1x + a_2$;
a polynomial of degree 2 has the form $a_1x^2 + a_2x + a_3$;
a polynomial of degree 3 has the form $a_1x^3 + a_2x^2 + a_3x + a_4$
and so on.

Since we know the values to which the polynomial is applied,
we can easily compute the value of the $x$-part of the terms.
They are that value raised to the power of the weight.
The challenge, then, is to find the coefficient by which
that value is multiplied.

The first element in the sequence, the one created by applying
the polynomial to 0, is just the last coefficient,
the one ``without'' an $x$, since the other terms ``disappear'',
when we apply to 0. Consider for example a polynomial of the form
$x^2 + x + a$. When we apply it to 0,
we get $0^2 + 0 + a = c$, where $c$ is the first
value in the sequence. Thus, $a=c$.

The second element is 1 applied to the formula and, therefore,
all terms equal their coefficients, since $cx^n$, for $x=1$, 
is just $c$. The third element results from applying 2 to the polynomial,
it hence adheres to a formula where unknown values (the coefficients)
are multiplied by $2$, $2^2=4$, $2^3=8$ and so on.

In other words, for a polynomial of degree $n$, we can devise
a system of linear equations with $n+1$ unknowns and
the $n+1$ first elements of the sequence as constant values.
A polynomial of degree 2, for instance, yields the system

\begin{equation}
\begin{array}{rcrcrcr}
    &   &    &   & a  & = &  a_1 \\
  a & + & b  & + & c  & = &  a_2 \\
  a & + & 2b & + & 4c & = &  a_3
\end{array}
\end{equation}

where the constant numbers $a_1$, $a_2$ and $a_3$
are the first three elements of the sequence.
A polynomial of degree 3 would generate the system

\begin{equation}
\begin{array}{rcrcrcrcr}
    &   &    &   &    &   &   a & = &  a_1 \\
  a & + &  b & + &  c & + &   d & = &  a_2 \\
  a & + & 2b & + & 4c & + &  8d & = &  a_3 \\
  a & + & 3b & + & 9c & + & 27d & = &  a_4 
\end{array}
\end{equation}

We have already learnt how to solve such systems:
we can apply Gaussian elimination.
The result of the elminiation is 
the coefficients of the generating polynomial,
which are the unknowns in the linear equations.
The known values (which we would call the coefficients
in a linear equation) are the values obtained
by computing $x^i$ where $i$ is the weight 
of the coefficient.
Here is a function to extract the known values,
the $x$es raised to the weight, from a given 
sequence with a given degree:

\begin{minipage}{\textwidth}
\begingroup\par\noindent\advance\leftskip\mathindent\(
\begin{pboxed}\SaveRestoreHook
\column{B}{@{}>{\hspre}l<{\hspost}@{}}%
\column{3}{@{}>{\hspre}l<{\hspost}@{}}%
\column{5}{@{}>{\hspre}l<{\hspost}@{}}%
\column{12}{@{}>{\hspre}l<{\hspost}@{}}%
\column{20}{@{}>{\hspre}c<{\hspost}@{}}%
\column{20E}{@{}l@{}}%
\column{23}{@{}>{\hspre}l<{\hspost}@{}}%
\column{28}{@{}>{\hspre}l<{\hspost}@{}}%
\column{34}{@{}>{\hspre}c<{\hspost}@{}}%
\column{34E}{@{}l@{}}%
\column{37}{@{}>{\hspre}l<{\hspost}@{}}%
\column{E}{@{}>{\hspre}l<{\hspost}@{}}%
\>[3]{}\Varid{genCoeff}\mathbin{::}\Conid{Zahl}\to \Conid{Zahl}\to \Conid{Zahl}\to [\mskip1.5mu \Conid{Zahl}\mskip1.5mu]{}\<[E]%
\\
\>[3]{}\Varid{genCoeff}\;\Varid{d}\;\Varid{n}\;\Varid{x}\mathrel{=}\Varid{go}\;\mathrm{0}\;\Varid{x}{}\<[E]%
\\
\>[3]{}\hsindent{2}{}\<[5]%
\>[5]{}\mathbf{where}\;{}\<[12]%
\>[12]{}\Varid{go}\;\Varid{i}\;\Varid{x}{}\<[20]%
\>[20]{}\mid {}\<[20E]%
\>[23]{}\Varid{i}\mathbin{>}{}\<[28]%
\>[28]{}\Varid{d}{}\<[34]%
\>[34]{}\mathrel{=}{}\<[34E]%
\>[37]{}[\mskip1.5mu \Varid{x}\mskip1.5mu]{}\<[E]%
\\
\>[20]{}\mid {}\<[20E]%
\>[23]{}\Varid{otherwise}{}\<[34]%
\>[34]{}\mathrel{=}{}\<[34E]%
\>[37]{}\Varid{n}\mathbin{\uparrow}\Varid{i}\mathbin{:}\Varid{go}\;(\Varid{i}\mathbin{+}\mathrm{1})\;\Varid{x}{}\<[E]%
\ColumnHook
\end{pboxed}
\)\par\noindent\endgroup\resethooks
\end{minipage}

Here, $d$ is the degree of the polynomial,
$n$ is the value to which the polynomial is applied
and $x$ is the result, \ie\ the value from the sequence.
The local function \ensuremath{\Varid{go}} repeats from 0 to $d$,
raising $n$, the input value, to the current weight
and adding it to the resulting list.
At the end, when we have reached a value greater 
than the degree, we add the value from the sequence
as known constant yielding one line of the system
of linear equations.

When we apply \ensuremath{\Varid{genCoeff}} on the the sequence
generated by $x^4$, we would have:

\begin{minipage}{\textwidth}
\ensuremath{\Varid{genCoeff}\;\mathrm{4}\;\mathrm{0}\;\mathrm{0}} resulting in \ensuremath{[\mskip1.5mu \mathrm{1},\mathrm{0},\mathrm{0},\mathrm{0},\mathrm{0},\mathrm{0}\mskip1.5mu]}\\
\ensuremath{\Varid{genCoeff}\;\mathrm{4}\;\mathrm{1}\;\mathrm{1}} resulting in \ensuremath{[\mskip1.5mu \mathrm{1},\mathrm{1},\mathrm{1},\mathrm{1},\mathrm{1},\mathrm{1}\mskip1.5mu]}\\
\ensuremath{\Varid{genCoeff}\;\mathrm{4}\;\mathrm{2}\;\mathrm{16}} resulting in \ensuremath{[\mskip1.5mu \mathrm{1},\mathrm{2},\mathrm{4},\mathrm{8},\mathrm{16},\mathrm{16}\mskip1.5mu]}\\
\ensuremath{\Varid{genCoeff}\;\mathrm{4}\;\mathrm{3}\;\mathrm{81}} resulting in \ensuremath{[\mskip1.5mu \mathrm{1},\mathrm{3},\mathrm{9},\mathrm{27},\mathrm{81},\mathrm{81}\mskip1.5mu]}\\
\ensuremath{\Varid{genCoeff}\;\mathrm{4}\;\mathrm{4}\;\mathrm{256}} resulting in \ensuremath{[\mskip1.5mu \mathrm{1},\mathrm{4},\mathrm{16},\mathrm{64},\mathrm{256},\mathrm{256}\mskip1.5mu]}
\end{minipage}

Note that the results are very regular:
we see constant 1 in the first column,
the natural numbers in the first column,
the squares in the third, the cubes in the fourth and
the powers in the fifth and sixth column.
This are just the values for $x^i$, 
for $i \in \lbrace 0\dots 4\rbrace$.
Since the value in the sixth column, the one we took
from the sequence, equals the value in the fifth column,
we can already guess that the polynomial is simply $x^4$.
Here is another sequence, generated by a secret polynomial:

14,62,396,1544,4322,9834,19472,34916,58134,91382,137204

We compute the difference lists using 
\ensuremath{\Varid{dengine}} as \ensuremath{\Varid{ds}} and compute the degree of the polynomial
using \ensuremath{\Varid{length}\;\Varid{ds}}. The result is 4.
Now we call \ensuremath{\Varid{genCoeff}} on the first four elements of the sequence:

\begin{minipage}{\textwidth}
\ensuremath{\Varid{genCoeff}\;\mathrm{4}\;\mathrm{0}\;\mathrm{14}} resulting in \ensuremath{[\mskip1.5mu \mathrm{1},\mathrm{0},\mathrm{0},\mathrm{0},\mathrm{0},\mathrm{14}\mskip1.5mu]}\\
\ensuremath{\Varid{genCoeff}\;\mathrm{4}\;\mathrm{1}\;\mathrm{62}} resulting in \ensuremath{[\mskip1.5mu \mathrm{1},\mathrm{1},\mathrm{1},\mathrm{1},\mathrm{1},\mathrm{62}\mskip1.5mu]}\\
\ensuremath{\Varid{genCoeff}\;\mathrm{4}\;\mathrm{2}\;\mathrm{396}} resulting in \ensuremath{[\mskip1.5mu \mathrm{1},\mathrm{2},\mathrm{4},\mathrm{8},\mathrm{16},\mathrm{396}\mskip1.5mu]}\\
\ensuremath{\Varid{genCoeff}\;\mathrm{4}\;\mathrm{3}\;\mathrm{1544}} resulting in \ensuremath{[\mskip1.5mu \mathrm{1},\mathrm{3},\mathrm{9},\mathrm{27},\mathrm{81},\mathrm{1544}\mskip1.5mu]}\\
\ensuremath{\Varid{genCoeff}\;\mathrm{4}\;\mathrm{4}\;\mathrm{4322}} resulting in \ensuremath{[\mskip1.5mu \mathrm{1},\mathrm{4},\mathrm{16},\mathrm{64},\mathrm{256},\mathrm{4322}\mskip1.5mu]}
\end{minipage}

We use \ensuremath{\Varid{genCoeff}} to create a matrix representing
the entire system of equations:

\begin{minipage}{\textwidth}
\begingroup\par\noindent\advance\leftskip\mathindent\(
\begin{pboxed}\SaveRestoreHook
\column{B}{@{}>{\hspre}l<{\hspost}@{}}%
\column{3}{@{}>{\hspre}l<{\hspost}@{}}%
\column{5}{@{}>{\hspre}l<{\hspost}@{}}%
\column{12}{@{}>{\hspre}l<{\hspost}@{}}%
\column{21}{@{}>{\hspre}l<{\hspost}@{}}%
\column{25}{@{}>{\hspre}c<{\hspost}@{}}%
\column{25E}{@{}l@{}}%
\column{28}{@{}>{\hspre}l<{\hspost}@{}}%
\column{39}{@{}>{\hspre}c<{\hspost}@{}}%
\column{39E}{@{}l@{}}%
\column{42}{@{}>{\hspre}l<{\hspost}@{}}%
\column{E}{@{}>{\hspre}l<{\hspost}@{}}%
\>[3]{}\Varid{findCoeffs}\mathbin{::}[\mskip1.5mu [\mskip1.5mu \Conid{Zahl}\mskip1.5mu]\mskip1.5mu]\to [\mskip1.5mu \Conid{Zahl}\mskip1.5mu]\to \Conid{\Conid{L}.Matrix}\;\Conid{Zahl}{}\<[E]%
\\
\>[3]{}\Varid{findCoeffs}\;\Varid{ds}\;\Varid{seq}\mathrel{=}\Conid{\Conid{L}.M}\;(\Varid{go}\;\mathrm{0}\;\Varid{seq}){}\<[E]%
\\
\>[3]{}\hsindent{2}{}\<[5]%
\>[5]{}\mathbf{where}\;{}\<[12]%
\>[12]{}\Varid{d}\mathrel{=}\Varid{fromIntegral}\;(\Varid{length}\;\Varid{ds}){}\<[E]%
\\
\>[12]{}\Varid{go}\;\anonymous \;[\mskip1.5mu \mskip1.5mu]{}\<[21]%
\>[21]{}\mathrel{=}[\mskip1.5mu \mskip1.5mu]{}\<[E]%
\\
\>[12]{}\Varid{go}\;\Varid{n}\;(\Varid{x}\mathbin{:}\Varid{xs}){}\<[25]%
\>[25]{}\mid {}\<[25E]%
\>[28]{}\Varid{n}\mathbin{>}\Varid{d}{}\<[39]%
\>[39]{}\mathrel{=}{}\<[39E]%
\>[42]{}[\mskip1.5mu \mskip1.5mu]{}\<[E]%
\\
\>[25]{}\mid {}\<[25E]%
\>[28]{}\Varid{otherwise}{}\<[39]%
\>[39]{}\mathrel{=}{}\<[39E]%
\>[42]{}\Varid{genCoeff}\;\Varid{d}\;\Varid{n}\;\Varid{x}\mathbin{:}\Varid{go}\;(\Varid{n}\mathbin{+}\mathrm{1})\;\Varid{xs}{}\<[E]%
\ColumnHook
\end{pboxed}
\)\par\noindent\endgroup\resethooks
\end{minipage}

The function \ensuremath{\Varid{findCoeffs}} receives 
the list of difference sequences created by \ensuremath{\Varid{dengine}} and
the original sequence.
It computes the degree of the generating polynomial
as \ensuremath{\Varid{length}\;\Varid{ds}} and, then, it goes through the 
first \ensuremath{\Varid{d}} elements of the sequence calling \ensuremath{\Varid{genCoeff}}
with \ensuremath{\Varid{d}}, the known input value, $n$, and $x$,
the element of the sequence.
For the sequence generated by $x^4$, we obtain
\ensuremath{\Conid{M}\;[\mskip1.5mu [\mskip1.5mu \mathrm{1},\mathrm{0},\mathrm{0},\mathrm{0},\mathrm{0},\mathrm{0}\mskip1.5mu],[\mskip1.5mu \mathrm{1},\mathrm{1},\mathrm{1},\mathrm{1},\mathrm{1},\mathrm{1}\mskip1.5mu],[\mskip1.5mu \mathrm{1},\mathrm{2},\mathrm{4},\mathrm{8},\mathrm{16},\mathrm{16}\mskip1.5mu],}
\ensuremath{[\mskip1.5mu \mathrm{1},\mathrm{3},\mathrm{9},\mathrm{27},\mathrm{81},\mathrm{81}\mskip1.5mu],[\mskip1.5mu \mathrm{1},\mathrm{4},\mathrm{16},\mathrm{64},\mathrm{256},\mathrm{256}\mskip1.5mu]\mskip1.5mu]}, 
which corresponds to the matrix

\[
\begin{pmatrix}
 1 &  0 &   0 &   0 &   0 &    0\\
 1 &  1 &   1 &   1 &   1 &    1\\
 1 &  2 &   4 &   8 &  16 &   16\\
 1 &  3 &   9 &  27 &  81 &   81\\
 1 &  4 &  16 &  64 & 256 &  256
\end{pmatrix}
\]

For the sequence of the unknown polynomial, we obtain
\ensuremath{\Conid{M}\;[\mskip1.5mu [\mskip1.5mu \mathrm{1},\mathrm{0},\mathrm{0},\mathrm{0},\mathrm{0},\mathrm{14}\mskip1.5mu],\mid [\mskip1.5mu \mathrm{1},\mathrm{1},\mathrm{1},\mathrm{1},\mathrm{1},\mathrm{62}\mskip1.5mu],}
\ensuremath{[\mskip1.5mu \mathrm{1},\mathrm{2},\mathrm{4},\mathrm{8},\mathrm{16},\mathrm{396}\mskip1.5mu],[\mskip1.5mu \mathrm{1},\mathrm{3},\mathrm{9},\mathrm{27},\mathrm{81},\mathrm{1544}\mskip1.5mu],[\mskip1.5mu \mathrm{1},\mathrm{4},\mathrm{16},\mathrm{64},\mathrm{256},\mathrm{4322}\mskip1.5mu]\mskip1.5mu]},
which corresponds to the matrix:

\[
\begin{pmatrix}
 1 &  0 &   0 &   0 &   0 &   14\\
 1 &  1 &   1 &   1 &   1 &   62\\
 1 &  2 &   4 &   8 &  16 &  396\\
 1 &  3 &   9 &  27 &  81 & 1544\\
 1 &  4 &  16 &  64 & 256 & 4322
\end{pmatrix}
\]

The next steps are simple. We create the echelon form
and solve by back-substitution. The following function
puts all the bits together to find the generating polynomial:

\begin{minipage}{\textwidth}
\begingroup\par\noindent\advance\leftskip\mathindent\(
\begin{pboxed}\SaveRestoreHook
\column{B}{@{}>{\hspre}l<{\hspost}@{}}%
\column{3}{@{}>{\hspre}l<{\hspost}@{}}%
\column{E}{@{}>{\hspre}l<{\hspost}@{}}%
\>[3]{}\Varid{findGen}\mathbin{::}[\mskip1.5mu [\mskip1.5mu \Conid{Zahl}\mskip1.5mu]\mskip1.5mu]\to [\mskip1.5mu \Conid{Zahl}\mskip1.5mu]\to [\mskip1.5mu \Conid{Quoz}\mskip1.5mu]{}\<[E]%
\\
\>[3]{}\Varid{findGen}\;\Varid{ds}\mathrel{=}\Varid{\Conid{L}.backsub}\mathbin{\circ}\Varid{\Conid{L}.echelon}\mathbin{\circ}\Varid{findCoeffs}\;\Varid{ds}{}\<[E]%
\ColumnHook
\end{pboxed}
\)\par\noindent\endgroup\resethooks
\end{minipage}

Applied on the difference list and the sequence
generated by $x^4$, \ensuremath{\Varid{findGen}} yields:
\ensuremath{[\mskip1.5mu \mathrm{0}\mathbin{\%}\mathrm{1},\mathrm{0}\mathbin{\%}\mathrm{1},\mathrm{0}\mathbin{\%}\mathrm{1},\mathrm{0}\mathbin{\%}\mathrm{1},\mathrm{1}\mathbin{\%}\mathrm{1}\mskip1.5mu]},
which indeed corresponds to the polynomial $x^4$. 
For the sequence generated by the unknown polynomial,
we get:
\ensuremath{[\mskip1.5mu \mathrm{14}\mathbin{\%}\mathrm{1},\mathrm{9}\mathbin{\%}\mathrm{1},\mathrm{11}\mathbin{\%}\mathrm{1},\mathrm{16}\mathbin{\%}\mathrm{1},\mathrm{12}\mathbin{\%}\mathrm{1}\mskip1.5mu]},
which corresponds to the polynomial
$12x^4 + 16x^3 + 11x^2 + 9x + 14$.
Let us test:

\ensuremath{\Varid{mapply}\;(\Conid{P}\;[\mskip1.5mu \mathrm{14},\mathrm{9},\mathrm{11},\mathrm{16},\mathrm{12}\mskip1.5mu])\;[\mskip1.5mu \mathrm{0}\mathinner{\ldotp\ldotp}\mathrm{10}\mskip1.5mu]} yields:

14,62,396,1544,4322,9834,19472,34916,58134,91382,137204,

which indeed is the same sequence we saw above!

Now, what about the differences generated
by the difference engine? Those, too, are sequences
of numbers. Are there polynomials
that generate those sequences?
The first difference sequence of our formerly unknown polynomial is

48,334,1148,2778,5512,9638,15444,23218,33248,45822

The next three difference sequences could be derived
from this sequence -- so, we can assume that this sequence
is generated by a polynomial of degree 3. Let us see
what \ensuremath{\Varid{findGen}\;(\Varid{tail}\;\Varid{ds})\;(\Varid{head}\;\Varid{ds})} yields with \ensuremath{\Varid{ds}}
being the list of difference sequences of that polynomial:
\ensuremath{[\mskip1.5mu \mathrm{48}\mathbin{\%}\mathrm{1},\mathrm{118}\mathbin{\%}\mathrm{1},\mathrm{120}\mathbin{\%}\mathrm{1},\mathrm{48}\mathbin{\%}\mathrm{1}\mskip1.5mu]}, 
which corresponds to the polynomial 
$48x^3 + 120x^2 + 118x + 48$.
Let us test again:

\ensuremath{\Varid{mapply}\;(\Conid{P}\;[\mskip1.5mu \mathrm{48},\mathrm{118},\mathrm{120},\mathrm{48}\mskip1.5mu])\;[\mskip1.5mu \mathrm{0}\mathinner{\ldotp\ldotp}\mathrm{10}\mskip1.5mu]} yields:

48,334,1148,2778,5512,9638,15444,23218,33248,45822,61228

The next difference sequence should then be generated
by a polynomial of degree 2. We try with
\ensuremath{\mathbf{let}\;\Varid{ds'}\mathrel{=}\Varid{tail}\;\Varid{ds}\;\mathbf{in}\;\Varid{findGen}\;(\Varid{tail}\;\Varid{ds'})\;(\Varid{head}\;\Varid{ds'})} and get
\ensuremath{[\mskip1.5mu \mathrm{286}\mathbin{\%}\mathrm{1},\mathrm{384}\mathbin{\%}\mathrm{1},\mathrm{144}\mathbin{\%}\mathrm{1}\mskip1.5mu]},
which corresponds to the polynomial 
$144x^2 + 384x + 286$.

\ensuremath{\Varid{mapply}\;(\Conid{P}\;[\mskip1.5mu \mathrm{286},\mathrm{384},\mathrm{144}\mskip1.5mu])\;[\mskip1.5mu \mathrm{0}\mathinner{\ldotp\ldotp}\mathrm{10}\mskip1.5mu]} yields:

286,814,1630,2734,4126,5806,7774,10030,12574,15406,18526

which, indeed, is the third difference sequence.

Finally, the last but one sequence, the last
that is not constant, should be generated by a polynomial
of degree 1. We try with
\ensuremath{\mathbf{let}\;\Varid{ds''}\mathrel{=}\Varid{tail}\;(\Varid{tail}\;\Varid{ds})\;\mathbf{in}\;\Varid{findGen}\;(\Varid{tail}\;\Varid{ds''})\;(\Varid{head}\;\Varid{ds''})}
and get \ensuremath{\mathrm{528}\mathbin{\%}\mathrm{1},\mathrm{288}\mathbin{\%}\mathrm{1}}
which corresponds to the polynomial $288x + 528$.

\ensuremath{\Varid{mapply}\;(\Conid{P}\;[\mskip1.5mu \mathrm{528},\mathrm{288}\mskip1.5mu])\;[\mskip1.5mu \mathrm{0}\mathinner{\ldotp\ldotp}\mathrm{10}\mskip1.5mu]} yields:

528,816,1104,1392,1680,1968,2256,2544,2832,3120,3408

which, again is the expected difference sequence.

The differences are closely related to the tremendously
important concept of the \term{derivative} of a function.
The derivative of a polynomial $\pi$ of degree $n$
is a polynomial $\pi'$ of degree $n-1$ that measures
the \term{rate of change} or \term{slope} of $\pi$.
The derivative expresses the rate of change precisely
for any point in $\pi$. We will look at this with
much more attention in the next chapter; 
the third part will then be entirely dedicated 
to derivatives and
related concepts.

The difference sequences and the polynomials that generate them
are also a measure of the rate of change.
Actually, the difference between two points \emph{is}
the rate of change of that polynomial between those two points.
The difference, however, is a sloppy measure.

Without going into too much detail here,
we can quickly look at how the derivative of a polynomial
is computed, which, in fact, is very easy.
For a polynomial of the form

\[
ax^n + bx^m + \dots + cx + d,
\]

the derivative is

\[
nax^{n-1} + mbx^{m-1} + \dots + c.
\]

In other words, we drop the last term (which is the first term
in our Haskell representation of polynomials)
and, for all other terms, we multiply the term by the exponent
and reduce the exponent by one.

The derivative of the polynomial $x^4$, for instance,
is $4x^3$; in the notation of our polynomial type,
we have \ensuremath{\Conid{P}\;[\mskip1.5mu \mathrm{0},\mathrm{0},\mathrm{0},\mathrm{0},\mathrm{1}\mskip1.5mu]} and its derivative \ensuremath{\Conid{P}\;[\mskip1.5mu \mathrm{0},\mathrm{0},\mathrm{0},\mathrm{4}\mskip1.5mu]}.
The derivative of $4x^3$ is $12x^2$, whose derivative then
is $24x$, whose derivative is just $24$.
The deriviative of our polynomial

\[
12x^4 + 16x^3 + 11x^2 + 9x + 14
\]

is 

\[
48x^3 + 48x^2 + 22x + 9.
\]

Note that the first term equals the first term
of the polynomial that we identified as the generator
of the first difference sequence. Indeed,
the differences are sloppy as a measure for
the rate of change -- but they are not completely wrong!

Here is a function to compute the derivative:

\begin{minipage}{\textwidth}
\begingroup\par\noindent\advance\leftskip\mathindent\(
\begin{pboxed}\SaveRestoreHook
\column{B}{@{}>{\hspre}l<{\hspost}@{}}%
\column{3}{@{}>{\hspre}l<{\hspost}@{}}%
\column{5}{@{}>{\hspre}l<{\hspost}@{}}%
\column{12}{@{}>{\hspre}l<{\hspost}@{}}%
\column{27}{@{}>{\hspre}c<{\hspost}@{}}%
\column{27E}{@{}l@{}}%
\column{30}{@{}>{\hspre}l<{\hspost}@{}}%
\column{E}{@{}>{\hspre}l<{\hspost}@{}}%
\>[3]{}\Varid{derivative}\mathbin{::}(\Conid{Eq}\;\Varid{a},\Conid{Num}\;\Varid{a},\Conid{Enum}\;\Varid{a})\Rightarrow \Conid{Poly}\;\Varid{a}\to \Conid{Poly}\;\Varid{a}{}\<[E]%
\\
\>[3]{}\Varid{derivative}\;(\Conid{P}\;\Varid{as})\mathrel{=}\Conid{P}\;(\Varid{cleanz}\;(\Varid{go}\mathbin{\$}\Varid{zip}\;[\mskip1.5mu \mathrm{1}\mathinner{\ldotp\ldotp}\mskip1.5mu]\;(\Varid{drop}\;\mathrm{1}\;\Varid{as}))){}\<[E]%
\\
\>[3]{}\hsindent{2}{}\<[5]%
\>[5]{}\mathbf{where}\;{}\<[12]%
\>[12]{}\Varid{go}\;[\mskip1.5mu \mskip1.5mu]{}\<[27]%
\>[27]{}\mathrel{=}{}\<[27E]%
\>[30]{}[\mskip1.5mu \mskip1.5mu]{}\<[E]%
\\
\>[12]{}\Varid{go}\;((\Varid{x},\Varid{c})\mathbin{:}\Varid{cs}){}\<[27]%
\>[27]{}\mathrel{=}{}\<[27E]%
\>[30]{}(\Varid{x}\mathbin{*}\Varid{c})\mathbin{:}\Varid{go}\;\Varid{cs}{}\<[E]%
\ColumnHook
\end{pboxed}
\)\par\noindent\endgroup\resethooks
\end{minipage}
\ignore{$}

What is the sequence generated by the derivative of our polynomial?
Well, we define the derivative as
\ensuremath{\mathbf{let}\;\Varid{p'}\mathrel{=}\Varid{derivative}\;(\Conid{P}\;[\mskip1.5mu \mathrm{14},\mathrm{9},\mathrm{11},\mathrm{16},\mathrm{12}\mskip1.5mu])}, which is \ensuremath{\Conid{P}\;[\mskip1.5mu \mathrm{9},\mathrm{22},\mathrm{48},\mathrm{48}\mskip1.5mu]},
apply it using \ensuremath{\Varid{mapply}\;\Varid{p'}\;[\mskip1.5mu \mathrm{0}\mathinner{\ldotp\ldotp}\mathrm{10}\mskip1.5mu]} and see:

9,127,629,1803,3937,7319,12237,18979,27833,39087,53029

Quite different from the first difference sequence we saw above!

What about the second derivative? we define as
\ensuremath{\mathbf{let}\;\Varid{p''}\mathrel{=}\Varid{derivative}\;\Varid{p'}} and get \ensuremath{\Conid{P}\;[\mskip1.5mu \mathrm{22},\mathrm{96},\mathrm{144}\mskip1.5mu]}.
This polynomial creates the sequence

22,262,790,1606,2710,4102,5782,7750,10006,12550,15382

The next derivative, \ensuremath{\mathbf{let}\;\Varid{p'''}\mathrel{=}\Varid{derivative}\;\Varid{p''}},
is \ensuremath{\Conid{P}\;[\mskip1.5mu \mathrm{96},\mathrm{288}\mskip1.5mu]} and generates the sequence

96,384,672,960,1248,1536,1824,2112,2400,2688,2976.

You can already predict the next derivative,
which is a polynomial of degree 0: it is \ensuremath{\Conid{P}\;[\mskip1.5mu \mathrm{288}\mskip1.5mu]}.
This is a constant polynomial and will generate a constant
sequence, namely the sequence 288. That, however,
was also the constant sequence generated by the
difference engine. Of course, when the rate of change
is the same everywhere in the original polynomial,
then precision does not make any difference anymore.
The two methods shall come to the same result.

Consider the simple polynomial $x^2$.
It generates the sequence

\[
0,1,4,9,16,25,36,49,\dots
\]

The differences are

\[
1,3,5,7,9,11,13,\dots
\]

The differences of this list are all 2.

The derivative of $x^2$ is $2x$.
It would generate the sequence

\[
0,2,4,6,8,10,12,14,\dots
\]

which does not equal the differences.
However, we can already see that the derivative
of $2x$, $2$, is constant and generates the constant
sequence 

\[
2,2,2,2,2,2,2,2,\dots
\]
\section{Differences and Binomial Coefficients}
\ignore{
\begingroup\par\noindent\advance\leftskip\mathindent\(
\begin{pboxed}\SaveRestoreHook
\column{B}{@{}>{\hspre}l<{\hspost}@{}}%
\column{3}{@{}>{\hspre}l<{\hspost}@{}}%
\column{E}{@{}>{\hspre}l<{\hspost}@{}}%
\>[B]{}\mathbf{module}\;\Conid{DiffBinom}{}\<[E]%
\\
\>[B]{}\mathbf{where}{}\<[E]%
\\
\>[B]{}\hsindent{3}{}\<[3]%
\>[3]{}\mathbf{import}\;\Conid{Natural}{}\<[E]%
\\
\>[B]{}\hsindent{3}{}\<[3]%
\>[3]{}\mathbf{import}\;\Conid{Zahl}{}\<[E]%
\\
\>[B]{}\hsindent{3}{}\<[3]%
\>[3]{}\mathbf{import}\;\Conid{Quoz}{}\<[E]%
\\
\>[B]{}\hsindent{3}{}\<[3]%
\>[3]{}\mathbf{import}\;\Conid{Real}{}\<[E]%
\\
\>[B]{}\hsindent{3}{}\<[3]%
\>[3]{}\mathbf{import}\;\Conid{NumSystem}{}\<[E]%
\\
\>[B]{}\hsindent{3}{}\<[3]%
\>[3]{}\mathbf{import}\;\Conid{PolyArith}{}\<[E]%
\\
\>[B]{}\hsindent{3}{}\<[3]%
\>[3]{}\mathbf{import}\;\Conid{DMachine}{}\<[E]%
\ColumnHook
\end{pboxed}
\)\par\noindent\endgroup\resethooks
}

The ingenious Isaac Newton studied the relation
between sequences and their differences 
intensely and came up with a formula. Before we go right
to it, let us observe on our own.
The following table shows, in the first line, 
the value of $n$, \ie\ the value to which the polynomial
is applied; in the second line, we see the result
for this $n$; in the first column we have the first
value from the sequence and from the difference lists:

\begin{center}
\begingroup
\renewcommand{\arraystretch}{1.5}
\begin{tabular}{||c||c|c|c|c|c||}
\hline
     &  0 &  1  &  2  &  3   &  4   \\\hline
     & 14 & 62  & 396 & 1544 & 4322 \\\hline\hline
  14 &  1 &  1  &  1  &  1   &  1   \\\hline
  48 &  0 &  1  &  2  &  3   &  4   \\\hline
 286 &  0 &  0  &  1  &  3   &  6   \\\hline
 528 &  0 &  0  &  0  &  1   &  4   \\\hline
 288 &  0 &  0  &  0  &  0   &  1   \\\hline
\end{tabular}
\endgroup
\end{center}

What we see in the cells of the table
are factors. With their help, we can compute
the values in the sequence by formulas of the type:

\begin{equation}
\begin{array}{rcrcrcrcrcrcrcrcrcrcr}
  1 & \times & 14 &   &   &        &    &   &   &        &     &   &   &        &     &   &   &        &     & = & 14\\
  1 & \times & 14 & + & 1 & \times & 48 &   &   &        &     &   &   &        &     &   &   &        &     & = & 62\\
  1 & \times & 14 & + & 2 & \times & 48 & + & 1 & \times & 286 &   &   &        &     &   &   &        &     & = & 396\\
  1 & \times & 14 & + & 3 & \times & 48 & + & 3 & \times & 286 & + & 1 & \times & 528 &   &   &        &     & = & 1544\\
  1 & \times & 14 & + & 4 & \times & 48 & + & 6 & \times & 286 & + & 4 & \times & 528 & + & 1 & \times & 288 & = & 4322
\end{array}
\end{equation}

The next question would then be: what are those numbers?
But, here, I have to ask you to look a bit more closely at the table.
What we see is left-to-right:

\begin{center}
\begin{tabular}{cccccccccc}
  &   &     &    &     & 1 &      &   &   &   \\
  &   &     &    &  1  &   &  1   &   &   &   \\
  &   &     &  1 &     & 2 &      & 1 &   &   \\
  &   &  1  &    &  3  &   &  3   &   & 1 &    \\
  & 1 &     & 4  &     & 6 &      & 4 &   & 1 
\end{tabular}
\end{center}

Those are binomial coefficients!
Indeed. We could rewrite the table as

\begin{center}
\begingroup
\renewcommand{\arraystretch}{1.5}
\begin{tabular}{||c||c|c|c|c|c||}
\hline
     &  0 &  1  &  2  &  3   &  4   \\\hline
     & 14         & 62         & 396        & 1544       & 4322       \\\hline\hline
  14 &$\binom{0}{0}$&$\binom{1}{0}$&$\binom{2}{0}$&$\binom{3}{0}$&$\binom{4}{0}$\\\hline
  48 &$\binom{0}{1}$&$\binom{1}{1}$&$\binom{2}{1}$&$\binom{3}{1}$&$\binom{4}{1}$\\\hline
 286 &$\binom{0}{2}$&$\binom{1}{2}$&$\binom{2}{2}$&$\binom{3}{2}$&$\binom{4}{2}$\\\hline
 528 &$\binom{0}{3}$&$\binom{1}{3}$&$\binom{2}{3}$&$\binom{3}{3}$&$\binom{4}{3}$\\\hline
 288 &$\binom{0}{3}$&$\binom{1}{4}$&$\binom{2}{4}$&$\binom{3}{4}$&$\binom{4}{4}$\\\hline
\end{tabular}
\endgroup
\end{center}

If this were universally true, we could devise a 
much better prediction function. The one we wrote
in the previous section has the disadvantage
that we can only predict the next number in the sequence.
To predict a value way ahead we need to generate
number by number before we are there.
With Newton's trick, we could compute any number
in the sequence in one step.

All we have to do is to get the \ensuremath{\Varid{head}}s of the sequences
and to calculate the formula:

\[
\sum_{k=0}^{d}{h_k\binom{n}{k}} 
\]

where $d$ is the degree of the polynomial, $n$
the position in the sequence, \ie\ the number
to which we apply the polynomial, and $h_k$
the head of the sequence starting to count
with the original sequence as $k=0$.
The sixth value ($n=5$) of the sequence would then be

\[
  14 \times \binom{5}{0} + 
  48 \times \binom{5}{1} + 
 286 \times \binom{5}{2} + 
 528 \times \binom{5}{3} + 
 288 \times \binom{5}{4}, 
\]

which is

\[
  14           + 
  48 \times  5 + 
 286 \times 10 + 
 528 \times 10 + 
 288 \times  5, 
\]

which, in its turn, is

\[
14 + 240 + 2860 + 5280 + 1440 = 9834,
\]

which is indeed the next value in the sequence.

Here is an implementation:

\begin{minipage}{\textwidth}
\begingroup\par\noindent\advance\leftskip\mathindent\(
\begin{pboxed}\SaveRestoreHook
\column{B}{@{}>{\hspre}l<{\hspost}@{}}%
\column{3}{@{}>{\hspre}l<{\hspost}@{}}%
\column{5}{@{}>{\hspre}l<{\hspost}@{}}%
\column{12}{@{}>{\hspre}l<{\hspost}@{}}%
\column{16}{@{}>{\hspre}c<{\hspost}@{}}%
\column{16E}{@{}l@{}}%
\column{19}{@{}>{\hspre}l<{\hspost}@{}}%
\column{E}{@{}>{\hspre}l<{\hspost}@{}}%
\>[3]{}\Varid{newton}\mathbin{::}\Conid{Zahl}\to [\mskip1.5mu [\mskip1.5mu \Conid{Zahl}\mskip1.5mu]\mskip1.5mu]\to [\mskip1.5mu \Conid{Zahl}\mskip1.5mu]\to \Conid{Zahl}{}\<[E]%
\\
\>[3]{}\Varid{newton}\;\Varid{n}\;\Varid{ds}\;\Varid{seq}\mathrel{=}\Varid{sum}\;\Varid{ts}{}\<[E]%
\\
\>[3]{}\hsindent{2}{}\<[5]%
\>[5]{}\mathbf{where}\;{}\<[12]%
\>[12]{}\Varid{hs}{}\<[16]%
\>[16]{}\mathrel{=}{}\<[16E]%
\>[19]{}\Varid{getHeads}\;\Varid{seq}\;\Varid{ds}{}\<[E]%
\\
\>[12]{}\Varid{ts}{}\<[16]%
\>[16]{}\mathrel{=}{}\<[16E]%
\>[19]{}[\mskip1.5mu \Varid{h}\mathbin{*}(\Varid{choose}\;\Varid{n}\;\Varid{k})\mid (\Varid{h},\Varid{k})\leftarrow \Varid{zip}\;\Varid{hs}\;[\mskip1.5mu \mathrm{0}\mathinner{\ldotp\ldotp}\Varid{n}\mskip1.5mu]\mskip1.5mu]{}\<[E]%
\\[\blanklineskip]%
\>[3]{}\Varid{getHeads}\mathbin{::}[\mskip1.5mu \Conid{Zahl}\mskip1.5mu]\to [\mskip1.5mu [\mskip1.5mu \Conid{Zahl}\mskip1.5mu]\mskip1.5mu]\to [\mskip1.5mu \Conid{Zahl}\mskip1.5mu]{}\<[E]%
\\
\>[3]{}\Varid{getHeads}\;\Varid{seq}\;\Varid{ds}\mathrel{=}\Varid{map}\;\Varid{head}\;(\Varid{seq}\mathbin{:}\Varid{ds}){}\<[E]%
\ColumnHook
\end{pboxed}
\)\par\noindent\endgroup\resethooks
\end{minipage}

To perform some experiments, here, as a reminder,
are the first 14 numbers of the sequence generated
by our polynomial \ensuremath{\Conid{P}\;[\mskip1.5mu \mathrm{14},\mathrm{9},\mathrm{11},\mathrm{16},\mathrm{12}\mskip1.5mu]}:

14,62,396,1544,4322,9834,19472,34916,58134,91382,137204,198432,278186,379874

We set \ensuremath{\Varid{s}\mathrel{=}\Varid{mapply}\;\Conid{P}\;[\mskip1.5mu \mathrm{14},\mathrm{9},\mathrm{11},\mathrm{16},\mathrm{12}\mskip1.5mu]\;[\mskip1.5mu \mathrm{0}\mathinner{\ldotp\ldotp}\mathrm{10}\mskip1.5mu]} and \ensuremath{\Varid{d}\mathrel{=}\Varid{dengine}\;\Varid{s}}.
Now we perform some tests:

\begin{minipage}{\textwidth}
\ensuremath{\Varid{newton}\;\mathrm{0}\;\Varid{d}\;\Varid{s}} gives      14.\\
\ensuremath{\Varid{newton}\;\mathrm{1}\;\Varid{d}\;\Varid{s}} gives      62.\\
\ensuremath{\Varid{newton}\;\mathrm{5}\;\Varid{d}\;\Varid{s}} gives    9834.\\
\ensuremath{\Varid{newton}\;\mathrm{11}\;\Varid{d}\;\Varid{s}} gives  198432.\\
\ensuremath{\Varid{newton}\;\mathrm{13}\;\Varid{d}\;\Varid{s}} gives  379874.
\end{minipage}

The approach, hence, seems to work.
But there is more.
The function \ensuremath{\Varid{newton}} gives us a closed form 
to compute any number in the sequence,
given that we have the beginning of that sequence
and its difference lists.
A closed form, however, is a generating formula --
it is the polynomial that generates the entire sequence.
We just need a way to make the formula implicit in
\ensuremath{\Varid{newton}} explicit.

We can do that using our polynomial data type.
When we can express the binomial coefficients
in terms of polynomials and apply them
to the formula used above, we will get the polynomial out
that generates this sequence.
Here is a function that does it:

\begin{minipage}{\textwidth}
\begingroup\par\noindent\advance\leftskip\mathindent\(
\begin{pboxed}\SaveRestoreHook
\column{B}{@{}>{\hspre}l<{\hspost}@{}}%
\column{3}{@{}>{\hspre}l<{\hspost}@{}}%
\column{5}{@{}>{\hspre}l<{\hspost}@{}}%
\column{12}{@{}>{\hspre}l<{\hspost}@{}}%
\column{18}{@{}>{\hspre}c<{\hspost}@{}}%
\column{18E}{@{}l@{}}%
\column{21}{@{}>{\hspre}l<{\hspost}@{}}%
\column{E}{@{}>{\hspre}l<{\hspost}@{}}%
\>[3]{}\Varid{bin2poly}\mathbin{::}\Conid{Zahl}\to \Conid{Zahl}\to \Conid{Poly}\;\Conid{Quoz}{}\<[E]%
\\
\>[3]{}\Varid{bin2poly}\;\Varid{h}\;\mathrm{0}{}\<[18]%
\>[18]{}\mathrel{=}{}\<[18E]%
\>[21]{}\Conid{P}\;[\mskip1.5mu \Varid{h}\mathbin{\%}\mathrm{1}\mskip1.5mu]{}\<[E]%
\\
\>[3]{}\Varid{bin2poly}\;\Varid{h}\;\mathrm{1}{}\<[18]%
\>[18]{}\mathrel{=}{}\<[18E]%
\>[21]{}\Conid{P}\;[\mskip1.5mu \mathrm{0},\Varid{h}\mathbin{\%}\mathrm{1}\mskip1.5mu]{}\<[E]%
\\
\>[3]{}\Varid{bin2poly}\;\Varid{h}\;\Varid{k}{}\<[18]%
\>[18]{}\mathrel{=}{}\<[18E]%
\>[21]{}\Conid{P}\;[\mskip1.5mu \Varid{h}\mathbin{\%}(\Varid{\Conid{B}.fac}\;\Varid{k})\mskip1.5mu]\mathbin{`\Varid{mul}`}\Varid{go}\;(\Varid{k}\mathbin{\%}\mathrm{1}){}\<[E]%
\\
\>[3]{}\hsindent{2}{}\<[5]%
\>[5]{}\mathbf{where}\;{}\<[12]%
\>[12]{}\Varid{go}\;\mathrm{1}{}\<[18]%
\>[18]{}\mathrel{=}{}\<[18E]%
\>[21]{}\Conid{P}\;[\mskip1.5mu \mathrm{0},\mathrm{1}\mskip1.5mu]{}\<[E]%
\\
\>[12]{}\Varid{go}\;\Varid{i}{}\<[18]%
\>[18]{}\mathrel{=}{}\<[18E]%
\>[21]{}\Conid{P}\;[\mskip1.5mu \mathbin{-}(\Varid{i}\mathbin{-}\mathrm{1}),\mathrm{1}\mskip1.5mu]\mathbin{`\Varid{mul}`}(\Varid{go}\;(\Varid{i}\mathbin{-}\mathrm{1})){}\<[E]%
\ColumnHook
\end{pboxed}
\)\par\noindent\endgroup\resethooks
\end{minipage}

The function receives two integers:
the first one is a factor (the head) 
by which we multiply the resulting binomial polynomial
and the second one is $k$ in $\binom{n}{k}$.
Note that we do not need $n$, since $n$ is the unknown,
the base of our polynomial.

If $k=0$, the binomial is 1, since for all binomials:
$\binom{n}{0} = 1$. We, hence, return a constant polynomial
consisting of the factor. This corresponds to 
$h_0 \times \binom{n}{0}$. The result is just $h$.
Note that we convert the coefficients to rational numbers,
since that is the type the function is supposed to yield.

If $k=1$, the binomial is $n$, since for all binomials:
$\binom{n}{1} = n$. Because $n$ is the base of the polynomial,
$n$ itself is expressed by \ensuremath{\Conid{P}\;[\mskip1.5mu \mathrm{0},\mathrm{1}\mskip1.5mu]}. 
This is just $n+0$ and, hence, $n$.
Since we multiply with $h$, the result in this case is
$h \times n = hn$, or, in the language of our Haskell
polynomials \ensuremath{\Conid{P}\;[\mskip1.5mu \mathrm{0},\Varid{h}\mskip1.5mu]}.

Otherwise, we go into the recursive \ensuremath{\Varid{go}} function.
The function receives one rational number, namely $k$
(which, de facto, is an integer)
The base case is $k=1$. In that case we yield \ensuremath{\Conid{P}\;[\mskip1.5mu \mathrm{0},\mathrm{1}\mskip1.5mu]},
which is just $n$.
Otherwise, we create the polynomial
\ensuremath{\Conid{P}\;[\mskip1.5mu \mathbin{-}(\Varid{i}\mathbin{-}\mathrm{1}),\mathrm{1}\mskip1.5mu]}, that is $n-(k-1)$ and multiply
with the result of \ensuremath{\Varid{go}} applied to $i-1$.
The function, hence, creates the numerator
of the fraction formula of the binomial coefficient:

\[
n(n-1)(n-2)\dots (n-k+1).
\]

The result of the function is then multiplied by
$h$ divided by $k!$. The former, still, is some head
from the difference sequences and
the latter is the denominator
of the fraction formula. We, thus, compute:

\[
\frac{hn(n-1)(n-2)\dots (n-k+1)}{k!}.
\]

Now, we can use this formula represented by a 
polynomial to compute the generating polynomial.
The function that does so has exactly the same
structure as the \ensuremath{\Varid{newton}} function. The difference
is just that it expresses binomial coefficients
as polynomials and that it does not receive 
a concrete number $n$ for which we want to compute
the corresponding value (because we want to compute
the formula generating all the values):

\begin{minipage}{\textwidth}
\begingroup\par\noindent\advance\leftskip\mathindent\(
\begin{pboxed}\SaveRestoreHook
\column{B}{@{}>{\hspre}l<{\hspost}@{}}%
\column{3}{@{}>{\hspre}l<{\hspost}@{}}%
\column{5}{@{}>{\hspre}l<{\hspost}@{}}%
\column{12}{@{}>{\hspre}l<{\hspost}@{}}%
\column{16}{@{}>{\hspre}c<{\hspost}@{}}%
\column{16E}{@{}l@{}}%
\column{19}{@{}>{\hspre}l<{\hspost}@{}}%
\column{E}{@{}>{\hspre}l<{\hspost}@{}}%
\>[3]{}\Varid{newtonGen}\mathbin{::}[\mskip1.5mu [\mskip1.5mu \Conid{Zahl}\mskip1.5mu]\mskip1.5mu]\to [\mskip1.5mu \Conid{Zahl}\mskip1.5mu]\to \Conid{Poly}\;[\mskip1.5mu \Conid{Quoz}\mskip1.5mu]{}\<[E]%
\\
\>[3]{}\Varid{newtonGen}\;\Varid{ds}\;\Varid{seq}\mathrel{=}\Varid{sump}\;\Varid{ts}{}\<[E]%
\\
\>[3]{}\hsindent{2}{}\<[5]%
\>[5]{}\mathbf{where}\;{}\<[12]%
\>[12]{}\Varid{hs}{}\<[16]%
\>[16]{}\mathrel{=}{}\<[16E]%
\>[19]{}\Varid{getHeads}\;\Varid{seq}\;\Varid{ds}{}\<[E]%
\\
\>[12]{}\Varid{ts}{}\<[16]%
\>[16]{}\mathrel{=}{}\<[16E]%
\>[19]{}[\mskip1.5mu \Varid{bin2poly}\;\Varid{h}\;\Varid{k}\mid (\Varid{h},\Varid{k})\leftarrow \Varid{zip}\;\Varid{hs}\;[\mskip1.5mu \mathrm{0}\mathinner{\ldotp\ldotp}\Varid{n}\mskip1.5mu]\mskip1.5mu]{}\<[E]%
\\
\>[12]{}\Varid{n}{}\<[16]%
\>[16]{}\mathrel{=}{}\<[16E]%
\>[19]{}\Varid{fromIntegral}\;(\Varid{length}\mathbin{\$}\Varid{ds}){}\<[E]%
\ColumnHook
\end{pboxed}
\)\par\noindent\endgroup\resethooks
\end{minipage}

When we call \ensuremath{\Varid{newtonGen}\;\Varid{ds}\;\Varid{s}}, $ds$ 
still being the difference lists and
$s$ the sequence in question, we see:

\ensuremath{\Conid{P}\;[\mskip1.5mu \mathrm{14}\mathbin{\%}\mathrm{1},\mathrm{9}\mathbin{\%}\mathrm{1},\mathrm{11}\mathbin{\%}\mathrm{1},\mathrm{16}\mathbin{\%}\mathrm{1},\mathrm{12}\mathbin{\%}\mathrm{1}\mskip1.5mu]},

which we immediately recognise as our polynomial
$12x^4 + 16x^3 + 11x^2 + 9x + 14$.

For another test, we apply the monomial $x^5$ as

\ensuremath{\mathbf{let}\;\Varid{s}\mathrel{=}\Varid{mapply}\;(\Conid{P}\;[\mskip1.5mu \mathrm{0},\mathrm{0},\mathrm{0},\mathrm{0},\mathrm{0},\mathrm{1}\mskip1.5mu])\;[\mskip1.5mu \mathrm{0}\mathinner{\ldotp\ldotp}\mathrm{10}\mskip1.5mu]\;\mathbf{in}\;\Varid{newtonGen}\;(\Varid{dengine}\;\Varid{s})\;\Varid{s}}

and see

\ensuremath{\Conid{P}\;[\mskip1.5mu \mathrm{0}\mathbin{\%}\mathrm{1},\mathrm{0}\mathbin{\%}\mathrm{1},\mathrm{0}\mathbin{\%}\mathrm{1},\mathrm{0}\mathbin{\%}\mathrm{1},\mathrm{0}\mathbin{\%}\mathrm{1},\mathrm{1}\mathbin{\%}\mathrm{1}\mskip1.5mu]},

which is indeed the polynomial $x^5$.

But now comes the hard question:
why does that work at all???

To answer this question, we should make sure to understand
how Newton's formula works. The point is that
we restrict ourselves to the heads of the sequences as basic
building blocks. When we compute some value $x_n$ in the sequence,
we need to recursively compute $x_{n-1}$ and the difference between
$x_{n-1}$ and $x_{n}$ and add them together.
Let us build a model that simulates this approach
and that allows us to reason about 
what is going on more easily.

We use as a model a polynomial of degree 3;
that model is sufficiently complex to simulate the problem
completely and is, on the other hand, somewhat simpler
than a model based on a polynomial of degree 4,
like the one we have studied above.

The model consists of a data type:

\begin{minipage}{\textwidth}
\begingroup\par\noindent\advance\leftskip\mathindent\(
\begin{pboxed}\SaveRestoreHook
\column{B}{@{}>{\hspre}l<{\hspost}@{}}%
\column{3}{@{}>{\hspre}l<{\hspost}@{}}%
\column{5}{@{}>{\hspre}l<{\hspost}@{}}%
\column{E}{@{}>{\hspre}l<{\hspost}@{}}%
\>[3]{}\mathbf{data}\;\Conid{Newton}\mathrel{=}\Conid{H}\mid \Conid{X}\mid \Conid{Y}\mid \Conid{Z}{}\<[E]%
\\
\>[3]{}\hsindent{2}{}\<[5]%
\>[5]{}\mathbf{deriving}\;(\Conid{Show},\Conid{Eq}){}\<[E]%
\ColumnHook
\end{pboxed}
\)\par\noindent\endgroup\resethooks
\end{minipage}

The \ensuremath{\Conid{Newton}} type has four constructors:
\ensuremath{\Conid{H}} represents the head of the original sequence;
\ensuremath{\Conid{X}} is the head of the first difference list;
\ensuremath{\Conid{Y}} is the head of the second difference list and
\ensuremath{\Conid{Z}} is the constant element repeated in the last difference list.
(Remember that a polynomial of degree 3 
generates 3 difference lists.)

The model also contains a function
to compute positions in the sequence.
This function, called \ensuremath{\Varid{cn}} (for ``computeNewton''),
takes two arguments: a \ensuremath{\Conid{Newton}} constructor and an integer.
The integer tells us the position we want to compute
starting with the head $H = 0$:

\begin{minipage}{\textwidth}
\begingroup\par\noindent\advance\leftskip\mathindent\(
\begin{pboxed}\SaveRestoreHook
\column{B}{@{}>{\hspre}l<{\hspost}@{}}%
\column{3}{@{}>{\hspre}l<{\hspost}@{}}%
\column{E}{@{}>{\hspre}l<{\hspost}@{}}%
\>[3]{}\Varid{cn}\mathbin{::}\Conid{Newton}\to \Conid{Natural}\to [\mskip1.5mu \Conid{Newton}\mskip1.5mu]{}\<[E]%
\\
\>[3]{}\Varid{cn}\;\Conid{H}\;\mathrm{0}\mathrel{=}[\mskip1.5mu \Conid{H}\mskip1.5mu]{}\<[E]%
\\
\>[3]{}\Varid{cn}\;\Conid{H}\;\Varid{n}\mathrel{=}\Varid{cn}\;\Conid{H}\;(\Varid{n}\mathbin{-}\mathrm{1})\plus \Varid{cn}\;\Conid{X}\;(\Varid{n}\mathbin{-}\mathrm{1}){}\<[E]%
\ColumnHook
\end{pboxed}
\)\par\noindent\endgroup\resethooks
\end{minipage}

When we want to compute the first element in the sequence,
\ensuremath{\Varid{cn}\;\Conid{H}\;\mathrm{0}}, we just return \ensuremath{[\mskip1.5mu \Conid{H}\mskip1.5mu]}. When we want to compute
any other number, we recursively call \ensuremath{\Varid{cn}\;\Conid{H}\;(\Varid{n}\mathbin{-}\mathrm{1})},
which computes the previous data point, and add \ensuremath{\Varid{cn}\;\Conid{X}\;(\Varid{n}\mathbin{-}\mathrm{1})},
which computes the difference between $n$ and $n-1$.
Here is how we compute the difference:

\begin{minipage}{\textwidth}
\begingroup\par\noindent\advance\leftskip\mathindent\(
\begin{pboxed}\SaveRestoreHook
\column{B}{@{}>{\hspre}l<{\hspost}@{}}%
\column{3}{@{}>{\hspre}l<{\hspost}@{}}%
\column{E}{@{}>{\hspre}l<{\hspost}@{}}%
\>[3]{}\Varid{cn}\;\Conid{X}\;\mathrm{0}\mathrel{=}[\mskip1.5mu \Conid{X}\mskip1.5mu]{}\<[E]%
\\
\>[3]{}\Varid{cn}\;\Conid{X}\;\Varid{n}\mathrel{=}\Varid{cn}\;\Conid{X}\;(\Varid{n}\mathbin{-}\mathrm{1})\plus \Varid{cn}\;\Conid{Y}\;(\Varid{n}\mathbin{-}\mathrm{1}){}\<[E]%
\ColumnHook
\end{pboxed}
\)\par\noindent\endgroup\resethooks
\end{minipage}

If we need the first difference, \ensuremath{\Varid{cn}\;\Conid{X}\;\mathrm{0}}, we just return
\ensuremath{[\mskip1.5mu \Conid{X}\mskip1.5mu]}. Otherwise, we call \ensuremath{\Varid{cn}\;\Conid{X}\;(\Varid{n}\mathbin{-}\mathrm{1})}, this computes
the previous difference, and compute \ensuremath{\Varid{cn}\;\Conid{Y}\;(\Varid{n}\mathbin{-}\mathrm{1})},
the difference between the previous and the current difference.
Here is how we compute the difference of the difference:

\begin{minipage}{\textwidth}
\begingroup\par\noindent\advance\leftskip\mathindent\(
\begin{pboxed}\SaveRestoreHook
\column{B}{@{}>{\hspre}l<{\hspost}@{}}%
\column{3}{@{}>{\hspre}l<{\hspost}@{}}%
\column{E}{@{}>{\hspre}l<{\hspost}@{}}%
\>[3]{}\Varid{cn}\;\Conid{Y}\;\mathrm{0}\mathrel{=}[\mskip1.5mu \Conid{Y}\mskip1.5mu]{}\<[E]%
\\
\>[3]{}\Varid{cn}\;\Conid{Y}\;\Varid{n}\mathrel{=}\Conid{Z}\mathbin{:}\Varid{cn}\;\Conid{Y}\;(\Varid{n}\mathbin{-}\mathrm{1}){}\<[E]%
\ColumnHook
\end{pboxed}
\)\par\noindent\endgroup\resethooks
\end{minipage}

If we need the first difference, \ensuremath{\Varid{cn}\;\Conid{Y}\;\mathrm{0}}, we just return
\ensuremath{[\mskip1.5mu \Conid{Y}\mskip1.5mu]}. Otherwise, we compute the previous difference \ensuremath{\Varid{cn}\;\Conid{Y}\;(\Varid{n}\mathbin{-}\mathrm{1})}
adding \ensuremath{\Conid{Z}}, the constant difference, to the result.

The simplest case is of course 
computing the first in the sequence.
This is just:

\ensuremath{\Varid{cn}\;\Conid{H}\;\mathrm{0}}, which yields \ensuremath{[\mskip1.5mu \Conid{H}\mskip1.5mu]}.

Computing the second in the sequence is slightly more work:

\ensuremath{\Varid{cn}\;\Conid{H}\;\mathrm{1}} goes to\\
\ensuremath{\Varid{cn}\;\Conid{H}\;\mathrm{0}\plus \Varid{cn}\;\Conid{X}\;\mathrm{0}} which is\\
\ensuremath{[\mskip1.5mu \Conid{H}\mskip1.5mu]\plus [\mskip1.5mu \Conid{X}\mskip1.5mu]}.

We, hence, get \ensuremath{[\mskip1.5mu \Conid{H},\Conid{X}\mskip1.5mu]}. That is the head of the sequence
plus the head of the first difference list.

Computing the third in the sequence

\ensuremath{\Varid{cn}\;\Conid{H}\;\mathrm{2}} calls\\
\ensuremath{\Varid{cn}\;\Conid{H}\;\mathrm{1}\plus \Varid{cn}\;\Conid{X}\;\mathrm{1}}, which is\\
\ensuremath{\Varid{cn}\;\Conid{H}\;\mathrm{0}\plus \Varid{cn}\;\Conid{X}\;\mathrm{0}} and \ensuremath{\Varid{cn}\;\Conid{X}\;\mathrm{0}\plus \Varid{cn}\;\Conid{Y}\;\mathrm{0}}.

We hence get \ensuremath{[\mskip1.5mu \Conid{H},\Conid{X},\Conid{X},\Conid{Y}\mskip1.5mu]}.
This is the head of the original sequence
plus the head of the first difference sequence
(we are now at \ensuremath{\Conid{H}\;\mathrm{1}})
plus this difference plus the first of
the second difference sequence.

This looks simple, but already after a few steps,
the result looks weird. For \ensuremath{\Varid{cn}\;\Conid{H}\;\mathrm{5}}, for example, we see

\ensuremath{[\mskip1.5mu \Conid{H},\Conid{X},\Conid{X},\Conid{Y},\Conid{X},\Conid{Y},\Conid{Z},\Conid{Y},\Conid{X},\Conid{Y},\Conid{Z},\Conid{Y},\Conid{Z},\Conid{Z},\Conid{Y},\Conid{X},\Conid{Y},\Conid{Z},\Conid{Y},\Conid{Z},\Conid{Z},\Conid{Y},\Conid{Z},\Conid{Z},\Conid{Z},\Conid{Y}\mskip1.5mu]},

which is somewhat confusing. The result, however,
is correct. When we generate a random polynomial of degree 3,
say, \ensuremath{\Conid{P}\;[\mskip1.5mu \mathrm{2},\mathrm{28},\mathrm{15},\mathrm{22}\mskip1.5mu]}, this is the polynomial
$22x^3 + 15x^2 + 28x + 2$, we get the sequence
2, 67, 294, 815, 1762, 3267, 5462, 8479, 12450, 17507, 23782.
We now define a function that substitutes the symbols
of our model by the heads of the sequence and the
difference lists:

\begin{minipage}{\textwidth}
\begingroup\par\noindent\advance\leftskip\mathindent\(
\begin{pboxed}\SaveRestoreHook
\column{B}{@{}>{\hspre}l<{\hspost}@{}}%
\column{3}{@{}>{\hspre}l<{\hspost}@{}}%
\column{26}{@{}>{\hspre}l<{\hspost}@{}}%
\column{E}{@{}>{\hspre}l<{\hspost}@{}}%
\>[3]{}\Varid{new2a}\mathbin{::}(\Varid{a},\Varid{a},\Varid{a},\Varid{a})\to \Conid{Newton}\to \Varid{a}{}\<[E]%
\\
\>[3]{}\Varid{new2a}\;(\Varid{h},\Varid{x},\Varid{y},\Varid{z})\;\Varid{n}\mathrel{=}\mathbf{case}\;\Varid{n}\;\mathbf{of}{}\<[E]%
\\
\>[3]{}\hsindent{23}{}\<[26]%
\>[26]{}\Conid{H}\to \Varid{h}{}\<[E]%
\\
\>[3]{}\hsindent{23}{}\<[26]%
\>[26]{}\Conid{X}\to \Varid{x}{}\<[E]%
\\
\>[3]{}\hsindent{23}{}\<[26]%
\>[26]{}\Conid{Y}\to \Varid{y}{}\<[E]%
\\
\>[3]{}\hsindent{23}{}\<[26]%
\>[26]{}\Conid{Z}\to \Varid{z}{}\<[E]%
\\[\blanklineskip]%
\>[3]{}\Varid{subst}\mathbin{::}(\Varid{a},\Varid{a},\Varid{a},\Varid{a})\to [\mskip1.5mu \Conid{Newton}\mskip1.5mu]\to [\mskip1.5mu \Varid{a}\mskip1.5mu]{}\<[E]%
\\
\>[3]{}\Varid{subst}\;\Varid{as}\mathrel{=}\Varid{map}\;(\Varid{new2a}\;\Varid{as}){}\<[E]%
\ColumnHook
\end{pboxed}
\)\par\noindent\endgroup\resethooks
\end{minipage}

The head of the sequence is 2; the head of the difference
sequences are 65, 162 and 132.
We call the function as \ensuremath{\Varid{subst}\;(\mathrm{2},\mathrm{65},\mathrm{162},\mathrm{132})\;(\Varid{cn}\;\Conid{H}\;\mathrm{5})}
and see

2,65,65,162,65,162,132,162,65,162,132,162,132,132,162,\\
  65,162,132,162,132,132,162,132,132,132,162.

When we sum this together,
\ensuremath{\Varid{sum}\;(\Varid{subst}\;(\mathrm{2},\mathrm{65},\mathrm{162},\mathrm{132})\;(\Varid{cn}\;\Conid{H}\;\mathrm{5}))},
we get 3267, which is indeed the number appearing at
position 5 in the sequence (starting to count with 0).

We implement one more function: \ensuremath{\Varid{ccn}}, for
``count cn'':

\begin{minipage}{\textwidth}
\begingroup\par\noindent\advance\leftskip\mathindent\(
\begin{pboxed}\SaveRestoreHook
\column{B}{@{}>{\hspre}l<{\hspost}@{}}%
\column{3}{@{}>{\hspre}l<{\hspost}@{}}%
\column{15}{@{}>{\hspre}l<{\hspost}@{}}%
\column{E}{@{}>{\hspre}l<{\hspost}@{}}%
\>[3]{}\Varid{ccn}\mathbin{::}[\mskip1.5mu \Conid{Newton}\mskip1.5mu]\to (\Conid{Int},\Conid{Int},\Conid{Int},\Conid{Int}){}\<[E]%
\\
\>[3]{}\Varid{ccn}\;\Varid{ls}\mathrel{=}({}\<[15]%
\>[15]{}\Varid{length}\;(\Varid{filter}\;(\equiv \Conid{H})\;\Varid{ls}),{}\<[E]%
\\
\>[15]{}\Varid{length}\;(\Varid{filter}\;(\equiv \Conid{X})\;\Varid{ls}),{}\<[E]%
\\
\>[15]{}\Varid{length}\;(\Varid{filter}\;(\equiv \Conid{Y})\;\Varid{ls}),{}\<[E]%
\\
\>[15]{}\Varid{length}\;(\Varid{filter}\;(\equiv \Conid{Z})\;\Varid{ls})){}\<[E]%
\ColumnHook
\end{pboxed}
\)\par\noindent\endgroup\resethooks
\end{minipage}

When we apply this function, \eg\ \ensuremath{\Varid{ccn}\;(\Varid{cn}\;\Conid{H}\;\mathrm{3})},
we see:

\ensuremath{(\mathrm{1},\mathrm{3},\mathrm{3},\mathrm{1})}

The binomial coefficients $\binom{3}{k}$, 
for $k \in \lbrace 0\dots 3\rbrace$.

To see some more examples we call
\ensuremath{\Varid{map}\;(\Varid{ccn}\mathbin{\circ}\Varid{cn}\;\Conid{H})\;[\mskip1.5mu \mathrm{4}\mathinner{\ldotp\ldotp}\mathrm{10}\mskip1.5mu]} and get

\begin{minipage}{\textwidth}
\ensuremath{[\mskip1.5mu (\mathrm{1},\mathrm{4},\mathrm{6},\mathrm{4}),}\\
\ensuremath{(\mathrm{1},\mathrm{5},\mathrm{10},\mathrm{10}),}\\
\ensuremath{(\mathrm{1},\mathrm{6},\mathrm{15},\mathrm{20}),}\\
\ensuremath{(\mathrm{1},\mathrm{7},\mathrm{21},\mathrm{35}),}\\
\ensuremath{(\mathrm{1},\mathrm{8},\mathrm{28},\mathrm{56}),}\\
\ensuremath{(\mathrm{1},\mathrm{9},\mathrm{36},\mathrm{84}),}\\
\ensuremath{(\mathrm{1},\mathrm{10},\mathrm{45},\mathrm{120})\mskip1.5mu]}
\end{minipage}

What we see, in terms of the table we used above, is

\begin{center}
\begingroup
\renewcommand{\arraystretch}{1.5}
\begin{tabular}{||c||c|c|c|c||}
\hline
     &  0    &  1    &  2    &  3   \\\hline
     & $n_0$ & $n_1$ & $n_2$ & $n_3$ \\\hline\hline
   H &$\binom{0}{0}$&$\binom{1}{0}$&$\binom{2}{0}$&$\binom{3}{0}$\\\hline
   X &$\binom{0}{1}$&$\binom{1}{1}$&$\binom{2}{1}$&$\binom{3}{1}$\\\hline
   Y &$\binom{0}{2}$&$\binom{1}{2}$&$\binom{2}{2}$&$\binom{3}{2}$\\\hline
   Z &$\binom{0}{3}$&$\binom{1}{3}$&$\binom{2}{3}$&$\binom{3}{3}$\\\hline
\end{tabular}
\endgroup
\end{center}

So, why do we see binomial coefficients and
can we prove that we will always see binomial coefficients?
To answer the first question, we will analyse the
execution tree of \ensuremath{\Varid{cn}}. Here is the tree for \ensuremath{\Varid{cn}\;\Conid{H}\;\mathrm{3}}:

\begin{center}
\begin{tikzpicture}
% root
\node (H3) at (6,     5) {\ensuremath{\Varid{cn}\;\Conid{H}\;\mathrm{3}}};

\node (H2) at (5 ,  4) {\ensuremath{\Varid{cn}\;\Conid{H}\;\mathrm{2}}};
\node (X2) at (7,   4) {\ensuremath{\Varid{cn}\;\Conid{X}\;\mathrm{2}}};

\node (X1) at (7  ,   3) {\ensuremath{\Varid{cn}\;\Conid{X}\;\mathrm{1}}};
\node (Y1) at (10 ,   3) {\ensuremath{\Varid{cn}\;\Conid{Y}\;\mathrm{1}}};

\node (X0) at (7  ,   2) {\ensuremath{\Varid{cn}\;\Conid{X}\;\mathrm{0}}};
\node (Y0) at (8.5,   2) {\ensuremath{\Varid{cn}\;\Conid{Y}\;\mathrm{0}}};

\node (X) at (7  ,   1) {\ensuremath{[\mskip1.5mu \Conid{X}\mskip1.5mu]}};
\node (Y) at (8.5,   1) {\ensuremath{[\mskip1.5mu \Conid{Y}\mskip1.5mu]}};

\node (Z)   at (10 ,  2) {\ensuremath{\Conid{Z}\mathbin{:}}};
\node (Y02) at (11.5,   2) {\ensuremath{\Varid{cn}\;\Conid{Y}\;\mathrm{0}}};
\node (Y')  at (11.5,  1) {\ensuremath{[\mskip1.5mu \Conid{Y}\mskip1.5mu]}};

\node (H1) at (4   ,  3) {\ensuremath{\Varid{cn}\;\Conid{H}\;\mathrm{1}}};
\node (X12) at (5.5 ,  3) {\ensuremath{\Varid{cn}\;\Conid{X}\;\mathrm{1}}};

\node (H0)  at (3,   2) {\ensuremath{\Varid{cn}\;\Conid{H}\;\mathrm{0}}};
\node (X02) at (4.5 ,2) {\ensuremath{\Varid{cn}\;\Conid{X}\;\mathrm{0}}};

\node (H) at  (3 ,    1) {\ensuremath{[\mskip1.5mu \Conid{H}\mskip1.5mu]}};

\connect{H3} {H2};
\connect{H3} {X2};
\connect{X2} {X1};
\connect{X2} {Y1};
\connect{X1} {X0};
\connect{X1} {Y0};
\connect{X0} {X};
\connect{Y0} {Y};
\connect{Y1} {Z};
\connect{Y1} {Y02};
\connect{Y02} {Y'};
\connect{H2} {H1};
\connect{H2} {X12};
\connect{H1} {H0};
\connect{H0} {H};
\connect{H1} {X02};
\end{tikzpicture}
\end{center}

On the left-hand side of the tree,
you see the main execution path 
calling \ensuremath{\Varid{cn}\;\Conid{H}\;(\Varid{n}\mathbin{-}\mathrm{1})} and \ensuremath{\Varid{cn}\;\Conid{X}\;(\Varid{n}\mathbin{-}\mathrm{1})}
on each level. The sketch expands \ensuremath{\Varid{cn}\;\Conid{X}}
only for one case, namely the top-level call
\ensuremath{\Varid{cn}\;\Conid{X}\;\mathrm{2}} on the right-hand side. 
Otherwise, the tree would be quite confusing.

Anyway, what we can see:
\begin{itemize}
\item Any top-level call of type \ensuremath{\Varid{cn}\;\Conid{A}} 
      (for $A \in \lbrace H,X,Y\rbrace$)
      creates only one \ensuremath{\Conid{A}};
      we therefore have always exactly one \ensuremath{\Conid{H}}.
\item Every call to \ensuremath{\Varid{cn}\;\Conid{H}\;\Varid{n}}, for $n > 0$,
      calls one instance of \ensuremath{\Varid{cn}\;\Conid{X}}.
      We therefore have exactly $n$ \ensuremath{\Conid{X}}.
\item Every call to \ensuremath{\Varid{cn}\;\Conid{X}\;\Varid{n}}, for $n > 0$,
      calls one instance of \ensuremath{\Varid{cn}\;\Conid{Y}}.
      We therefore have exactly $n$ \ensuremath{\Conid{Y}} per \ensuremath{\Varid{cn}\;\Conid{X}\;\Varid{n}},
      $n > 0$.
\item Every call to \ensuremath{\Varid{cn}\;\Conid{Y}\;\Varid{n}}, for $n>0$, creates one \ensuremath{\Conid{Z}}.
\item The call to \ensuremath{\Varid{cn}\;\Conid{X}\;\mathrm{1}} would expand to
      \ensuremath{\Varid{cn}\;\Conid{X}\;\mathrm{0}\plus \Varid{cn}\;\Conid{Y}\;\mathrm{0}}; it would, hence,
      create one more \ensuremath{\Conid{X}} and one more \ensuremath{\Conid{Y}}.
\item The call to \ensuremath{\Varid{cn}\;\Conid{X}\;\mathrm{0}} would create one more \ensuremath{\Conid{X}}.
\item This execution, thus, creates
      1 \ensuremath{\Conid{H}}, 3 \ensuremath{\Conid{X}}, 3 \ensuremath{\Conid{Y}} and 1 \ensuremath{\Conid{Z}}.
\end{itemize}

We now prove by induction that if a call to \ensuremath{\Varid{cn}\;\Conid{H}\;\Varid{n}}
creates 

\[
\binom{n}{0}H, \binom{n}{1}X, \binom{n}{2}Y 
\text{and} \binom{n}{3}Z
\]

(and the previous calls to \ensuremath{\Varid{cn}\;\Conid{H}\;(\Varid{n}\mathbin{-}\mathrm{1})}, \ensuremath{\Varid{cn}\;\Conid{H}\;(\Varid{n}\mathbin{-}\mathrm{2})},
$\dots$, \ensuremath{\Varid{cn}\;\Conid{H}\;\mathrm{0}} created similar patterns including
the binomial coefficients),
then \ensuremath{\Varid{cn}\;\Conid{H}\;(\Varid{n}\mathbin{+}\mathrm{1})} creates

\[
\binom{n+1}{0}H, \binom{n+1}{1}X, \binom{n+1}{2}Y 
\text{and} \binom{n+1}{3}Z.
\]

Note that the number of \ensuremath{\Conid{H}} does not increase,
because, as observed, each top-level call to \ensuremath{\Varid{cn}\;\Conid{A}\;\Varid{n}}
creates exactly one \ensuremath{\Conid{A}}.
If \ensuremath{\Varid{cn}\;\Conid{H}\;\Varid{n}} creates one \ensuremath{\Conid{H}},
\ensuremath{\Varid{cn}\;\Conid{H}\;(\Varid{n}\mathbin{+}\mathrm{1})} creates exactly one \ensuremath{\Conid{H}}, too.
We conclude that we create $\binom{n+1}{0}H$ as requested.

When we call \ensuremath{\Varid{cn}\;\Conid{H}\;(\Varid{n}\mathbin{+}\mathrm{1})}, we will call \ensuremath{\Varid{cn}\;\Conid{H}\;\Varid{n}}.
We, therefore, create all instances of \ensuremath{\Conid{X}} created by \ensuremath{\Varid{cn}\;\Conid{H}\;\Varid{n}}
plus those created in the first level of \ensuremath{\Varid{cn}\;\Conid{H}\;(\Varid{n}\mathbin{+}\mathrm{1})}.
This new level calls \ensuremath{\Varid{cn}\;\Conid{X}\;\Varid{n}} exaclty once,
which creates one \ensuremath{\Conid{X}} (because any top-level call to \ensuremath{\Varid{cn}\;\Conid{A}\;\Varid{n}}
creates exactly one \ensuremath{\Conid{A}}).
We, hence, create one \ensuremath{\Conid{X}} more.
This, however, is 
$\binom{n}{0} + \binom{n}{1} = \binom{n+1}{1}$
according to Pascal's Rule.
We conclude that we create $\binom{n+1}{1}X$ as requested.

Since we call \ensuremath{\Varid{cn}\;\Conid{H}\;\Varid{n}}, when we call \ensuremath{\Varid{cn}\;\Conid{H}\;(\Varid{n}\mathbin{+}\mathrm{1})}, 
we also create all instances of \ensuremath{\Conid{Y}} that were created
by \ensuremath{\Varid{cn}\;\Conid{H}\;\Varid{n}}. We additionally create all instances of \ensuremath{\Conid{Y}}
that are created by the new call to \ensuremath{\Varid{cn}\;\Conid{X}\;\Varid{n}}.
This, in its turn, calls $n$ instances of \ensuremath{\Varid{cn}\;\Conid{Y}}.
Since $n = \binom{n}{1}$ and any top-level call to 
\ensuremath{\Varid{cn}\;\Conid{Y}\;\Varid{n}} creates exactly one \ensuremath{\Conid{Y}}, we create
$\binom{n}{1} + \binom{n}{2} = \binom{n+1}{2}Y$ as requested.

Finally, since we call \ensuremath{\Varid{cn}\;\Conid{H}\;\Varid{n}}, when we call \ensuremath{\Varid{cn}\;\Conid{H}\;(\Varid{n}\mathbin{+}\mathrm{1})},
we also create all instances of \ensuremath{\Conid{Z}} that were created before.
But we call one more instance of \ensuremath{\Varid{cn}\;\Conid{X}\;\Varid{n}}, which creates a
certain amount of new \ensuremath{\Conid{Z}}. How many?
We create again all \ensuremath{\Conid{Z}} that were created anew by \ensuremath{\Varid{cn}\;\Conid{H}\;\Varid{n}},
those that did not exist in \ensuremath{\Varid{cn}\;\Conid{H}\;(\Varid{n}\mathbin{-}\mathrm{1})}.
Let us call the number of \ensuremath{\Conid{Z}} created by \ensuremath{\Varid{cn}\;\Conid{H}\;\Varid{n}} $z_n$
and the number of \ensuremath{\Conid{Z}} created by \ensuremath{\Varid{cn}\;\Conid{H}\;(\Varid{n}\mathbin{-}\mathrm{1})} $z_{n-1}$.
The number of \ensuremath{\Conid{Z}} created anew in \ensuremath{\Varid{cn}\;\Conid{H}\;\Varid{n}} is then 
$z_n - z_{n-1}$.

But since, in \ensuremath{\Varid{cn}\;\Conid{H}\;(\Varid{n}\mathbin{+}\mathrm{1})}, we call \ensuremath{\Varid{cn}\;\Conid{X}} one level up,
more \ensuremath{\Conid{Z}} are created than before.
All calls to \ensuremath{\Varid{cn}\;\Conid{Y}\;\mathrm{0}}, those that did not create a new \ensuremath{\Conid{Z}}
in \ensuremath{\Varid{cn}\;\Conid{H}\;\Varid{n}},
are now called as \ensuremath{\Varid{cn}\;\Conid{Y}\;\mathrm{1}} and, hence, create a \ensuremath{\Conid{Z}}
that was not created before. The calls to \ensuremath{\Varid{cn}\;\Conid{Y}\;\mathrm{0}} create
\ensuremath{\Conid{Y}} that were not created by \ensuremath{\Varid{cn}\;\Conid{H}\;(\Varid{n}\mathbin{-}\mathrm{1})}.
We, therefore, need to add to the number of \ensuremath{\Conid{Z}} the number
of \ensuremath{\Conid{Y}} that did not exist in \ensuremath{\Varid{cn}\;\Conid{H}\;(\Varid{n}\mathbin{-}\mathrm{1})}.
We use the same convention as for \ensuremath{\Conid{Z}}, \ie\
the  number of \ensuremath{\Conid{Y}} created anew in \ensuremath{\Varid{cn}\;\Conid{H}\;\Varid{n}} is
$y_n - y_{n-1}$.
The number of additional \ensuremath{\Conid{Z}} 
created by the additional call to 
\ensuremath{\Varid{cn}\;\Conid{X}\;\Varid{n}}, hence, is

\[
y_n - y_{n-1} + z_n - z_{n-1}
\]

But we are dealing with binomial coefficients.
We, therefore, have $z_n = y_{n-1} + z_{n-1}$
by Pascals' Rule applied backwards.
When we substitute this back, we get

\[
y_n - y_{n-1} + y_{n-1} + z_{n-1} - z_{n-1},
\]

which simplifies to $y_n$, \ie\ the number of 
instances of \ensuremath{\Conid{Y}} created by \ensuremath{\Varid{cn}\;\Conid{H}\;\Varid{n}}.
In other words: the number of \ensuremath{\Conid{Z}} we 
additionally create in \ensuremath{\Varid{cn}\;\Conid{H}\;(\Varid{n}\mathbin{+}\mathrm{1})} is the number
of \ensuremath{\Conid{Y}} in \ensuremath{\Varid{cn}\;\Conid{H}\;\Varid{n}}.
So, the complete number of \ensuremath{\Conid{Z}} we have 
in \ensuremath{\Varid{cn}\;\Conid{H}\;(\Varid{n}\mathbin{+}\mathrm{1})} is
the number of \ensuremath{\Conid{Y}} in \ensuremath{\Varid{cn}\;\Conid{H}\;\Varid{n}} 
plus the number \ensuremath{\Conid{Z}} in \ensuremath{\Varid{cn}\;\Conid{H}\;\Varid{n}}.
Since the number of \ensuremath{\Conid{Y}} is $\binom{n}{2}$
and the number of \ensuremath{\Conid{Z}} is  $\binom{n}{3}$,
we now have 
$\binom{n}{2} + \binom{n}{3} = \binom{n+1}{3}$ 
according to Pascal's Rule as requested
and this completes the proof.\qed
\section{Roots}
\ignore{
module Roots
where
  import DMachine
}

In the previous sections, we looked at the results,
when applying polynomials to given values.
That is, we applied a polynomial $\pi(x)$ to
a given value (or sequence of values) for $x$ 
and studied the result $y = \pi(x)$.
Now we are turning this around. We will look at
a given $y$ and ask which value $x$ would
create that $y$. In other words, 
we look at polynomials as equations of the form:

\begin{equation}
a_nx^n + a_{n-1}x^{n-1} + \dots + a_0 = a
\end{equation}

and search for ways to solve such equations.
In the focus of this investigation is usually
the special case $a=0$, \ie\:

\begin{equation}
a_nx^n + a_{n-1}x^{n-1} + \dots + a_0 = 0.
\end{equation}

The values for $x$ fulfilling this equation
are called the \term{root}s of the polynomial.
A trivial example is $x^2$, whose root is 0.
A slightly less trivial example is $x^2 - 4$,
whose roots are $x_1 = -2$ and $x_2 = 2$, since

\[
(-2)^2 - 4 = 4 - 4 = 0
\]

and

\[
2^2 - 4 = 4 - 4 = 0.
\]

Note that these examples are polynomials of even degree.
Polynomials of even degree do not need to have any roots.
Since even powers are always positive (or zero), negative values
are turned into positive numbers and, since the term of highest
degree is even, the whole expression may always be positive.
This is true for the polynomial $x^2 + 1$. Since all negative values
are transformed into positive values by $x^2$, the smallest value
that we can reach is the result for $x=0$, which is $0+1=1$.

On the other hand, even polynomials may have negative values,
namely when they have terms with negative coefficients that, 
for smaller numbers,
result in numbers that are greater than those resulting from
the term of highest degree. 
The polynomial $x^2 - 4$, for instance, is negative
in the interval $]-2\dots 2[$. It, therefore, must have two roots:
one at -2, where the polynomial results become negative, and the other at 2,
where the polynomial results become positive again.

Odd polynomials, by contrast, usually have negative values, because
the term with the highest degree may result in a negative or a
positive number depending on the signedness of the input value
and that of the coefficient.
The trivial polynomial $x^3$, for instance, is negative for
negative values and positive for positive values. The slightly
less trivial polynomial $x^3 + 9$ has its root at -3, while
$x^3 - 9$ has its root at 3.

In summary, we can say that even polynomials do not necessarily
have negative values and, hence, do not need to have a root.
Odd polynomials, on the other hand, usually have both, negative
and positive values, and, hence, must have a root.

Those are strong claims. They are true, because polynomials
belong to a specific set of \term{functions}, 
namely \term{continuous} functions.
That, basically, means that they have no \emph{holes}, \ie\
for any value $x$ of a certain number type there is a result $y$
of that number type. For instance, when the coefficients of the
polynomial are all integers and the $x$-value is an integer,
then the result is an integer, too. When the polynomial is defined
over a field (all coefficients are part of that field and
the values to which we apply the polynomial lie in that field),
then the result is in that field, too. Rational polynomials,
for instance, have rational results. 
Real polynomials have real results.

Furthermore, the function does not ``jump'', \ie\ the growth
of the results growth with the input values -- not necessarily
at the same rate, in fact, for polynomials of degree greater 1,
the result grows much faster than the input -- but the growth
is regular.

These properties appear to be ``natural'' at the first sight.
But there are functions that do not fulful these criteria.
In the next chapter, when we properly define the term \term{function},
we will actually see functions with holes and jumps.

The reason that polynomials behave regularily is that we only
use basic arithmetic functions in their definition: we add, multiply
and raise to powers. 
Those operations together with the integers form monoids and,
with rational and real numbers, they form groups.
Both, monoids and groups, are closed over their base set.
We can therefore be sure that, for any input value from the base set,
the result is in the same base set, too.

Furthermore, the form of polynomials guarantees that they develop
in a certain way. For very large numbers (negative or positive), 
it is the term with the greatest exponents, \ie\ the degree 
of the polynomial, that most significantly determines the
outcome, that is, the result for very large numbers
approaches the result for the term with the largest exponent. 
For smaller values, however, the terms of lower degree have
stronger impact. The terms ``large'' and ``small'', here, 
must be understood relative to the coefficients. If the coefficients
are very large, the values to which the polynomial is applied
must be even larger to approach the result for the first term.

There are also polynomials with a quite confusing behaviour that
make it hard to guess the roots, for instance, Wilkinson's polynomial
named for James Hardy Wilkinson (1919 -- 1986), an American mathematician
and computer scientist. The Wilkinson polynomial is defined as

\begin{equation}
w(x) = \prod_{i=1}^{20}{(x-i)}.
\end{equation}

We can generate it in terms of our polynomial type as

\begin{minipage}{\textwidth}
\begingroup\par\noindent\advance\leftskip\mathindent\(
\begin{pboxed}\SaveRestoreHook
\column{B}{@{}>{\hspre}l<{\hspost}@{}}%
\column{3}{@{}>{\hspre}l<{\hspost}@{}}%
\column{E}{@{}>{\hspre}l<{\hspost}@{}}%
\>[3]{}\Varid{wilkinson}\mathbin{::}(\Conid{Num}\;\Varid{a},\Conid{Enum}\;\Varid{a},\Conid{Show}\;\Varid{a},\Conid{Eq}\;\Varid{a})\Rightarrow \Conid{Poly}\;\Varid{a}{}\<[E]%
\\
\>[3]{}\Varid{wilkinson}\mathrel{=}\Varid{prodp}\;\Varid{mul}\;[\mskip1.5mu \Conid{P}\;[\mskip1.5mu \mathbin{-}\Varid{i},\mathrm{1}\mskip1.5mu]\mid \Varid{i}\leftarrow [\mskip1.5mu \mathrm{1}\mathinner{\ldotp\ldotp}\mathrm{20}\mskip1.5mu]\mskip1.5mu]{}\<[E]%
\ColumnHook
\end{pboxed}
\)\par\noindent\endgroup\resethooks
\end{minipage}

It looks like this:

\ensuremath{\Conid{P}\;[\mskip1.5mu }\\
\ensuremath{\mathrm{2432902008176640000},\mathbin{-}\mathrm{8752948036761600000},\mathrm{13803759753640704000},}\\
\ensuremath{\mathbin{-}\mathrm{12870931245150988800},\mathrm{8037811822645051776},\mathbin{-}\mathrm{3599979517947607200},}\\
\ensuremath{\mathrm{1206647803780373360},\mathbin{-}\mathrm{311333643161390640},\mathrm{63030812099294896},}\\
\ensuremath{\mathbin{-}\mathrm{10142299865511450},\mathrm{1307535010540395},\mathbin{-}\mathrm{135585182899530},\mathrm{11310276995381},}\\
\ensuremath{\mathbin{-}\mathrm{756111184500},\mathrm{40171771630},\mathbin{-}\mathrm{1672280820},\mathrm{53327946},\mathbin{-}\mathrm{1256850},\mathrm{20615},\mathbin{-}\mathrm{210},\mathrm{1}\mskip1.5mu]}

The first terms are

\[
x^{20} - 210x^{19} + 20615x^{18} - 1256850x^{17} \dots
\]

When we apply Wilkinson's polynomial to the integers $1\dots 10$, we see:

\ensuremath{\mathrm{0},\mathrm{0},\mathrm{0},\mathrm{0},\mathrm{0},\mathrm{0},\mathrm{0},\mathrm{0},\mathrm{0},\mathrm{0},\mathrm{0},\mathrm{0},\mathrm{0},\mathrm{0},\mathrm{0},\mathrm{0},\mathrm{0},\mathrm{0},\mathrm{0},\mathrm{0},\mathrm{2432902008176640000},}\\
\ensuremath{\mathrm{51090942171709440000},\mathrm{562000363888803840000},\mathrm{4308669456480829440000},}\\
\ensuremath{\mathrm{25852016738884976640000}},

which looks very confusing. When we try non-integers, we see

\ensuremath{\Varid{apply}\;\Varid{wilkinson}\;\mathrm{0.9}} is $1.7213\dots$\\
\ensuremath{\Varid{apply}\;\Varid{wilkinson}\;\mathrm{1.1}} is $-8.4600\dots$\\
\ensuremath{\Varid{apply}\;\Varid{wilkinson}\;\mathrm{1.9}} is $-8.1111\dots$\\
\ensuremath{\Varid{apply}\;\Varid{wilkinson}\;\mathrm{2.1}} is $4.9238\dots$\\

As we see, the results switch sign at the integers or,
more precisely, at the integers in the interval $[1\dots 20]$,
which are the roots of Wilkinson's polynomial.
Looking at the factors of the polynomial

\[
(x-1)(x-2)\dots (x-20),
\]

this result is much less surprising, since, obviously,
when any of these factors becomes 0, then the whole
expression becomes 0. So, for the value $x=3$, we would have

\[
2 \times 1 \times 0 \times \dots \times -17 = 0.
\]

The results when applying the polynomial, however,
look quite irregular and, on the first sight,
completely unrelated to the coefficients.
When we say that polynomials show a regular behaviour,
that must be, therefore, taken with a grain of salt.
Anyway, that they behave like this 
gives rise to a number of simple
methods to find roots based on approximation,
at least when we start with a fair guess,
which requires some knowledge about the rough shape
of the polynomial in the first place.

These methods can be split into two major groups:
\term{bracketing} methods and \term{open} methods.
Bracketing methods start with two distinct values
somewhere on the ``left'' and the ``right'' of
the root. Bracketing methods, hence, require a
pre-knowledge about where, more or less, a root
is located. We then choose two values that limit
this interval on the lower and on the upper side.

The simplest variant of bracketing is the \term{bisect}
algorithm. It is very similar to Heron's method
to find the square root of a given number.
We start with two values $a$ and $b$ and, on each step,
we compute the average $(a+b)/2$ and substitute
either $a$ or $b$ by this value depending on the side
the value is located relative to the root.
Here is an implementation:

\begin{minipage}{\textwidth}
\begingroup\par\noindent\advance\leftskip\mathindent\(
\begin{pboxed}\SaveRestoreHook
\column{B}{@{}>{\hspre}l<{\hspost}@{}}%
\column{3}{@{}>{\hspre}l<{\hspost}@{}}%
\column{5}{@{}>{\hspre}l<{\hspost}@{}}%
\column{12}{@{}>{\hspre}l<{\hspost}@{}}%
\column{14}{@{}>{\hspre}l<{\hspost}@{}}%
\column{16}{@{}>{\hspre}c<{\hspost}@{}}%
\column{16E}{@{}l@{}}%
\column{19}{@{}>{\hspre}l<{\hspost}@{}}%
\column{22}{@{}>{\hspre}l<{\hspost}@{}}%
\column{47}{@{}>{\hspre}c<{\hspost}@{}}%
\column{47E}{@{}l@{}}%
\column{50}{@{}>{\hspre}l<{\hspost}@{}}%
\column{E}{@{}>{\hspre}l<{\hspost}@{}}%
\>[3]{}\Varid{bisect}\mathbin{::}{}\<[14]%
\>[14]{}(\Conid{Num}\;\Varid{a},\Conid{Eq}\;\Varid{a},\Conid{Ord}\;\Varid{a},\Conid{Fractional}\;\Varid{a},\Conid{Show}\;\Varid{a}){}\<[E]%
\\
\>[14]{}\Rightarrow \Conid{Poly}\;\Varid{a}\to \Varid{a}\to \Varid{a}\to \Varid{a}\to \Varid{a}{}\<[E]%
\\
\>[3]{}\Varid{bisect}\;\Varid{p}\;\Varid{t}\;\Varid{a}\;\Varid{b}{}\<[19]%
\>[19]{}\mid {}\<[22]%
\>[22]{}\Varid{abs}\;\Varid{fc}\mathbin{<}\Varid{abs}\;\Varid{t}{}\<[47]%
\>[47]{}\mathrel{=}{}\<[47E]%
\>[50]{}\Varid{c}{}\<[E]%
\\
\>[19]{}\mid {}\<[22]%
\>[22]{}\Varid{signum}\;\Varid{fc}\equiv \Varid{signum}\;\Varid{fa}{}\<[47]%
\>[47]{}\mathrel{=}{}\<[47E]%
\>[50]{}\Varid{bisect}\;\Varid{p}\;\Varid{t}\;\Varid{c}\;\Varid{b}{}\<[E]%
\\
\>[19]{}\mid {}\<[22]%
\>[22]{}\Varid{otherwise}{}\<[47]%
\>[47]{}\mathrel{=}{}\<[47E]%
\>[50]{}\Varid{bisect}\;\Varid{p}\;\Varid{t}\;\Varid{a}\;\Varid{c}{}\<[E]%
\\
\>[3]{}\hsindent{2}{}\<[5]%
\>[5]{}\mathbf{where}\;{}\<[12]%
\>[12]{}\Varid{fa}{}\<[16]%
\>[16]{}\mathrel{=}{}\<[16E]%
\>[19]{}\Varid{apply}\;\Varid{p}\;\Varid{a}{}\<[E]%
\\
\>[12]{}\Varid{fb}{}\<[16]%
\>[16]{}\mathrel{=}{}\<[16E]%
\>[19]{}\Varid{apply}\;\Varid{p}\;\Varid{b}{}\<[E]%
\\
\>[12]{}\Varid{fc}{}\<[16]%
\>[16]{}\mathrel{=}{}\<[16E]%
\>[19]{}\Varid{apply}\;\Varid{p}\;\Varid{c}{}\<[E]%
\\
\>[12]{}\Varid{c}{}\<[16]%
\>[16]{}\mathrel{=}{}\<[16E]%
\>[19]{}(\Varid{a}\mathbin{+}\Varid{b})\mathbin{/}\mathrm{2}{}\<[E]%
\ColumnHook
\end{pboxed}
\)\par\noindent\endgroup\resethooks
\end{minipage}

The function receives four arguments.
The first is the polynomial.
The second is a tolerance.
When we reach a result that is smaller
than the tolerance, we return the result.
$a$ and $b$ are the starting values.

We distinguish three cases:
\begin{itemize}
\item The result for the new value, $c$, 
      is below the tolerance threshold.
      In this case, $c$ is sufficiently close
      to the root and we yield this value.
\item the sign of the result for the new value
      equals the sign of $a$. Then we replace
      $a$ by $c$.
\item the sign of the result for the new value
      equals the sign of $b$. In this case,
      we replace $b$ by $c$.
\end{itemize}

We try \ensuremath{\Varid{bisect}} on the polynomial $x^2$ with
the initial guess $a=-1$ and $b=1$ (because we
assume that the root should be close to 0) and
a tolerance of 0.1:

\ensuremath{\Varid{bisect}\;(\Conid{P}\;[\mskip1.5mu \mathrm{0},\mathrm{0},\mathrm{1}\mskip1.5mu])\;\mathrm{0.1}\;(\mathbin{-}\mathrm{1})\;\mathrm{1}}

and see the correct result \ensuremath{\mathrm{0.0}}.

For the polynomial $x^2 - 4$, which has two roots,
we try 

\ensuremath{\Varid{bisect}\;(\Conid{P}\;[\mskip1.5mu \mathbin{-}\mathrm{4},\mathrm{0},\mathrm{1}\mskip1.5mu])\;\mathrm{0.1}\;(\mathbin{-}\mathrm{3})\;(\mathbin{-}\mathrm{1})},

which yields \ensuremath{\mathbin{-}\mathrm{2}} and 

\ensuremath{\Varid{bisect}\;(\Conid{P}\;[\mskip1.5mu \mathbin{-}\mathrm{4},\mathrm{0},\mathrm{1}\mskip1.5mu])\;\mathrm{0.1}\;\mathrm{1}\;\mathrm{3}},

which yields \ensuremath{\mathrm{2}}.

With Wilkinson's polynomial, however,
we get a surprise:

\ensuremath{\Varid{bisect}\;\Varid{wilkinson}\;\mathrm{0.1}\;\mathrm{0.5}\;\mathrm{1.5}},

for which we expect to find the root 1.
But the function does not return.
Indeed, when we try \ensuremath{\Varid{apply}\;\Varid{wilkinson}\;\mathrm{1.0}}, we see

\ensuremath{\mathrm{1148.0}},

a somewhat surprising result.
Wilkinson used this polynomial to demonstrate
the sensivity of coefficients to small differences
in the input values. Using Haskell real numbers,
The computation in leads to a loss of precision
in representing the terms. Indeed, considering
terms raised to the $20^{th}$ power and multiplied
by large coefficients, the number 1148 appears to
be a tiny inprecision.

We can work around this, using rational numbers:

\ensuremath{\Varid{apply}\;\Varid{wilkinson}\;(\mathrm{1}\mathbin{\%}\mathrm{1})}

gives without any surprise \ensuremath{\mathrm{0}\mathbin{\%}\mathrm{1}}.
So, we try

\ensuremath{\Varid{bisect}\;\Varid{wilkinson}\;(\mathrm{1}\mathbin{\%}\mathrm{10})\;(\mathrm{1}\mathbin{\%}\mathrm{2})\;(\mathrm{3}\mathbin{\%}\mathrm{2})}

and get the correct result \ensuremath{\mathrm{1}\mathbin{\%}\mathrm{1}}.
The function with this parameters
returns almost instantanious. That is
because the average of 0.5 and 1.5 is already 1.
The function finds the root in the first step.
A more serious challenge is

\ensuremath{\Varid{bisect}\;\Varid{wilkinson}\;(\mathrm{1}\mathbin{\%}\mathrm{10})\;(\mathrm{1}\mathbin{\%}\mathrm{3})\;(\mathrm{3}\mathbin{\%}\mathrm{2})},

which needs more than one recursion.
The function, now, runs for a short while and
comes up with the result

\ensuremath{\mathrm{1729382256910270463}\mathbin{\%}\mathrm{1729382256910270464}},

which is pretty close to 1 and, hence,
the correct result.

The most widely known open method is Newton's method,
also called Newton-Raphson method.
It was first developed by Newton in about 1670
and, then, by Joseph Raphson in 1690.
Newton's version was probably not known to Raphson,
since Newton did not publish the work in which
he presents the approach. Raphson's version is
simpler and, therefore, preferred.

Anyway, the method starts with only one approximation
and is therefore not a bracketing method.
The approximation is then applied to the polynomial $\pi$
and the derivative of the polynomial $\pi'$.
Then, the quotient of the results, $\frac{\pi(x)}{\pi'(x)}$ 
is computed and subtracted from the initial guess.
Here is an implementation:

\begin{minipage}{\textwidth}
\begingroup\par\noindent\advance\leftskip\mathindent\(
\begin{pboxed}\SaveRestoreHook
\column{B}{@{}>{\hspre}l<{\hspost}@{}}%
\column{3}{@{}>{\hspre}l<{\hspost}@{}}%
\column{5}{@{}>{\hspre}l<{\hspost}@{}}%
\column{12}{@{}>{\hspre}l<{\hspost}@{}}%
\column{16}{@{}>{\hspre}l<{\hspost}@{}}%
\column{17}{@{}>{\hspre}c<{\hspost}@{}}%
\column{17E}{@{}l@{}}%
\column{20}{@{}>{\hspre}l<{\hspost}@{}}%
\column{21}{@{}>{\hspre}c<{\hspost}@{}}%
\column{21E}{@{}l@{}}%
\column{24}{@{}>{\hspre}l<{\hspost}@{}}%
\column{36}{@{}>{\hspre}c<{\hspost}@{}}%
\column{36E}{@{}l@{}}%
\column{39}{@{}>{\hspre}l<{\hspost}@{}}%
\column{E}{@{}>{\hspre}l<{\hspost}@{}}%
\>[3]{}\Varid{newguess}\mathbin{::}{}\<[16]%
\>[16]{}(\Conid{Num}\;\Varid{a},\Conid{Eq}\;\Varid{a},\Conid{Ord}\;\Varid{a},\Conid{Enum}\;\Varid{a},\Conid{Fractional}\;\Varid{a}){}\<[E]%
\\
\>[16]{}\Rightarrow \Conid{Poly}\;\Varid{a}\to \Conid{Natural}\to \Varid{a}\to \Varid{a}\to \Varid{a}{}\<[E]%
\\
\>[3]{}\Varid{newguess}\;\Varid{p}\;\Varid{m}\;\Varid{t}\;\Varid{a}{}\<[21]%
\>[21]{}\mid {}\<[21E]%
\>[24]{}\Varid{abs}\;\Varid{pa}\mathbin{<}\Varid{t}{}\<[36]%
\>[36]{}\mathrel{=}{}\<[36E]%
\>[39]{}\Varid{a}{}\<[E]%
\\
\>[21]{}\mid {}\<[21E]%
\>[24]{}\Varid{m}\leq \mathrm{0}{}\<[36]%
\>[36]{}\mathrel{=}{}\<[36E]%
\>[39]{}\Varid{a}{}\<[E]%
\\
\>[21]{}\mid {}\<[21E]%
\>[24]{}\Varid{otherwise}{}\<[36]%
\>[36]{}\mathrel{=}{}\<[36E]%
\>[39]{}\Varid{newguess}\;\Varid{p}\;(\Varid{m}\mathbin{-}\mathrm{1})\;\Varid{t}\;(\Varid{a}\mathbin{-}\Varid{pa}\mathbin{/}\Varid{p'a}){}\<[E]%
\\
\>[3]{}\hsindent{2}{}\<[5]%
\>[5]{}\mathbf{where}\;{}\<[12]%
\>[12]{}\Varid{p'}{}\<[17]%
\>[17]{}\mathrel{=}{}\<[17E]%
\>[20]{}\Varid{derivative}\;(\mathbin{*})\;\Varid{p}{}\<[E]%
\\
\>[12]{}\Varid{pa}{}\<[17]%
\>[17]{}\mathrel{=}{}\<[17E]%
\>[20]{}\Varid{apply}\;\Varid{p}\;\Varid{a}{}\<[E]%
\\
\>[12]{}\Varid{p'a}{}\<[17]%
\>[17]{}\mathrel{=}{}\<[17E]%
\>[20]{}\Varid{apply}\;\Varid{p'}\;\Varid{a}{}\<[E]%
\ColumnHook
\end{pboxed}
\)\par\noindent\endgroup\resethooks
\end{minipage}

The function receives four parameters.
The polynomial \ensuremath{\Varid{p}}, the natural number \ensuremath{\Varid{m}},
the tolerance \ensuremath{\Varid{t}} and the initial guess \ensuremath{\Varid{a}}.
The natural number \ensuremath{\Varid{m}} is a limiter.
It is not guaranteed that the value increases
in precision with always more repetitions.
It may get worse at some point.
It is therefore useful to restrict the number
of iterations.

The function terminates when we have 
reached either the intended precision or 
the number of repetitions, $m$.
Otherwise, we repeat with $m-1$ and with
$a - \frac{\pi(a)}{\pi'(a)}$.

For the polynomial $x^2 - 4$, we call first

\ensuremath{\Varid{newguess}\;(\Conid{P}\;[\mskip1.5mu \mathbin{-}\mathrm{4},\mathrm{0},\mathrm{1}\mskip1.5mu])\;\mathrm{10}\;\mathrm{0.1}\;\mathrm{1}}

and get $2.00069\dots$, which is very close
to the known root 2.
For the other root we call

\ensuremath{\Varid{newguess}\;(\Conid{P}\;[\mskip1.5mu \mathbin{-}\mathrm{4},\mathrm{0},\mathrm{1}\mskip1.5mu])\;\mathrm{10}\;\mathrm{0.1}\;(\mathbin{-}\mathrm{1})}

and get the equally close result $-2.00069\dots$
For the Wilkinson polynomial, we call

\ensuremath{\Varid{newguess}\;\Varid{wilkinson}\;\mathrm{20}\;(\mathrm{0.0001})\;\mathrm{10}\;\mathrm{1.5}}

and get $1.99999\dots$, which is very close
to the real root 2. We can further improve
precision by increasing the number of iterations:

\ensuremath{\Varid{newguess}\;\Varid{wilkinson}\;\mathrm{20}\;(\mathrm{0.0001})\;\mathrm{20}\;\mathrm{1.5}}

The difference is at the $12^{th}$ decimal digit.

Note that the Newton-Raphson method
is not only more precise (that is: converges earlier
with a good result), but also more robust against
real representation imprecision.

To understand why this method works at all,
we need to better understand what the derivative is.
We will come back to this issue in the next chapter.
In the strict sense, the derivative does not belong
here anyway, since the concept of derivative is
analysis, not algebra. Both kinds of methods,
the bracketing and the open methods, in fact, come
from numerical analysis.
They do not have the ``look and feel'' of algebraic
methods. So, how would an algebraist tackle the
problem of finding the roots of a polynomial?

\ignore{
 - most root-finding algorithms are analysis not algebra

 - factoring (https://en.wikipedia.org/wiki/Factorization_of_polynomials)
   + simple methods
 - quadratic formula
}
\section{Vieta's Formula}
\section{The Method of partial Fractions}
\ignore{
http://www.purplemath.com/modules/partfrac.htm
}
\section{Generationfunctionology 1}
\section{The closed Form of the Fibonacci Sequence}
\ignore{
\begingroup\par\noindent\advance\leftskip\mathindent\(
\begin{pboxed}\SaveRestoreHook
\column{B}{@{}>{\hspre}l<{\hspost}@{}}%
\column{E}{@{}>{\hspre}l<{\hspost}@{}}%
\>[B]{}\mathbf{module}\;\Conid{ClosedFib}{}\<[E]%
\\
\>[B]{}\mathbf{where}{}\<[E]%
\ColumnHook
\end{pboxed}
\)\par\noindent\endgroup\resethooks
}

\begin{equation}
G(x) = F_0 + F_1x + F_2x^2 + F_3x^3 + \dots
\end{equation}

\begin{equation}
G(x) = 0 + x + x^2 + 2x^3 + 3x^4 + 5x^5 + 8x^6 + \dots
\end{equation}

\begin{equation}
xG(x) = F_0x + F_1x^2 + F_2x^3 + F_3x^4 + \dots
\end{equation}

\begin{equation}
x^2G(x) = F_0x^2 + F_1x^3 + F_2x^4 + F_3x^5 + \dots
\end{equation}

\begin{equation}
G(x) - xG(x) - x^2G(x) = (1-x-x^2)G(x).
\end{equation}

\begin{align*}
(1-x-x^2)G(x) & = & (&F_0 & + & F_1x & + & F_2x^2 & + & F_3x^3 & + & \dots) & - \\
              &   & (&    &   & F_0x & + & F_1x^2 & + & F_2x^3 & + & \dots) & - \\
              &   & (&    &   &      & + & F_0x^2 & + & F_1x^3 & + & \dots) &
\end{align*}

\begin{align*}
(1-x-x^2)G(x) & = & F_0 & + (F_1 - F_0)x \\
              &   &     & + (F_2 - F_1 - F_0)x^2 \\
              &   &     & + (F_3 - F_2 - F_1)x^3 \\
              &   &     & + \dots
\end{align*}

\begin{equation}
(1-x-x^2)G(x) = x.
\end{equation}

\begin{equation}
G(x) = \frac{x}{1-x-x^2}.
\end{equation}

\ignore{
- factor denominator using roots:
  - (x+1/2+sqrt(5)/2) 
  - (x+1/2-sqrt(5)/2)
- then use partial fractions
}



\section{Factoring Polynomials}


\end{document}
