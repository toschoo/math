%% =======================================================
%% (c) Tobias Schoofs
%% =======================================================
%% Math 4 Programmers - The Inverse Element
%% =======================================================

% Plain Style
\documentclass[tikz]{scrreprt}

% Springer Style
%\documentclass[envcountsame]{llncs}

%% ODER: format ==         = "\mathrel{==}"
%% ODER: format /=         = "\neq "
%
%
\makeatletter
\@ifundefined{lhs2tex.lhs2tex.sty.read}%
  {\@namedef{lhs2tex.lhs2tex.sty.read}{}%
   \newcommand\SkipToFmtEnd{}%
   \newcommand\EndFmtInput{}%
   \long\def\SkipToFmtEnd#1\EndFmtInput{}%
  }\SkipToFmtEnd

\newcommand\ReadOnlyOnce[1]{\@ifundefined{#1}{\@namedef{#1}{}}\SkipToFmtEnd}
\usepackage{amstext}
\usepackage{amssymb}
\usepackage{stmaryrd}
\DeclareFontFamily{OT1}{cmtex}{}
\DeclareFontShape{OT1}{cmtex}{m}{n}
  {<5><6><7><8>cmtex8
   <9>cmtex9
   <10><10.95><12><14.4><17.28><20.74><24.88>cmtex10}{}
\DeclareFontShape{OT1}{cmtex}{m}{it}
  {<-> ssub * cmtt/m/it}{}
\newcommand{\texfamily}{\fontfamily{cmtex}\selectfont}
\DeclareFontShape{OT1}{cmtt}{bx}{n}
  {<5><6><7><8>cmtt8
   <9>cmbtt9
   <10><10.95><12><14.4><17.28><20.74><24.88>cmbtt10}{}
\DeclareFontShape{OT1}{cmtex}{bx}{n}
  {<-> ssub * cmtt/bx/n}{}
\newcommand{\tex}[1]{\text{\texfamily#1}}	% NEU

\newcommand{\Sp}{\hskip.33334em\relax}


\newcommand{\Conid}[1]{\mathit{#1}}
\newcommand{\Varid}[1]{\mathit{#1}}
\newcommand{\anonymous}{\kern0.06em \vbox{\hrule\@width.5em}}
\newcommand{\plus}{\mathbin{+\!\!\!+}}
\newcommand{\bind}{\mathbin{>\!\!\!>\mkern-6.7mu=}}
\newcommand{\rbind}{\mathbin{=\mkern-6.7mu<\!\!\!<}}% suggested by Neil Mitchell
\newcommand{\sequ}{\mathbin{>\!\!\!>}}
\renewcommand{\leq}{\leqslant}
\renewcommand{\geq}{\geqslant}
\usepackage{polytable}

%mathindent has to be defined
\@ifundefined{mathindent}%
  {\newdimen\mathindent\mathindent\leftmargini}%
  {}%

\def\resethooks{%
  \global\let\SaveRestoreHook\empty
  \global\let\ColumnHook\empty}
\newcommand*{\savecolumns}[1][default]%
  {\g@addto@macro\SaveRestoreHook{\savecolumns[#1]}}
\newcommand*{\restorecolumns}[1][default]%
  {\g@addto@macro\SaveRestoreHook{\restorecolumns[#1]}}
\newcommand*{\aligncolumn}[2]%
  {\g@addto@macro\ColumnHook{\column{#1}{#2}}}

\resethooks

\newcommand{\onelinecommentchars}{\quad-{}- }
\newcommand{\commentbeginchars}{\enskip\{-}
\newcommand{\commentendchars}{-\}\enskip}

\newcommand{\visiblecomments}{%
  \let\onelinecomment=\onelinecommentchars
  \let\commentbegin=\commentbeginchars
  \let\commentend=\commentendchars}

\newcommand{\invisiblecomments}{%
  \let\onelinecomment=\empty
  \let\commentbegin=\empty
  \let\commentend=\empty}

\visiblecomments

\newlength{\blanklineskip}
\setlength{\blanklineskip}{0.66084ex}

\newcommand{\hsindent}[1]{\quad}% default is fixed indentation
\let\hspre\empty
\let\hspost\empty
\newcommand{\NB}{\textbf{NB}}
\newcommand{\Todo}[1]{$\langle$\textbf{To do:}~#1$\rangle$}

\EndFmtInput
\makeatother
%

%% =======================================================
%% (c) Tobias Schoofs
%% =======================================================
%% Commands 4 Programmers
%% =======================================================

%include lhs2TeX.fmt
%include lhs2TeX.sty

\usepackage[pdftex]{graphicx}
\usepackage{ucs}
\usepackage[utf8x]{inputenc} 
\usepackage{tabto}
%\usepackage[T1]{fontenc} 
\usepackage[portuguese, english]{babel}
\usepackage{amsfonts}

\usepackage{amsmath}
\usepackage{amssymb}
\usepackage{amsthm}
\usepackage{amscd}

\usepackage{siunitx}

\usepackage{tikz}
\usepackage{pgfplots}
%\usetikzlibrary{graphs,graphdrawing,arrows.meta}
\usetikzlibrary{graphs} % graphdrawing,arrows.meta}
%\usegdlibrary{trees}

\long\def\ignore#1{}

\newcommand{\acronym}[1]{\textsc{#1}}

\newcommand{\term}[1]{\textit{#1}}
\newcommand{\latin}[1]{\textit{#1}}
\newcommand{\speech}[1]{\textit{#1}}

\newcommand{\ie}{\textit{i.e.}}
\newcommand{\eg}{\textit{e.g.}}
\newcommand{\etc}{\textit{etc.}}
\newcommand{\viz}{\textit{viz.}}

\newcommand{\haskell}[1]{{\ttfamily #1}}

\newcommand{\Rom}[1]{\uppercase\expandafter{\romannumeral #1\relax}}

\DeclareRobustCommand{\stirlingOne}{\genfrac[]{0pt}{}}
\DeclareRobustCommand{\stirlingTwo}{\genfrac\{\}{0pt}{}}
\DeclareRobustCommand{\eulerianOne}{\genfrac<>{0pt}{}}
\newcommand{\eulerianTwo}[2]{\left\langle\!\!\left\langle{#1 \atop #2}\right\rangle\!\!\right\rangle}

\newcommand{\connect}[2]{
  \draw [-,color=black] (#1) to (#2)
}

% \renewcommand{\gcd}{\textsc{gcd}}


\begin{document}
\setlength{\parindent}{0pt}
\setlength{\parskip}{8pt}

\chapter{Real Numbers} % c06
\section{$\sqrt{2}$}
\ignore{
\begingroup\par\noindent\advance\leftskip\mathindent\(
\begin{pboxed}\SaveRestoreHook
\column{B}{@{}>{\hspre}l<{\hspost}@{}}%
\column{3}{@{}>{\hspre}l<{\hspost}@{}}%
\column{E}{@{}>{\hspre}l<{\hspost}@{}}%
\>[B]{}\mathbf{module}\;\Conid{Sqrt2}{}\<[E]%
\\
\>[B]{}\mathbf{where}{}\<[E]%
\\
\>[B]{}\hsindent{3}{}\<[3]%
\>[3]{}\mathbf{import}\;\Conid{Natural}{}\<[E]%
\\
\>[B]{}\hsindent{3}{}\<[3]%
\>[3]{}\mathbf{import}\;\Conid{Quoz}{}\<[E]%
\ColumnHook
\end{pboxed}
\)\par\noindent\endgroup\resethooks
}

Until now, we have looked at \term{discrete} numbers,
that is numbers that are nicely separated from each other
so that we can write them down unmistakebly
and always know of which number we are currently talking.
Now we enter a completely different universe.
The universe of continuous numbers that cannot be
written down in a finite number of steps.
The representation of these numbers consists of
infinitely many elements and, therefore, we will never
be able to write the number down completely
with all its elements.
We may give a finite formula 
that describes how to compute the specific number,
but we will never see the whole number written down.
Those numbers are known since antiquity and,
apparently, their existence came as a great surprise
to Greek mathematicians.

The first step of our investigations into
this kind of numbers, is to show that they \term{exists},
\ie\, there are contexts where they arise naturally.
To start, we will assume that they are not necessary.
We assume that all numbers are either natural,
integral or fractional.
Indeed, any of the fundamental arithmetic operations,
$+,-,\times$ and $/$, applied on two natural numbers,
results either in another natural number, a negative number
or a fraction. We could therefore suspect
that the result of any operation is a rational number,
\ie\ an integeger or a fraction.

What about $\sqrt{2}$, the square root of 2?
Let us assume that $\sqrt{2}$ is as well a fraction.
Then we have two integers $n$ and $d$,
coprime to each other, such that

\begin{equation}
\sqrt{2} = \frac{n}{d}
\end{equation}

and

\begin{equation}
2 = \left(\frac{n}{d}\right)^2 = \frac{n^2}{d^2}.
\end{equation}

Any number can be represented as a product of primes.
If $p_1p_2\dots p_n$ is the prime factorsiation of $n$
and $q_1q_2\dots q_d$ is that of $d$,
we can write:

\begin{equation}
\sqrt{2} = \frac{p_1p_2\dots p_n}{q_1q_2\dots q_d}.
\end{equation}

It follows that 

\begin{equation}\label{eq:sqrt2_4}
2 = \frac{p_1^2p_2^2\dots p_n^2}{q_1^2q_2^2\dots q_d^2}.
\end{equation}

As we know from the previous chapter,
two different prime numbers $p$ and $q$ squared 
(or raised to any integer)
do not result in two numbers that share factors.
The factorisation of $p^n$ is just $p^n$ and that
of $q^n$ is just $q^n$. They are coprime to each other.
The fraction in equation \ref{eq:sqrt2_4}, thus,
cannot be an integer.
There is only one way for such a fraction to
result in an integer, \viz\, when it is an integer itself.
From this follows that,
if the root of an integer is not an integer itself,
it is not a rational number either.

But, if $\sqrt{2}$ is not a rational number,
a number that can be represented as the fraction
of two integers, what the heck is it then?
\section{$\Phi$}
\ignore{
\begingroup\par\noindent\advance\leftskip\mathindent\(
\begin{pboxed}\SaveRestoreHook
\column{B}{@{}>{\hspre}l<{\hspost}@{}}%
\column{3}{@{}>{\hspre}l<{\hspost}@{}}%
\column{E}{@{}>{\hspre}l<{\hspost}@{}}%
\>[B]{}\mathbf{module}\;\Conid{Phi}{}\<[E]%
\\
\>[B]{}\mathbf{where}{}\<[E]%
\\
\>[B]{}\hsindent{3}{}\<[3]%
\>[3]{}\mathbf{import}\;\Conid{Natural}{}\<[E]%
\\
\>[B]{}\hsindent{3}{}\<[3]%
\>[3]{}\mathbf{import}\;\Conid{Quoz}{}\<[E]%
\ColumnHook
\end{pboxed}
\)\par\noindent\endgroup\resethooks
}

\section{$\pi$}
\ignore{
\begingroup\par\noindent\advance\leftskip\mathindent\(
\begin{pboxed}\SaveRestoreHook
\column{B}{@{}>{\hspre}l<{\hspost}@{}}%
\column{3}{@{}>{\hspre}l<{\hspost}@{}}%
\column{E}{@{}>{\hspre}l<{\hspost}@{}}%
\>[B]{}\mathbf{module}\;\Conid{Pi}{}\<[E]%
\\
\>[B]{}\mathbf{where}{}\<[E]%
\\
\>[B]{}\hsindent{3}{}\<[3]%
\>[3]{}\mathbf{import}\;\Conid{Natural}{}\<[E]%
\\
\>[B]{}\hsindent{3}{}\<[3]%
\>[3]{}\mathbf{import}\;\Conid{Quoz}{}\<[E]%
\ColumnHook
\end{pboxed}
\)\par\noindent\endgroup\resethooks
}

\section{$e$}
\ignore{
\begingroup\par\noindent\advance\leftskip\mathindent\(
\begin{pboxed}\SaveRestoreHook
\column{B}{@{}>{\hspre}l<{\hspost}@{}}%
\column{3}{@{}>{\hspre}l<{\hspost}@{}}%
\column{E}{@{}>{\hspre}l<{\hspost}@{}}%
\>[B]{}\mathbf{module}\;\Conid{E}{}\<[E]%
\\
\>[B]{}\mathbf{where}{}\<[E]%
\\
\>[B]{}\hsindent{3}{}\<[3]%
\>[3]{}\mathbf{import}\;\Conid{Natural}{}\<[E]%
\\
\>[B]{}\hsindent{3}{}\<[3]%
\>[3]{}\mathbf{import}\;\Conid{Quoz}{}\<[E]%
\ColumnHook
\end{pboxed}
\)\par\noindent\endgroup\resethooks
}

\section{$\gamma$}
\ignore{
\begingroup\par\noindent\advance\leftskip\mathindent\(
\begin{pboxed}\SaveRestoreHook
\column{B}{@{}>{\hspre}l<{\hspost}@{}}%
\column{3}{@{}>{\hspre}l<{\hspost}@{}}%
\column{E}{@{}>{\hspre}l<{\hspost}@{}}%
\>[B]{}\mathbf{module}\;\Conid{Gamma}{}\<[E]%
\\
\>[B]{}\mathbf{where}{}\<[E]%
\\
\>[B]{}\hsindent{3}{}\<[3]%
\>[3]{}\mathbf{import}\;\Conid{Natural}{}\<[E]%
\\
\>[B]{}\hsindent{3}{}\<[3]%
\>[3]{}\mathbf{import}\;\Conid{Quoz}{}\<[E]%
\ColumnHook
\end{pboxed}
\)\par\noindent\endgroup\resethooks
}

\section{Arithmetic with Real Numbers}
\section{Representation of Real Numbers} % natural numbers and fractions
\section{$\mathbb{R}$}
\section{Real Factorials}
\section{Real Binomial Coefficients}
\section{The Continuum}
\section{Review of the Number Zoo}


\end{document}
