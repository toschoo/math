%% =======================================================
%% (c) Tobias Schoofs
%% =======================================================
%% Math 4 Programmers - The Inverse Element
%% =======================================================

% Plain Style
\documentclass[tikz]{scrreprt}

%% ODER: format ==         = "\mathrel{==}"
%% ODER: format /=         = "\neq "
%
%
\makeatletter
\@ifundefined{lhs2tex.lhs2tex.sty.read}%
  {\@namedef{lhs2tex.lhs2tex.sty.read}{}%
   \newcommand\SkipToFmtEnd{}%
   \newcommand\EndFmtInput{}%
   \long\def\SkipToFmtEnd#1\EndFmtInput{}%
  }\SkipToFmtEnd

\newcommand\ReadOnlyOnce[1]{\@ifundefined{#1}{\@namedef{#1}{}}\SkipToFmtEnd}
\usepackage{amstext}
\usepackage{amssymb}
\usepackage{stmaryrd}
\DeclareFontFamily{OT1}{cmtex}{}
\DeclareFontShape{OT1}{cmtex}{m}{n}
  {<5><6><7><8>cmtex8
   <9>cmtex9
   <10><10.95><12><14.4><17.28><20.74><24.88>cmtex10}{}
\DeclareFontShape{OT1}{cmtex}{m}{it}
  {<-> ssub * cmtt/m/it}{}
\newcommand{\texfamily}{\fontfamily{cmtex}\selectfont}
\DeclareFontShape{OT1}{cmtt}{bx}{n}
  {<5><6><7><8>cmtt8
   <9>cmbtt9
   <10><10.95><12><14.4><17.28><20.74><24.88>cmbtt10}{}
\DeclareFontShape{OT1}{cmtex}{bx}{n}
  {<-> ssub * cmtt/bx/n}{}
\newcommand{\tex}[1]{\text{\texfamily#1}}	% NEU

\newcommand{\Sp}{\hskip.33334em\relax}


\newcommand{\Conid}[1]{\mathit{#1}}
\newcommand{\Varid}[1]{\mathit{#1}}
\newcommand{\anonymous}{\kern0.06em \vbox{\hrule\@width.5em}}
\newcommand{\plus}{\mathbin{+\!\!\!+}}
\newcommand{\bind}{\mathbin{>\!\!\!>\mkern-6.7mu=}}
\newcommand{\rbind}{\mathbin{=\mkern-6.7mu<\!\!\!<}}% suggested by Neil Mitchell
\newcommand{\sequ}{\mathbin{>\!\!\!>}}
\renewcommand{\leq}{\leqslant}
\renewcommand{\geq}{\geqslant}
\usepackage{polytable}

%mathindent has to be defined
\@ifundefined{mathindent}%
  {\newdimen\mathindent\mathindent\leftmargini}%
  {}%

\def\resethooks{%
  \global\let\SaveRestoreHook\empty
  \global\let\ColumnHook\empty}
\newcommand*{\savecolumns}[1][default]%
  {\g@addto@macro\SaveRestoreHook{\savecolumns[#1]}}
\newcommand*{\restorecolumns}[1][default]%
  {\g@addto@macro\SaveRestoreHook{\restorecolumns[#1]}}
\newcommand*{\aligncolumn}[2]%
  {\g@addto@macro\ColumnHook{\column{#1}{#2}}}

\resethooks

\newcommand{\onelinecommentchars}{\quad-{}- }
\newcommand{\commentbeginchars}{\enskip\{-}
\newcommand{\commentendchars}{-\}\enskip}

\newcommand{\visiblecomments}{%
  \let\onelinecomment=\onelinecommentchars
  \let\commentbegin=\commentbeginchars
  \let\commentend=\commentendchars}

\newcommand{\invisiblecomments}{%
  \let\onelinecomment=\empty
  \let\commentbegin=\empty
  \let\commentend=\empty}

\visiblecomments

\newlength{\blanklineskip}
\setlength{\blanklineskip}{0.66084ex}

\newcommand{\hsindent}[1]{\quad}% default is fixed indentation
\let\hspre\empty
\let\hspost\empty
\newcommand{\NB}{\textbf{NB}}
\newcommand{\Todo}[1]{$\langle$\textbf{To do:}~#1$\rangle$}

\EndFmtInput
\makeatother
%

%% =======================================================
%% (c) Tobias Schoofs
%% =======================================================
%% Commands 4 Programmers
%% =======================================================

%include lhs2TeX.fmt
%include lhs2TeX.sty

\usepackage[pdftex]{graphicx}
\usepackage{ucs}
\usepackage[utf8x]{inputenc} 
\usepackage{tabto}
%\usepackage[T1]{fontenc} 
\usepackage[portuguese, english]{babel}
\usepackage{amsfonts}

\usepackage{amsmath}
\usepackage{amssymb}
\usepackage{amsthm}
\usepackage{amscd}

\usepackage{siunitx}

\usepackage{tikz}
\usepackage{pgfplots}
%\usetikzlibrary{graphs,graphdrawing,arrows.meta}
\usetikzlibrary{graphs} % graphdrawing,arrows.meta}
%\usegdlibrary{trees}

\long\def\ignore#1{}

\newcommand{\acronym}[1]{\textsc{#1}}

\newcommand{\term}[1]{\textit{#1}}
\newcommand{\latin}[1]{\textit{#1}}
\newcommand{\speech}[1]{\textit{#1}}

\newcommand{\ie}{\textit{i.e.}}
\newcommand{\eg}{\textit{e.g.}}
\newcommand{\etc}{\textit{etc.}}
\newcommand{\viz}{\textit{viz.}}

\newcommand{\haskell}[1]{{\ttfamily #1}}

\newcommand{\Rom}[1]{\uppercase\expandafter{\romannumeral #1\relax}}

\DeclareRobustCommand{\stirlingOne}{\genfrac[]{0pt}{}}
\DeclareRobustCommand{\stirlingTwo}{\genfrac\{\}{0pt}{}}
\DeclareRobustCommand{\eulerianOne}{\genfrac<>{0pt}{}}
\newcommand{\eulerianTwo}[2]{\left\langle\!\!\left\langle{#1 \atop #2}\right\rangle\!\!\right\rangle}

\newcommand{\connect}[2]{
  \draw [-,color=black] (#1) to (#2)
}

% \renewcommand{\gcd}{\textsc{gcd}}


\begin{document}
\setlength{\parindent}{0pt}
\setlength{\parskip}{8pt}

\chapter{Elliptic Curves} % c10
\section{Geometry Intuition} 
\ignore{
\begingroup\par\noindent\advance\leftskip\mathindent\(
\begin{pboxed}\SaveRestoreHook
\column{B}{@{}>{\hspre}l<{\hspost}@{}}%
\column{E}{@{}>{\hspre}l<{\hspost}@{}}%
\>[B]{}\mathbf{module}\;\Conid{ECGeometry}{}\<[E]%
\\
\>[B]{}\mathbf{where}{}\<[E]%
\ColumnHook
\end{pboxed}
\)\par\noindent\endgroup\resethooks
}

Before we go on with more theoretical topics,
let us examine another application of the quite
abstract theory of algebra we have studied
in the previous chapter, namely 
elliptic curves cryptography. It should be
mentioned that linear algebra is by far 
not the only topic relevant in this context.
In fact, important aspects of the theory
of elliptic curves require the understanding
of function analysis -- where it actually
comes from -- but here we will focus on 
algebra and group theory.

Elliptic Curves (\acronym{ec}) provide the mathematical
background for variants of public key cryptography.
This kind of cryptography is being developed
since the eighties, but it took a while until it
was accepted by industry. 
It is the main public key cryptographic scheme today.
Its acceptance was accelerated by the smartphone boom.
In smartphones and other devices with restricited
resources, classic cryptographic schemes are not
very practical. Their drawback is
the computational overhead resulting from key size.
Many cryptoanalytical attacks on classic
cryptography are known that force cryptographers
to use huge keys. To achieve 128-bit security 
with \acronym{rsa}, we need keys with at least
\num{3072} bits. The same level of security
can be reached with \acronym{ec} cryptography, 
according to known attacks today,
with 256 bits. A huge improvement!

\acronym{ec} cryptography is different from
classic cryptography in various respects.
First, it includes much more math.
That is to say, it does not include 
theory from only one or two branches
of mathematics like number theory
in classic cryptography,
but from many different branches.
This has huge impact on cryptoanalysis.
Hidden attacks may linger around 
in apparently remote fields of mathematics
that we did not account for.
However, the theory surrounding \acronym{ec}
is very well understood today and, as said,
it is the mainline cryptography today.

Second, the basic means, especially the group
we need for public key cryptography, are
much more ``engineered'' than in classic
cryptography. Classic schemes are based
mainly on modular arithmetic, which was
well known centuries before anyone thought
of cryptography. The groups found in modular
arithmetic, in particular the multiplicative
group, was then used to define cryptographic tools.
In elliptic curves, there are no such groups
``by nature''. They are constructed on the
curves with the purpose to use them in cryptography.
Therefore, \acronym{ec} may sometimes feel a bit
artificial. It is important to understand that
the group we define on the curves is defined
voluntarily according to our purpose.
When we speak of \emph{point addition} in this
context, one must not confuse this operation
with the arithmetic operation of addition.
It is something totally different.

Anyway, what are elliptic curves in the first place?
Elliptic curves are polynomials that were
intensively studied in the late $19^{th}$ century,
especially by German mathematician Karl Weierstrass (1815 -- 1897),
who was of huge importance in the sound fundamentation
of analysis. We will meet him again in the third part.
He studied polynomials of the form

\begin{equation}
  y^2 = x^3 + ax + b,
\end{equation}

which is said to be in \emph{Weierstrass form}.
Obviously, we can easily transform this equation
into a form that looks more like something that
can be computed, namely:

\begin{equation}
  y = \sqrt{x^3 + ax + b}.
\end{equation}

But be careful! Weierstrass polynomials are not functions,
at least not in $\mathbb{R}$, since there is not exactly
one $y$ for each $x$. When the expression $x^3 + ax + b$
becomes negative, there is, in the world of real numbers, 
no solution for the right-hand side of the equation.
That is, there is no $y$ for $x$ that cause that expression
to be negative.

This is quite obvious, when we look at the geometric
interpretation of that polynomial. It looks -- more or less --
like in the following sketch:

\begin{center}
\begin{tikzpicture}
   \draw [<->] (-3,0) -- (3,0);
   \draw [<->] (0,-3) -- (0,3);
   \draw [teal, 
scale=0.5,domain=-1.769292354238631:3,variable=\x,smooth,samples=250]
       plot ({\x}, {sqrt(\x*\x*\x - 2*\x + 2)});
   \draw [teal, 
scale=0.5,domain=-1.769292354238631:3,variable=\x,smooth,samples=250]
       plot ({\x}, {-sqrt(\x*\x*\x - 2*\x + 2)});
\end{tikzpicture}
\end{center}

The exact shape depends on the coefficients $a$ and $b$.
The bubble on the left may sometimes be a circle or ellipse
completely separated from the ``tail'' on the right;
it may, in other cases, be less clearly distinguished
from the tail on the right, forming just a tiny bulge in the tail.

Anyhow, the point is that the curve ``ends'' on the left-hand side
for some $x < 0$. More precisely, it ends where $x^3$, for a negative
value, becomes greater than $ax + b$, because, then, the whole 
expression becomes negative and there is no real square root for it
anymore.

We will now start to construct a group on this kind of curves.
We call it an \emph{additional group}, but be aware that this
is not addition in the sense of the arithmetic operation.
It has nothing to do with that! Is a way to combine points
with each other that can be captured in a  -- more or less -- 
simple formula. We will start by giving a geometric interpretation
of this operation. This will help getting an intuition.
But, again, be aware that we are not dealing with geometry.
We will soon deviate from geometry and talk about curves
in a quite abstract way.

The following sketch shows an elliptic curve
with three points $P$, $Q$ and $R$,
all coloured in red.
These points are in the relation
$P + Q = R$.
The meaning of this operation is indicated by
the lines:

\begin{center}
\begin{tikzpicture}
   \draw [<->] (-4,0) -- (4,0);
   \draw [<->] (0,-4) -- (0,4);
   \draw [teal, 
          scale=0.75,domain=-1.769292354238631:3.5,variable=\x,smooth,samples=250]
       plot ({\x}, {sqrt(\x*\x*\x - 2*\x + 2)});
   \draw [teal, 
          scale=0.75,domain=-1.769292354238631:3.5,variable=\x,smooth,samples=250]
       plot ({\x}, {-sqrt(\x*\x*\x - 2*\x + 2)});
   \draw [red,fill=red] (-1.05,1.1) circle (1.5pt);
   \node [red,font=\small,anchor=south east] (p) at (-1.2,0.9) {P};
   \draw [red,fill=red] (0.32,-0.82) circle (1.5pt);
   \node [red,font=\small,anchor=north east] (q) at (0.5,-0.95) {Q};
   \draw [gray, 
          scale=0.75,domain=-3.5:3.5,variable=\x,smooth,samples=15]
         plot ({\x}, {-1.407*\x-0.5});
   \draw [black,fill=black] (2.17,-3.42) circle (1.5pt);
   \node [font=\small,anchor=south west] (q) at (2.2,-3.4) {R'};
   \draw [dotted] (2.17,-3.9) -- (2.17,3.9);
   \draw [red,fill=red] (2.17,3.42) circle (1.5pt);
   \node [red,font=\small,anchor=north west] (q) at (2.2,3.5) {R};
\end{tikzpicture}
\end{center}

When adding two points $P$ and $Q$ on an elliptic curve, 
we draw a straight line through them (the grey one).
From the nature of the elliptic curve, it is obvious
that the straight line will meet the curve once again.
At that cross point, we draw a helper point, $R'$.
Then we draw another line (the dotted one) that goes
straight up crossing $R'$. This line will meet the curve
again, namely at a point with the same $x$ coordinate,
but with the inverse of the $y$ coordinate $-y$.
That point is $R$, the result of $P + Q$.

You see that this operation has in fact nothing to do
with arithmetic addition. It is an arbitrary construction
to relate three different points.

The straight line is defined as:

\begin{equation}
l = mx + c,
\end{equation}

where $m$ is the slope and $c$ the $y$-intercept.

\begin{equation}
m = \frac{y_Q - y_P}{x_Q - x_P} 
\end{equation}

\begin{equation}
c = y_P - mx_P
\end{equation}

Now we set the two formulas equal:

\begin{equation}  
(mx + c)^2 = x^3 + ax + b, 
\end{equation}  

which is equivalent to

\begin{equation}  
x^3 + ax + b - (mx + c)^2 = 0
\end{equation}  

We know already that $x_P$ and $x_Q$ are intersections and,
therefore, know that $x - x_P$ and $x - x_Q$ are factors
of the polynomial above.
We postulate that $x - x_R$ is also a factor and, hence,
get

\begin{equation}  
x^3 + ax + b - (mx + c)^2 = (x - x_P)(x - x_Q)(x - x_R), 
\end{equation}  

which is the same as

\[
x^3 + (x_P + x_Q + x_R)x^2 + (x_Px_Q + x_Px_R + x_Qx_R)x - x_Px_Qx_R.
\]

We can derive

\begin{equation}
-m^2 = -x_P - x_Q - x_R
\end{equation}

and, hence:

\begin{equation}
x_R = m^2 -x_P - x_Q.
\end{equation}
\section{Projective Geometry} 
\section{EC modulo a Prime} 
\section{EC Crypto Systems} 
\section{Cryptoanalysis} 
\section{Mr. Frobenius} 
\section{Mr. Schoof} 
\section{Mr. Elkies and Mr. Aktin} 
\section{EC in Practice} 


\end{document}
